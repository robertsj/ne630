\documentclass[aspectratio=1610]{beamer}
\usepackage[T1]{fontenc}
\usetheme{wildcat}
\usetikzlibrary{arrows.meta,angles,quotes,calc,intersections,positioning}

\usepackage{amsmath,amssymb,amsfonts}
\usepackage{booktabs}
\usepackage{relsize}
\usepackage{pgfplots}
\pgfplotsset{compat=1.16}
\usepackage{array}
\usepackage{siunitx}
\usepackage{cancel}
\usepackage{jupyter}
\usepackage{minted}
\usepackage{xfrac}
\usepackage{mathtools}

\let\oldfootnotesize\footnotesize
\renewcommand*{\footnotesize}{\oldfootnotesize\tiny}

\def\mathdefault#1{#1}
\everymath=\expandafter{\the\everymath\displaystyle}


\title{Semi-Analytical Solution of the \\ 
       Multigroup Diffusion Equations in \\ Piecewise Homogeneous Slabs\\
       {\small\it NE 630 - Appendix Z}}

\date{\input{term.txt} \\ {\footnotesize Git SHA: \input{git_sha.txt}}}

\author{Jeremy Roberts}


\definecolor{ksupurple}{HTML}{512888}
\definecolor{orange}{HTML}{CA7C1B}
\definecolor{skyblue}{RGB}{180,220,255}

% --- Shortcuts ---
\newcommand{\br}{\mathbf{r}}
\newcommand{\bO}{\hat{\Omega}}
\newcommand{\bn}{\hat{\mathbf{n}}}
\newcommand{\jj}{\mathbf{j}}
\newcommand{\angflux}{\psi}
\newcommand{\dV}{\,d^3\br}
\newcommand{\dE}{\,dE}
\newcommand{\dO}{\,d\hat{\Omega}}
\newcommand{\ds}{\,d\mathbf{S}}
\newcommand{\mnote}[1]{\marginpar{\raggedright\footnotesize\color{ksupurple}{#1}}}

% --- Common shortcuts ---
\newcommand{\flux}{\phi}
\newcommand{\JJ}{\mathbf{J}}
\newcommand{\dd}{\,\mathrm{d}}
\newcommand{\SigT}{\Sigma_t}
\newcommand{\SigA}{\Sigma_a}
\newcommand{\SigS}{\Sigma_s}
\newcommand{\SigF}{\Sigma_f}
\newcommand{\nuSigF}{\nu\Sigma_f}
\newcommand{\D}{D}
\newcommand{\Ldiff}{L}
\newcommand{\xhat}{\hat{\mathbf{x}}}
\newcommand{\rhat}{\hat{\mathbf{r}}}

\begin{document}

\begin{frame}
\titlepage
\end{frame}
 
 
%%%%%%%%%%%%%%%%%%%%%%%%%%%%%%%%%%%%%%%%%%%%%%%%%%%
\begin{frame}{Primary Objective}

Students will be able to 

\vfill

\begin{quote}
\textcolor{wcprimary}{develop solutions for the multigroup diffusion equation in 
slab geometry for systems with multiple homogeneous regions subject to continuity conditions and boundary conditions.
}
\end{quote}


\end{frame}


%%%%%%%%%%%%%%%%%%%%%%%%%%%%%%%%%%%%%%%%%%%%%%%%%%%%%%%%%%%%%%%%%%%%%%%%%%%%%%
\begin{frame}[fragile]{Review: Multigroup Diffusion}

The multigroup diffusion equations for a slab comprised of $R$ homogeneous regions are 
\begin{align*}
\tag{MGD}
 - D_{g}^i \frac{d^2 \phi_{g}^i}{dx^2} &
   + \Sigma_{t,g}^i \, \phi_{g}^i(x) \\
  &= \sum_{g'=1}^{G} \Sigma_{s,g\gets g'}^i \phi_{g'}^i(x) 
   + \frac{\chi^i_g}{k_{\text{eff}}} \sum_{g'=1}^{G} \bar{\nu}\Sigma_{f, g'}^i \phi_{g'}^i(x) 
   +  S_{g}^i \, , \\
   g &= 1, 2, \ldots, G \, , \qquad i = 1, 2, \ldots, R \, , 
\end{align*}
subject to the continuity conditions
\begin{align*}
\tag{CC}
 \phi_g^i(x_{i+\frac{1}{2}}) &=  \phi_g^{i+1}(x_{i+\frac{1}{2}}) \\
 -D^i_g \frac{d\phi_g^i}{dx}\big |_{x = x_{i+\frac{1}{2}}} 
   &=  -D^{i+1}_g \frac{d\phi_g^{i+1}}{dx}\big |_{x = x_{i+\frac{1}{2}}} \, , \quad i = 1 \ldots R-1 \, ,
\end{align*}
and appropriate boundary conditions.  For a source-driven problkem, $k_{\text{eff}} = 1$; otherwise, it is the multiplication factor.


\end{frame}


\begin{frame}[fragile]{The Process}

The basic procedure for solving Eq. (MGD) is straightforward.  For a given region, rewrite the equations as 

\begin{align*}
 \frac{d^2 \boldsymbol{\phi}_i}{dx} &
   - \mathbf{B}^2_i \boldsymbol{\phi}_i(x) 
   =  -\mathbf{D}_i^{-1} \mathbf{S}_i \, ,
\end{align*}
where $\ldots$.

\end{frame}


\begin{frame}[fragile]{Regionwise Matrix Form}
For region $i=1,\dots,R$, stack the group fluxes into $\boldsymbol\phi_i(x)\in\mathbb{R}^G$ and define
\[
\mathbf{D}_i=\mathrm{diag}(D_1^i,\dots,D_G^i),\quad
\boldsymbol\Sigma_{t,i}=\mathrm{diag}(\Sigma^i_{t,1},\dots,\Sigma^i_{t,G}),
\]
\[
\boldsymbol\Sigma_{s,i}=[\,\Sigma^i_{s,g\gets g'}\,]_{g,g'=1}^G,\qquad
\mathbf{F}_i=\frac{1}{k_{\text{eff}}}\,\boldsymbol\chi_i\,(\bar\nu\boldsymbol\Sigma_{f,i})^{\!\top},
\]
\[
\mathbf{S}_i=\begin{bmatrix}S_1^i\\[-2pt]\vdots\\S_G^i\end{bmatrix},\qquad
\mathbf{R}_i=\boldsymbol\Sigma_{t,i}-\boldsymbol\Sigma_{s,i}-\mathbf{F}_i,\qquad
\mathbf{B}_i^2=\mathbf{D}_i^{-1}\mathbf{R}_i.
\]
Then (MGD) in region $i$ becomes the constant-coefficient ODE
\[
\mathbf{D}_i\,\frac{d^2\boldsymbol\phi_i}{dx^2}-\mathbf{R}_i\,\boldsymbol\phi_i(x)=-\,\mathbf{S}_i
\qquad\Longleftrightarrow\qquad
\frac{d^2\boldsymbol\phi_i}{dx^2}-\mathbf{B}_i^{2}\boldsymbol\phi_i(x)=-\,\mathbf{D}_i^{-1}\mathbf{S}_i.
\]


\end{frame}

\begin{frame}[fragile]{Decouple by Modes}

\textbf{Diagonalize the pair $(\mathbf{R}_i,\mathbf{D}_i)$:}
solve the generalized eigenproblem
\[
\mathbf{R}_i\,\mathbf{V}_i=\mathbf{D}_i\,\mathbf{V}_i\,\boldsymbol\Lambda_i,
\qquad
\text{scale so that }\;\mathbf{V}_i^{\top}\mathbf{D}_i\mathbf{V}_i=\mathbf{I}.
\]
 
\vfill 

\textbf{Go to modal space:} write $\boldsymbol\phi_i=\mathbf{V}_i\,\boldsymbol\psi_i$ and multiply the ODE by $\mathbf{V}_i^{\top}$:
\[
\boldsymbol\psi_i''(x)-\boldsymbol\Lambda_i\,\boldsymbol\psi_i(x)
= -\,\mathbf{V}_i^{\top}\mathbf{S}_i
\;\;\Longrightarrow\;\;
 \;\psi_{im}''-\lambda_{im}\psi_{im}=-s_{im},
\]
with $s_{im}=(\mathbf{V}_i^{\top}\mathbf{S}_i)_m$ and $\kappa_{im}=\sqrt{\lambda_{im}}$.

\vfill 

\textbf{Closed-form per mode ($y=x-x_{i,c}$):}
\[
\psi_{im}(y)=A_{im}\cosh(\kappa_{im}y)+B_{im}\sinh(\kappa_{im}y)+\frac{s_{im}}{\lambda_{im}}
\quad(\lambda_{im}>0).
\]
If a rare $\lambda_{im}=0$ occurs: $\psi_{im}''=-s_{im}$ gives a quadratic particular $-\tfrac{s_{im}}{2}y^2$.

\vfill 

\textbf{From modes to groups:} $\displaystyle \boldsymbol\phi_i(y)=\mathbf{V}_i\!\left[
\mathbf{A}_i\!\odot\!\cosh(\boldsymbol\kappa_i y)+
\mathbf{B}_i\!\odot\!\sinh(\boldsymbol\kappa_i y)+
\boldsymbol\Lambda_i^{-1}\mathbf{V}_i^{\top}\mathbf{S}_i\right]\!,
$
where $\boldsymbol\kappa_i=\mathrm{diag}(\kappa_{i1},\ldots,\kappa_{iG})$ and $\odot$ is elementwise.

\vfill

{\small \emph{Why cosh/sinh?} They are numerically stable on symmetric intervals $y\in[-\ell_i/2,\ell_i/2]$ and keep things real when $\lambda_{im}>0$ (use $\cos/\sin$ if $\lambda_{im}<0$).}

\end{frame}


\begin{frame}[fragile]{Glue regions \& apply BCs $\Rightarrow$ one linear solve}

\textbf{Interface matching (vector form at $x_{i+\frac12}$):}
\[
\boldsymbol\phi_i=\boldsymbol\phi_{i+1}\quad\text{and}\quad
\mathbf{D}_i\,\boldsymbol\phi_i'=\mathbf{D}_{i+1}\,\boldsymbol\phi_{i+1}'.
\]
In modal variables (evaluate at $y=\pm\,\ell_i/2$),
\[
\begin{aligned}
\boldsymbol\psi_i^R &= \mathbf{C}_i\,\mathbf{A}_i + \mathbf{S}_i\,\mathbf{B}_i + \boldsymbol\psi_i^p,
&
\boldsymbol\psi_i^{\prime R} &= \boldsymbol\kappa_i\!\left(\mathbf{S}_i\,\mathbf{A}_i + \mathbf{C}_i\,\mathbf{B}_i\right),\\
\boldsymbol\psi_i^L &= \mathbf{C}_i\,\mathbf{A}_i - \mathbf{S}_i\,\mathbf{B}_i + \boldsymbol\psi_i^p,
&
\boldsymbol\psi_i^{\prime L} &= \boldsymbol\kappa_i\!\left(-\mathbf{S}_i\,\mathbf{A}_i + \mathbf{C}_i\,\mathbf{B}_i\right),
\end{aligned}
\]
with $\mathbf{C}_i=\cosh(\boldsymbol\kappa_i\ell_i/2)$, $\mathbf{S}_i=\sinh(\boldsymbol\kappa_i\ell_i/2)$, $\boldsymbol\psi_i^p=\boldsymbol\Lambda_i^{-1}\mathbf{V}_i^{\top}\mathbf{S}_i$,
and $\boldsymbol\phi_{i}^{R/L}=\mathbf{V}_i\boldsymbol\psi_{i}^{R/L}$, $\boldsymbol\phi_{i}^{\prime\,R/L}=\mathbf{V}_i\boldsymbol\psi_{i}^{\prime\,R/L}$.

\textbf{Boundary conditions (examples, per group):}
\[
\text{Vacuum:}\quad \alpha^-\,\boldsymbol\phi - \mathbf{D}_1\,\boldsymbol\phi'=\mathbf{0}\;\text{at left},\qquad
\alpha^+\,\boldsymbol\phi + \mathbf{D}_R\,\boldsymbol\phi'=\mathbf{0}\;\text{at right},
\]
(choose $\alpha^\pm$ to match your preferred extrapolated/Robin form), or reflective: $\boldsymbol\phi'=0$.


\end{frame}

\begin{frame}[fragile]{Glue regions \& apply BCs $\Rightarrow$ one linear solve}


\medskip
\textbf{Unknowns \& solve:} Collect all modal constants $\{\mathbf{A}_i,\mathbf{B}_i\}$ ($2GR$ unknowns).  
Interface conditions ($2G$ per internal interface) $+$ boundary conditions ($2G$ total) $\Rightarrow$ one linear system
\[
\mathbf{M}\,\mathbf{c}=\mathbf{b}.
\]
Solve once for a source-driven problem ($k_{\text{eff}}=1$). For an eigenvalue problem, include $\mathbf{F}_i/k$ in $\mathbf{R}_i$ and do a simple outer loop:
\begin{enumerate}\item pick $k^{(0)}$,
\item build $\mathbf{M}(k^{(n)})$, solve for $\boldsymbol\phi^{(n)}$,
\item update $k^{(n+1)}$ via a Rayleigh quotient / fission-source normalization,
\item repeat until $k,\boldsymbol\phi$ converge.
\end{enumerate}

\smallskip
\textbf{Practical tips.} Work in local $y\in[-\ell_i/2,\ell_i/2]$ to keep $\cosh/\sinh$ arguments modest; prefer the generalized eigensolve $(\mathbf{R}_i,\mathbf{D}_i)$ to avoid explicitly forming $\mathbf{D}_i^{-1}$; if complex conjugate pairs appear, keep real solutions by using $\cos/\sin$ forms.
\end{frame}

\end{document}

