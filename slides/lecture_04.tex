\documentclass[aspectratio=1610]{beamer}
\usepackage[T1]{fontenc}
\usetheme{wildcat}

\usepackage{amsmath,amssymb,amsfonts}
\usepackage{booktabs}
\usepackage{relsize}
\usepackage{pgfplots}
\pgfplotsset{compat=1.16}

\usepackage{array}

\usepackage{siunitx}

\let\oldfootnotesize\footnotesize
\renewcommand*{\footnotesize}{\oldfootnotesize\tiny}

\def\mathdefault#1{#1}
\everymath=\expandafter{\the\everymath\displaystyle}


\title{Neutron Attenuation \\
       {\small\it NE 630 - Lecture 4}}

\date{\input{term.txt} \\ {\footnotesize Git SHA: \input{git_sha.txt}}}

\author{Jeremy Roberts}


\definecolor{ksupurple}{HTML}{512888}
\definecolor{orange}{HTML}{CA7C1B}

\begin{document}

\begin{frame}
\titlepage
\end{frame}
 
 
%%%%%%%%%%%%%%%%%%%%%%%%%%%%%%%%%%%%%%%%%%%%%%%%%%%
\begin{frame}{Primary Objective}

Students will be able to 

\vfill


\begin{quote}
\textcolor{wcprimary}{Characterize the attenuation of neutrons
in a parallel beam or emitted from a point source as a function 
of distance and material composition.}
\end{quote}

\vfill 

\end{frame}

%%%%%%%%%%%%%%%%%%%%%%%%%%%%%%%%%%%%%%%%%%%%%%%%%%%%%%%%%%%%%%%%%%%%%%%%%%%%%%
\begin{frame}{Review}

Last time, we reviewed radioactivate decay and modeled how the number of nuclei $N_X$ of 
of some species $X$ changes in time as 

\begin{align*}
 \frac{d N_X}{dt} &= \textcolor{wcalerted}{\text{loss rate}} + \textcolor{wcprimary}{\text{production rate}} \\
                  &= \textcolor{wcalerted}{-\lambda_X N_X(t) + \ldots} + \textcolor{wcprimary}{R_X(t)} \, ,
\end{align*}
where the production $\textcolor{wcprimary}{R_X(t)}$ might represent an external source (e.g., production of a radioisotope like ${}^{99}$Mo in a reactor) or the decay of some other species $Y$ to $X$.  

\vfill 

The only losses considered now are from decay; however, we'll find that 
some nuclides are also lost through transmutation (e.g., we can irradiate ${}^{98}$Mo to produce ${}^{99}$Mo through neutron capture).
 
\pause 
\vfill 

{\bf Concept Check}: What are the units of   $dN_X/dt$,  $R_X(t)$, and $-\lambda_X N_X$?

\end{frame}


%%%%%%%%%%%%%%%%%%%%%%%%%%%%%%%%%%%%%%%%%%%%%%%%%%%%%%%%%%%%%%%%%%%%%%%%%%%%%%
\begin{frame}{Neutronics}
 
A major part of reactor physics is the {\it physics of neutrons}, sometimes 
called {\bf neutronics}.  We're now a little familiar with 
what neutrons do with nuclei, but how do they get from their place of birth
to such reactions?

\pause 
\vfill 

Some facts, assumptions, and implications : \\
\begin{tabular}{c >{\raggedright\arraybackslash}p{7cm} >{\raggedright\arraybackslash}p{6cm}}
 1. & neutrons are neutral \pause
    &  \textcolor{wcalerted}{no Coulombic interactions!} \\
    \pause
 2. & neutrons travel in straight lines between interactions \pause
    & \textcolor{wcalerted}{simulations are easy!} \\
    \pause
 3. & upon exiting an interaction, a neutron's direction and energy 
      are different \pause
    & \textcolor{wcalerted}{otherwise, did it even interact?} \\
    \pause
 4. & neutronic systems have enough neutrons such that {\bf averages} are 
      useful \pause
    &  \textcolor{wcalerted}{ {\bf continuum} models like neutron diffusion theory are justified!} \\
    \pause
 5. & the neutron (number) density in these systems is much smaller than that of the surrounding 
      medium \pause
    &  \textcolor{wcalerted}{we can safely ignore neutron-neutron interactions!} \\
    \pause
 6. & the surrounding medium is {\bf isotropic} \pause
    &  \textcolor{wcalerted}{on the average, neutron paths through a {\bf homogeneous} medium are independent of angle} \\
\end{tabular}



\end{frame}


%%%%%%%%%%%%%%%%%%%%%%%%%%%%%%%%%%%%%%%%%%%%%%%%%%%%%%%%%%%%%%%%%%%%%%%%%%%%%%
\begin{frame}{So How Dense is that Surrounding Medium?}

The {\bf number density} of some species $X$ is defined as 
\begin{equation*}
  \boxed{n_X  =  \frac{ \rho_X N_A }{M_X}} \, ,
\end{equation*}
where
\begin{itemize}
  \item $n_X$: number density [$N_X$~\si{\per\cubic\centi\meter}]
  \item $\rho_X$: mass density [\si{\gram} X \si{\per\cubic\centi\meter}]
  \item $N_A$: Avogadro’s number [$N_X$~\si{\per\mole}]
  \item $M_X$: molar mass [\si{\gram} X \si{\per\mole}]
\end{itemize}
and $N_X$ is to be read as ``number of [atoms|molecules|things] of X''.

\vfill 
\pause 

\only<2>{%
  \textcolor{wcalerted}{{\bf Important!} $N_X$ and X are rarely included in the units,
  but doing so consistently will help you to avoid unit mismatches.}
}

\only<3>{%
  {\bf Examples}: 
  \begin{enumerate}
   \item Compute $n_{{}^{27}\text{Al}}$ 
     ($\rho_{{}^{27}\text{Al}} = 2.7~\si{\gram} ~ \textcolor{wcalerted}{{}^{27}\text{Al}} ~ \si{ \per \cubic\centi\meter}$)
   \item Compute $n_{\text{H}}$ for our reactor's coolant by first 
     computing (a) $n_{\text{H}_2\text{O}}$ and (b) $\rho_{H}$.
  \end{enumerate}
}

\only<4>{%
  {\bf Answers}: 
  \begin{enumerate}
   \item $n_{{}^{27}\text{Al}} = 6.02\cdot 10^{22} \, \si{ \per \cubic\centi\meter}$
   \item $n_{\text{H}} = 6.69\cdot 10^{22} \, \si{ \per \cubic\centi\meter}$
  \end{enumerate}
}

\end{frame}


%%%%%%%%%%%%%%%%%%%%%%%%%%%%%%%%%%%%%%%%%%%%%%%%%%%%%%%%%%%%%%%%%%%%%%%%%%%%%%
\begin{frame}{What Do Neutrons See?}

\only<1>{
\begin{minipage}{0.48\linewidth}
  \centering
  \includegraphics[width=0.8\linewidth]{figures/cross_section_cartoon_3D_x0p00.pdf}
\end{minipage}%
\hfill
\begin{minipage}{0.48\linewidth}
  \centering
  \includegraphics[width=0.8\linewidth]{figures/cross_section_cartoon_2D_x0p00.pdf}
\end{minipage}
}

\only<2>{
\begin{minipage}{0.48\linewidth}
  \centering
  \includegraphics[width=0.8\linewidth]{figures/cross_section_cartoon_3D_x0p25.pdf}
\end{minipage}%
\hfill
\begin{minipage}{0.48\linewidth}
  \centering
  \includegraphics[width=0.8\linewidth]{figures/cross_section_cartoon_2D_x0p25.pdf}
\end{minipage}
}

\only<3>{
\begin{minipage}{0.48\linewidth}
  \centering
  \includegraphics[width=0.8\linewidth]{figures/cross_section_cartoon_3D_x0p50.pdf}
\end{minipage}%
\hfill
\begin{minipage}{0.48\linewidth}
  \centering
  \includegraphics[width=0.8\linewidth]{figures/cross_section_cartoon_2D_x0p50.pdf}
\end{minipage}
}

\only<4>{
\begin{minipage}{0.48\linewidth}
  \centering
  \includegraphics[width=0.8\linewidth]{figures/cross_section_cartoon_3D_x0p75.pdf}
\end{minipage}%
\hfill
\begin{minipage}{0.48\linewidth}
  \centering
  \includegraphics[width=0.8\linewidth]{figures/cross_section_cartoon_2D_x0p75.pdf}
\end{minipage}
}

\only<5>{
\begin{minipage}{0.48\linewidth}
  \centering
  \includegraphics[width=0.8\linewidth]{figures/cross_section_cartoon_3D_x1p00.pdf}
\end{minipage}%
\hfill
\begin{minipage}{0.48\linewidth}
  \centering
  \includegraphics[width=0.8\linewidth]{figures/cross_section_cartoon_2D_x1p00.pdf}
\end{minipage}
\pause 
{\bf A Sneak Peak}: If each orange circle has a cross sectional area of $\sigma$,
what does $6 \sigma / (\Delta_x \Delta_y \Delta_z)$ represent?  (6 is the number 
of nuclei!)
}
\end{frame}


%%%%%%%%%%%%%%%%%%%%%%%%%%%%%%%%%%%%%%%%%%%%%%%%%%%%%%%%%%%%%%%%%%%%%%%%%%%%%%
\begin{frame}{Counting Uncollided Neutrons}

Suppose that 
\begin{itemize}
 \item the slice contains $N$ nuclei
 \item each nucleus has a geometric cross sectional area of $\sigma$ [\si{cm^2}], which is called the {\bf microscopic cross section}
 \item neutrons that pass through the area covered by any nucleus must interact and, hence, are removed from the parallel beam
\end{itemize}

The number of neutrons 
impinging on the slice is $\Delta_y \Delta_z I_0$.  The number that emerge without interacting is

\begin{equation*}
\overbrace{I(\Delta_x)\, \Delta_y \Delta_z}^{\text{\# we keep}}
= \overbrace{I_0\,\Delta_y \Delta_z}^{\text{\# we start with}}
- \overbrace{I_0\,\Delta_y \Delta_z\;\underbrace{\vphantom{\Big(}\frac{N\sigma}{\Delta_y \Delta_z}}_{\text{blocked fraction}}}^{\text{\# we lose}} \, ,
\end{equation*}
or
\begin{equation*}
I(\Delta x) = I_0\!\left(1 - \frac{N\sigma}{\Delta_y \Delta_z}\right).
\end{equation*}
 
\end{frame}


%%%%%%%%%%%%%%%%%%%%%%%%%%%%%%%%%%%%%%%%%%%%%%%%%%%%%%%%%%%%%%%%%%%%%%%%%%%%%%
\begin{frame}{Neutron Attenuation}

On the {\it average}, we expect $N = n\,\Delta_y \Delta_z\,\Delta_x$, so 
 
\begin{equation*}
 \dfrac{N\sigma}{\Delta_y \Delta_z} = n\sigma\,\Delta x \equiv \Sigma\,\Delta x \, ,
\end{equation*}
where $\Sigma [\si{cm^{-1}}]$ is the {\bf macroscopic cross section}.  Thus,
\begin{equation*}
   I(\Delta_x) \approx I_0\!\left(1 - \Sigma\,\Delta x\right) \, ,
\end{equation*}
and, in the usual manner, we let $\Delta_x \to 0$ to define
\begin{equation*}
\boxed{
\frac{dI}{dx} = -\,\Sigma\, I(x)
\quad\longrightarrow\quad
I(x) = I_0\,e^{-\Sigma x} 
} \, .
\end{equation*}

\vfill 
\pause 

\only<2>{
\textcolor{wcalerted}{
{\bf Important!} The product $\Sigma I(x)$ plays the same role in attenuation as does $\lambda N(t)$ in radioactive decay.  Just as $1/\lambda$ is the expected time for a radioactive nucleus to decay, $1/\Sigma$ is the expected distance a neutron travels before 
interaction, i.e., the {\bf mean-free path}.}
}
 

\only<3>{
{\bf Example}: Consider a thermal-neutron beam of intensity $I_0 = 10^{6}  \,\si{\per\centi\meter\squared\per\second}$ incident on an aluminum foil of thickness 0.05\,\si{\centi\meter}. If the measured intensity on the other side of the foil is $0.9949 I_0$, what is $\Sigma$?  What is $\sigma$?
}
 
\only<4>{
{\bf Answer}: $\Sigma \approx 0.10 \, \si{\per\centi\meter}$ and $\sigma \approx 1.7$ b (assuming $n_{{}^{27}\text{Al}} $ from earlier)
}
\end{frame}


\end{document}

