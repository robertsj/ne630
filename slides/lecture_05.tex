\documentclass[aspectratio=1610]{beamer}
\usepackage[T1]{fontenc}
\usetheme{wildcat}

\usepackage{amsmath,amssymb,amsfonts}
\usepackage{booktabs}
\usepackage{relsize}
\usepackage{pgfplots}
\pgfplotsset{compat=1.16}

\usepackage{array}

\usepackage{siunitx}

\let\oldfootnotesize\footnotesize
\renewcommand*{\footnotesize}{\oldfootnotesize\tiny}

\def\mathdefault#1{#1}
\everymath=\expandafter{\the\everymath\displaystyle}


\title{Using Cross-Section Data \\
       {\small\it NE 630 - Lecture 5}}

\date{\input{term.txt} \\ {\footnotesize Git SHA: \input{git_sha.txt}}}

\author{Jeremy Roberts}


\definecolor{ksupurple}{HTML}{512888}
\definecolor{orange}{HTML}{CA7C1B}

\begin{document}

\begin{frame}
\titlepage
\end{frame}
 
 
%%%%%%%%%%%%%%%%%%%%%%%%%%%%%%%%%%%%%%%%%%%%%%%%%%%
\begin{frame}{Primary Objective}

Students will be able to 

\vfill


\begin{quote}
\textcolor{wcprimary}{Determine the probability that a neutron of energy $E$ undergoes a particular reaction.}
\end{quote}

\vfill 

\end{frame}

%%%%%%%%%%%%%%%%%%%%%%%%%%%%%%%%%%%%%%%%%%%%%%%%%%%%%%%%%%%%%%%%%%%%%%%%%%%%%%
\begin{frame}{Review}

Consider again the $\vec{x}$-directed neutron beam of 
intensity $I_0 \, \si{\per\centi\meter\squared\per\second}$ incident 
on a slice of material with volume $V = \Delta_x\Delta_y \Delta_z$.
The number of neutrons 
impinging on the slice is $\Delta_y \Delta_z I_0$, while the number that exit the slice is
\begin{equation*}
\overbrace{I(\Delta_x)\, \Delta_y \Delta_z}^{\text{\# we keep}}
= \overbrace{I_0\,\Delta_y \Delta_z}^{\text{\# we start with}}
- \overbrace{I_0\,\Delta_y \Delta_z\;\vphantom{\Big(}\frac{N\sigma}{\Delta_y \Delta_z}}^{\text{\# we lose}} \, .
\end{equation*}
\pause
We {\it expect} $N = n\,\Delta_x\Delta_y \Delta_z$ target nuclei in the slice, so  
\begin{equation*}
 \dfrac{N\sigma}{\Delta_y \Delta_z} = n\sigma\,\Delta_x \equiv \Sigma\,\Delta_x
    \quad \longrightarrow \quad 
 I(\Delta_x) \approx I_0 [ 1 - \Sigma\,\Delta_x ] \, ,
\end{equation*}
where $\sigma [\si{\centi\meter\squared}]$ is the {\bf microscopic cross section} and $\Sigma [\si{\per\centi\meter}]$ is the {\bf macroscopic cross section}. \pause  In the usual manner, we let $\Delta_x \to 0$ to define
\begin{equation*}
\boxed{
\frac{dI}{dx} = -\,\Sigma\, I(x)
\quad\longrightarrow\quad
I(x) = I_0\,e^{-\Sigma x} 
} \, ,
\end{equation*}
which describes the {\bf exponential attenuation} of neutrons through material interactions.  What about {\bf geometric attenuation}?

\end{frame}


%%%%%%%%%%%%%%%%%%%%%%%%%%%%%%%%%%%%%%%%%%%%%%%%%%%%%%%%%%%%%%%%%%%%%%%%%%%%%%
\begin{frame}{What Do Neutrons Do?}
 
Let $\sigma_x$ represent the microscopic cross section for a particular 
reaction (or interaction) $x$ that neutrons can induce in target nuclei.
What are the primary reactions of interest?

\pause 

\begin{table}
\begin{tabular}{ll}
\toprule
  $\sigma_x$ & reaction \\
\midrule
  $\sigma_n$ or  $\sigma_e$      & elastic scattering \\
  $\sigma_{n'}$ or $\sigma_{in}$  & inelastic scattering \\
  $\sigma_s$      & scattering (usually $\sigma_n + \sigma_{n'}$,  but can be ambiguous!) \\
  $\sigma_{2n}$    & $(n, 2n)$, and $3n, 4n, \ldots$ \\
\midrule
  $\sigma_{\gamma}$  & capture (i.e., $(n, \gamma)$) \\
  $\sigma_f$      & fission \\
  $\sigma_{\alpha}$  & $(n, \alpha)$ \\
  $\sigma_a$         & absorption ($= \sigma_{\gamma} + \sigma_f + \sigma_{\alpha} + \ldots$) \\
\midrule
  $\sigma_t$       & total (i.e., the sum of all individual $\sigma_x$'s) \\
\bottomrule
\end{tabular}
\end{table}

\end{frame}


%%%%%%%%%%%%%%%%%%%%%%%%%%%%%%%%%%%%%%%%%%%%%%%%%%%%%%%%%%%%%%%%%%%%%%%%%%%%%%
\begin{frame}{Reaction Rates}
 
Reactor physics analysis is driven by {\it reaction-rate densities} $R_x$ [reactions of type $x$ \si{\per\cubic\centi\meter\per\second}].  The total number of interations per unit volume and time is given by
\begin{equation*}
   R_t =  \Sigma_t \phi \, ,
\end{equation*}
where $\phi$ [\si{\per\centi\meter\squared\per\second}] is the flux of neutrons.  
\pause
We can decompose this total reaction rate density by nuclide and reaction:
\begin{equation*}
 \begin{split}
  R_t =   \Sigma_t \phi
      &= \left ( \sum_{i \in \text{nuclides}} \Sigma^i_t \right ) \phi  \\
    &= \left ( \sum_{i \in \text{nuclides}} n_i \sigma^i_t \right ) \phi \\
    &= \left ( \sum_{i \in \text{nuclides}} n_i \sum_{x \in [e, \gamma, \ldots]} \sigma^i_x \right ) \phi 
    = \sum_{i} \sum_{x} R^i_x  \, .
\end{split}
\end{equation*}

\end{frame}


%%%%%%%%%%%%%%%%%%%%%%%%%%%%%%%%%%%%%%%%%%%%%%%%%%%%%%%%%%%%%%%%%%%%%%%%%%%%%%
\begin{frame}{Interaction Probabilities}

If the reaction rates are sufficiently large, then ratios of partial rates
can be used to define various probabilities:

\begin{itemize}
  \item $\frac{R_x}{R_t}$ = fraction of all reactions that are $x$ (this answers the question, ``if a neutron interacts, what is the probability that $x$ is the interaction?'')
  
 \item $\frac{R^i_t}{R_t}$ = fraction of all reactions that are with nuclide $i$
  
 \item $\frac{R^i_x}{R^i_t}$ = ??
  \item $\frac{R^i_x}{R_x}$ = ??
 \item $\frac{R_f}{R_a}$ = ??
  \item $\bar{\nu}^{\text{U-235}} R_f^{\text{U-235}}$ = ??
  \item $\displaystyle\frac{\bar{\nu}^{\text{U-235}} R_f^{\text{U-235}} + \bar{\nu}^{\text{U-238}} R_f^{\text{U-238}}}{R_f}$ = ??
\end{itemize}

\end{frame}



\end{document}

