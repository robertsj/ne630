\documentclass[aspectratio=1610]{beamer}
\usepackage[T1]{fontenc}
\usetheme{wildcat}

\usepackage{amsmath,amssymb,amsfonts}
\usepackage{booktabs}
\usepackage{relsize}
\usepackage{pgfplots}
\pgfplotsset{compat=1.16}
\usepackage{array}
\usepackage{siunitx}

\usepackage{jupyter}

\let\oldfootnotesize\footnotesize
\renewcommand*{\footnotesize}{\oldfootnotesize\tiny}

\def\mathdefault#1{#1}
\everymath=\expandafter{\the\everymath\displaystyle}


\title{Understanding Cross-Section Data \\
       {\small\it NE 630 - Lecture 6}}

\date{\input{term.txt} \\ {\footnotesize Git SHA: \input{git_sha.txt}}}

\author{Jeremy Roberts}


\definecolor{ksupurple}{HTML}{512888}
\definecolor{orange}{HTML}{CA7C1B}

\begin{document}

\begin{frame}
\titlepage
\end{frame}
 
 
%%%%%%%%%%%%%%%%%%%%%%%%%%%%%%%%%%%%%%%%%%%%%%%%%%%
\begin{frame}{Primary Objective}

Students will be able to 

\vfill


\begin{quote}
\textcolor{wcprimary}{Access and inspect differential cross sections using OpenMC and map features of these cross sections to the underlying nuclear physics.}
\end{quote}

\vfill 


\end{frame}

%%%%%%%%%%%%%%%%%%%%%%%%%%%%%%%%%%%%%%%%%%%%%%%%%%%%%%%%%%%%%%%%%%%%%%%%%%%%%%
\begin{frame}{Review}

\begin{minipage}{0.55\linewidth}

A small sample of a 1-to-1 mixture of water and ${}^{235}$U is {\it irradiated
uniformly} in a flux equal to $\phi_0$~\si{\per\centi\meter\squared\per\second}. 

\vspace{0.5cm}

If a neutron \emph{interacts}, what is the probability that it...
\begin{itemize}
 \item elastically scatters off of \emph{any} nucleus? 
 \item elastically scatters off of H-1?
 \item is absorbed by U-235?
\end{itemize}
 
\vspace{0.5cm}


If the neutron is \emph{absorbed}, what is the probability that it
causes fission?



\end{minipage}
\hfill
\begin{minipage}{0.35\linewidth}
\begin{table}[t]
\begin{tabular}{ccc}
\toprule
nuclide & quantity & value \\
\midrule
H-1    & \(\sigma_e\) & 20 b \\
       & \(\sigma_{\gamma}\) & 0.1 b \\
O-16   & \(\sigma_e\) & 4 b \\
U-235  & \(\sigma_e\) & 15 b \\
       & \(\sigma_{\gamma}\) & 100 b\\
       & \(\sigma_{f}\) & 500 b \\
       & \(\bar{\nu}\) & 2.5 \\
\bottomrule
\end{tabular}
\end{table}

\end{minipage}


\pause
\vspace{0.5cm}

{\footnotesize {\it Answers: 0.09, 0.06, 0.91, and 0.83.}}

\end{frame}

%%%%%%%%%%%%%%%%%%%%%%%%%%%%%%%%%%%%%%%%%%%%%%%%%%%%%%%%%%%%%%%%%%%%%%%%%%%%%%
\begin{frame}[fragile]{Exploring Cross Sections with OpenMC}

We'll use OpenMC and its nuclear data libraries to
explore common features of cross sections and to produce 
enhanced versions of Figures 2.3--10 of FNRP.  


\begin{Shaded}
\begin{Highlighting}[]
\ImportTok{import}\NormalTok{ openmc}
\ImportTok{import}\NormalTok{ numpy }\ImportTok{as}\NormalTok{ np}
\ImportTok{import}\NormalTok{ matplotlib.pyplot }\ImportTok{as}\NormalTok{ plt}
\end{Highlighting}
\end{Shaded}

\vfill 

The OpenMC data is stored in a single \texttt{.h5} per nuclide. 
On Beocat, those files are located in this folder:
\begin{Shaded}
\begin{Highlighting}[]
\NormalTok{PATH }\OperatorTok{=} \StringTok{"/homes/jaroberts/openmc/data/neutron"}
\end{Highlighting}
\end{Shaded}

\end{frame}

%%%%%%%%%%%%%%%%%%%%%%%%%%%%%%%%%%%%%%%%%%%%%%%%%%%%%%%%%%%%%%%%%%%%%%%%%%%%%%
\begin{frame}{Preparing to Explore...}

We can use OpenMC's libraries to explore cross sections, but
it helps first to map the reactions we already know to their 
{\tt MT} identifiers shown below:

\begin{table}[t]
\centering
\begin{tabular}{cll}
\toprule
Reaction Symbol & ENDF MT Number(s) & Description \\
\midrule
$\sigma_{t}$            & 1        & Total cross section \\
$\sigma_{n}$, $\sigma_{e}$ & 2        & Elastic scattering \\
$\sigma_{n'}$, $\sigma_{\mathrm{in}}$ & 4, 51--91 & Inelastic scattering \\
$\sigma_{2n}$           & 16       & $(n,2n)$  \\
$\sigma_{\gamma}$       & 102      & Radiative capture $(n,\gamma)$ \\
$\sigma_{f}$            & 18       & Fission \\
$\sigma_{\alpha}$       & 107      & $(n,\alpha)$ \\
\midrule
$\bar{\nu}$ (total)     & 452      & Average total neutrons per fission \\
$\bar{\nu}_{d}$ (delayed) & 455    & Average delayed neutrons per fission \\
$\bar{\nu}_{p}$ (prompt)  & 456    & Average prompt neutrons per fission \\
\bottomrule
\end{tabular}
\end{table}


\end{frame}

%%%%%%%%%%%%%%%%%%%%%%%%%%%%%%%%%%%%%%%%%%%%%%%%%%%%%%%%%%%%%%%%%%%%%%%%%%%%%%
\begin{frame}[fragile]{${}^{235}$U Fission}

Data for a single nuclide, e.g., U-235, can be loaded via

\begin{Shaded}
\begin{Highlighting}[]
\NormalTok{U235 }\OperatorTok{=}\NormalTok{ openmc.data.IncidentNeutron.from_hdf5(PATH}\OperatorTok{+}\StringTok{"U235.h5"}\NormalTok{)}
\end{Highlighting}
\end{Shaded}

Any of the previous reactions (if it exists) can be accessed directly, e.g.,

\begin{Shaded}
\begin{Highlighting}[]
\NormalTok{U235.reactions[}\DecValTok{18}\NormalTok{]}
\end{Highlighting}
\end{Shaded}

\begin{verbatim}
<Reaction: MT=18 (n,fission)>
\end{verbatim}

The corresponding cross sections are stored in the dictionary 
\texttt{xs}, whose keys are temperatures (e.g., {\tt '294K'}) and whose values 
are callable functions of energy.  
For example, the following defines \texttt{σ\_f\_u235} as a function for the
${}^{235}$U fission cross section evaluated at 294K and plots it over energy:

\begin{Shaded}
\begin{Highlighting}[]
\NormalTok{σ_f_U235 }\OperatorTok{=}\NormalTok{ U235.reactions[}\DecValTok{18}\NormalTok{].xs[}\StringTok{"294K"}\NormalTok{]}
\NormalTok{E }\OperatorTok{=}\NormalTok{ np.logspace(}\OperatorTok{-}\DecValTok{2}\NormalTok{, }\DecValTok{7}\NormalTok{, }\DecValTok{10000}\NormalTok{)}
\NormalTok{plt.loglog(E, σ_f_U235(E))}
\end{Highlighting}
\end{Shaded}

 
\end{frame}

\begin{frame}[fragile]{${}^{235}$U Fission}


\begin{figure}[h]
    \centering
    \resizebox{0.95\textwidth}{!}{\input{figures/u235_fission.pgf}}
\end{figure}
 
Surely, this cross section is important! Moreover, this
cross section is a complicated function of energy that varies by several
orders of magnitude over 10 decades (powers of 10) of energy. The jagged
spikes in between about 1 eV and 2 keV are called \textbf{resonances}
and provide a unique connection to the underlying physics of compound
nucleus formation.

  
\end{frame}

\begin{frame}[fragile]{Potential Scattering and $1/v$ Absorption}

Let's back up and consider another important nuclide, ${}^{1}$H, and inspect its elastic-scattering and capture cross sections:
\begin{figure}[h]
    \centering
    \resizebox{1\textwidth}{!}{\input{figures/h1_xsec.pgf}}
\end{figure}
 
\begin{equation*}
 \sigma_{\gamma}(E) = \sqrt{E_0/E} \,\sigma_{\gamma}(E_0) \, 
\tag{FNRP 2.36}
\end{equation*}

  
\end{frame}


\begin{frame}{The Single-Level Formulas of Breit and Wigner}
 
The earliest (1930's) quantum mechanical work of Breit and Wigner led to these expressions for the capture and elastic scattering cross sections near some resonance energy (or {\it level}) $E_r$:

\begin{align*}
\tag{FNRP 2.41}
\sigma_{\gamma}(E) &= \sigma_0 \frac{\Gamma_{\gamma}}{\Gamma} \left (\frac{E_r}{E} \right)^{1/2} \frac{1}{1+ 4(E-E_r)^2/\Gamma^2} \\
%
\tag{FNRP 2.42}
\sigma_{n}(E) &= \sigma_0 \frac{\Gamma_{n}}{\Gamma}  \frac{1}{1+ 4(E-E_r)^2/\Gamma^2}\\
              & \quad + \sigma_0 \frac{2R}{\lambda_0} \frac{ 2(E-E_r)/\Gamma }{1 + 4 (E-E_r)^2/\Gamma^2} 
              + 4\pi R^2 \, .
\end{align*}
 
To reconstruct $\sigma_{\gamma}$ and $\sigma_{n}$ given multiple resonances, Eqs.~(2.41) and (2.42) can be evaluated for each resonance (with its own location $E_r$, peak $\sigma_0$, and partial widths $\Gamma_{\gamma}$ and $\Gamma_{n}$) and then summed.

\end{frame}



\end{document}

