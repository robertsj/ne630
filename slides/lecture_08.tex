\documentclass[aspectratio=1610]{beamer}
\usepackage[T1]{fontenc}
\usetheme{wildcat}
\usetikzlibrary{arrows.meta,angles,quotes,calc,intersections,positioning}

\usepackage{amsmath,amssymb,amsfonts}
\usepackage{booktabs}
\usepackage{relsize}
\usepackage{pgfplots}
\pgfplotsset{compat=1.16}
\usepackage{array}
\usepackage{siunitx}

\usepackage{jupyter}

\let\oldfootnotesize\footnotesize
\renewcommand*{\footnotesize}{\oldfootnotesize\tiny}

\def\mathdefault#1{#1}
\everymath=\expandafter{\the\everymath\displaystyle}


\title{The Reproduction Factor $\eta$ \\
       {\small\it NE 630 - Lecture 8}}

\date{\input{term.txt} \\ {\footnotesize Git SHA: \input{git_sha.txt}}}

\author{Jeremy Roberts}


\definecolor{ksupurple}{HTML}{512888}
\definecolor{orange}{HTML}{CA7C1B}

\begin{document}

\begin{frame}
\titlepage
\end{frame}
 
 
%%%%%%%%%%%%%%%%%%%%%%%%%%%%%%%%%%%%%%%%%%%%%%%%%%%
\begin{frame}{Primary Objective}

Students will be able to 

\vfill

\begin{quote}
\textcolor{wcprimary}{Explain in a ``big picture'' sense what the goals of thermal and fast reactor designs are and how they differ based on $\eta(E)$, $\xi$, $\Sigma_s$ and $\Sigma_a$}
\end{quote}

\vfill 

\end{frame}

%%%%%%%%%%%%%%%%%%%%%%%%%%%%%%%%%%%%%%%%%%%%%%%%%%%%%%%%%%%%%%%%%%%%%%%%%%%%%%
\begin{frame}{Review}


The {\bf average logarithmic energy loss} aka {\bf slowing down
decrement} is

\begin{equation*}
  \xi = \overline{\ln(E/E')} 
    = 1 + \frac{\alpha}{1-\alpha}\ln\alpha 
  \tag{FNRP 2.54 and 2.56}
\end{equation*}

The number of elastic collisions \(n\) needed to reduce a neutron's
energy from \(E_0\) to \(E_n\) is approximately

\begin{equation*}
  n = \frac{1}{\xi} \ln(E_0/E_n) \, .
  \tag{FNRP 2.59}
\end{equation*}

\pause 
\vfill

{\bf Hmm.} How would one measure $n$ experimentally?
 
\end{frame}


%%%%%%%%%%%%%%%%%%%%%%%%%%%%%%%%%%%%%%%%%%%%%%%%%%%%%%%%%%%%%%%%%%%%%%%%%%%%%%
\begin{frame}[fragile]{How to Measure Neutron Production?}

The expected number of neutrons produced from fission per neutron absorbed at 
some location
can be expressed as

\pause 

\begin{equation*}
  \eta(E) = \frac{\bar{\nu}(E) \Sigma_f(E)}{\Sigma_{a}(E)} \, ,
\end{equation*}

where 
   
\begin{itemize}
 \item  $E$ is the energy of the incident neutron
 \item  $\bar{\nu}(E)$ is the average number of neutrons produced from fission due to incident neutrons of energy $E$ (and is typically between 2 and 3)
 \item  $\Sigma_f(E)$ is the macroscopic fission cross section for the fuel
 \item   $\Sigma_a(E)$ is the macroscopic absorption cross section for the fuel and includes capture, fission, and other absorption processes
\end{itemize}

\pause
   
Often the product $\bar{\nu}(E) \Sigma_f(E)$ is written $\nu \Sigma_f(E)$, as in 
\begin{equation*}
  \eta(E) = \frac{\nu\Sigma_f(E)}{\Sigma_a(E)} = \frac{\text{fission neutrons produced}}{\text{neutrons absorbed}} \, .
\tag{FNRP 3.3} 
\end{equation*}


\end{frame}
 
\begin{frame}[fragile]{OpenMC Examples on Beocat}

The computational demos shown so far are on Beocat.  You can 
copy the one for today by executing the following in the 
terminal after logging into Beocat:

{\tiny
\begin{verbatim}
  cp /homes/jaroberts/openmc/ne630_examples/NeutronReproductionFactor.ipynb .
\end{verbatim}

You can see all the examples present by executing
\begin{verbatim}
  ls /homes/jaroberts/openmc/ne630_examples/ 
\end{verbatim}
}

 
\end{frame}


\end{document}

