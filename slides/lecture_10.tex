\documentclass[aspectratio=1610]{beamer}
\usepackage[T1]{fontenc}
\usetheme{wildcat}
\usetikzlibrary{arrows.meta,angles,quotes,calc,intersections,positioning}

\usepackage{amsmath,amssymb,amsfonts}
\usepackage{booktabs}
\usepackage{relsize}
\usepackage{pgfplots}
\pgfplotsset{compat=1.16}
\usepackage{array}
\usepackage{siunitx}

\usepackage{jupyter}

\let\oldfootnotesize\footnotesize
\renewcommand*{\footnotesize}{\oldfootnotesize\tiny}

\def\mathdefault#1{#1}
\everymath=\expandafter{\the\everymath\displaystyle}


\title{Resonance Absorption \\
       {\small\it NE 630 - Lecture 10}}

\date{\input{term.txt} \\ {\footnotesize Git SHA: \input{git_sha.txt}}}

\author{Jeremy Roberts}


\definecolor{ksupurple}{HTML}{512888}
\definecolor{orange}{HTML}{CA7C1B}

\begin{document}

\begin{frame}
\titlepage
\end{frame}
 
 
%%%%%%%%%%%%%%%%%%%%%%%%%%%%%%%%%%%%%%%%%%%%%%%%%%%
\begin{frame}{Primary Objective}

Students will be able to 

\vfill

\begin{quote}
\textcolor{wcprimary}{approximate the flux spectrum $\phi(E)$ of epithermal neutrons subject to elastic collisions and resonance absorption in infinite, homogeneous systems.}
\end{quote}

\vfill 

\end{frame}

%%%%%%%%%%%%%%%%%%%%%%%%%%%%%%%%%%%%%%%%%%%%%%%%%%%%%%%%%%%%%%%%%%%%%%%%%%%%%%
\begin{frame}{Review}

Last time, we discussed the {\bf spectrum equation}:
\begin{align*}
\tag{like FNRP 3.15}
 \Sigma_t(E) \phi(E) &= \textcolor{wcprimary}{\int^{\infty}_0 p(E'\to E)\Sigma_s(E') \phi(E') dE'} \\ 
    &+ \textcolor{wcalerted}{ \chi(E) \int^{\infty}_0 \bar{\nu}(E')\Sigma_f(E') \phi(E') dE'}  
     + s'''_{\text{ext}}(E) \, .
\end{align*}

For $E > 0.1$ MeV, the spectrum can be approximated by
\begin{equation*}
 \phi(E) = \frac{\chi(E) s_f'''}{\Sigma_t(E)} \, ,
\end{equation*}
and for $1~\si{\electronvolt} < E < 0.1~\si{\mega\electronvolt}$, 
the spectrum {\it without absorption} 
follows
\begin{equation*}
  \phi(E) \propto \frac{1}{\Sigma_s(E) E} \, .
  \tag{FNRP 3.23}
\end{equation*}


\end{frame}

 

%%%%%%%%%%%%%%%%%%%%%%%%%%%%%%%%%%%%%%%%%%%%%%%%%%%%%%%%%%%%%%%%%%%%%%%%%%%%%%
\begin{frame}[fragile]{Slowing-Down in a Single Moderator}

For neutrons slowing down in a purely (elastic) scattering system with a single 
moderating nuclide, the spectrum equation becomes

\begin{equation*}
  \Sigma_s(E) \phi(E) = \int^{E/\alpha}_E \frac{1}{(1-\alpha)E'} \Sigma_s(E') \phi(E') dE' \, .
  \tag{FNRP 3.22}
\end{equation*}

Those integral bounds are {\bf important}: any neutron landing at energy $E$ after a collision 
must have had an initial energy $E'$ between $E$ (no energy loss) and $E/\alpha$ (maximum energy loss).
For ${}^{1}$H, that upper bound is, of course $\infty$.

\vfill 

{\bf Example}: If the system has two moderating nuclei (e.g., water), how would we extend FNRP 3.22?

\pause

{\small {\it Ans. The right-hand side would be split into a sum of integrals for each nuclide.}}

\end{frame}
 
 
%%%%%%%%%%%%%%%%%%%%%%%%%%%%%%%%%%%%%%%%%%%%%%%%%%%%%%%%%%%%%%%%%%%%%%%%%%%%%%
\begin{frame}[fragile]{Slowing Down with Absorption}
 
Consider a system with one moderator ($m$) nuclide and one fuel ($f$) nuclide.
The slowing-down equation is then 
 
\begin{align*}
  \tag{FNRP 3.28}
  \Sigma_t(E) \phi(E) 
   =& \textcolor{wcalerted}{\int^{E/\alpha^f}_E \frac{1}{(1-\alpha^f)E'} \Sigma^f_s(E') \phi(E') dE'} \\
   &+ \textcolor{wcprimary}{\int^{E/\alpha^m}_E \frac{1}{(1-\alpha^m)E'} \Sigma^m_s(E') \phi(E') dE'} \, . 
\end{align*}

In order to extract an approximate $\phi(E)$, we need to eliminate the integrals!

\end{frame}
 
%%%%%%%%%%%%%%%%%%%%%%%%%%%%%%%%%%%%%%%%%%%%%%%%%%%%%%%%%%%%%%%%%%%%%%%%%%%%%%
\begin{frame}[fragile]{How Wide is a Resonance?}
 
\centering
\includegraphics{figures/sde_NR.pdf}
 
\includegraphics{figures/sde_WR.pdf}

\end{frame}
 
%%%%%%%%%%%%%%%%%%%%%%%%%%%%%%%%%%%%%%%%%%%%%%%%%%%%%%%%%%%%%%%%%%%%%%%%%%%%%%
\begin{frame}[fragile]{The Narrow Resonance Approximation}

The energy loss due to scattering off moderating nuclei is always much larger 
than the resonance width.  Assuming {\it negligible absorption between resonances} and 
a {\it constant scattering cross section}, we can approximate
\begin{align*}
   \int^{E/\alpha^m}_E  & \frac{1}{(1-\alpha^m)E'} \Sigma^m_s(E') \phi(E') dE' \\
  &\approx \int^{E/\alpha^m}_E \frac{1}{(1-\alpha^m)E'} \Sigma^m_s  \textcolor{wcprimary}{\left ( \frac{1}{E'} \right )} dE' \\
  &= ? 
\end{align*}

\pause

If the energy loss due to scattering off resonant nuclei is also much larger than 
the resonance width, we can perform the same trick.  The resulting spectrum is

\begin{equation*}
\tag{like FNRP 3.29}
 \boxed{ \phi_{\text{NR}}(E) \propto \frac{1}{\Sigma_t(E) E} } \, .
\end{equation*}


\end{frame}
 

\end{document}

