\documentclass[aspectratio=1610]{beamer}
\usepackage[T1]{fontenc}
\usetheme{wildcat}
\usetikzlibrary{arrows.meta,angles,quotes,calc,intersections,positioning}

\usepackage{amsmath,amssymb,amsfonts}
\usepackage{booktabs}
\usepackage{relsize}
\usepackage{pgfplots}
\pgfplotsset{compat=1.16}
\usepackage{array}
\usepackage{siunitx}

\usepackage{jupyter}

\let\oldfootnotesize\footnotesize
\renewcommand*{\footnotesize}{\oldfootnotesize\tiny}

\def\mathdefault#1{#1}
\everymath=\expandafter{\the\everymath\displaystyle}


\title{Thermalization \\
       {\small\it NE 630 - Lecture 11}}

\date{\input{term.txt} \\ {\footnotesize Git SHA: \input{git_sha.txt}}}

\author{Jeremy Roberts}


\definecolor{ksupurple}{HTML}{512888}
\definecolor{orange}{HTML}{CA7C1B}

\begin{document}

\begin{frame}
\titlepage
\end{frame}
 
 
%%%%%%%%%%%%%%%%%%%%%%%%%%%%%%%%%%%%%%%%%%%%%%%%%%%
\begin{frame}{Primary Objective}

Students will be able to 

\vfill

\begin{quote}
\textcolor{wcprimary}{approximate the flux spectrum $\phi(E)$ of thermal neutrons in thermal-spectrum reactors.}
\end{quote}

\vfill 

\end{frame}

%%%%%%%%%%%%%%%%%%%%%%%%%%%%%%%%%%%%%%%%%%%%%%%%%%%%%%%%%%%%%%%%%%%%%%%%%%%%%%
\begin{frame}{Review}

Last time, we reasoned that if
\begin{itemize}
 \item $(1-\alpha^m)E \gg \Gamma$,
 \item $(1-\alpha^f)E \gg \Gamma$, and
 \item $\Sigma_s(E) \approx \Sigma_s$ (i.e., constant),
\end{itemize}
then the flux spectrum between about 1 eV and 0.1 MeV can be approximated by
\begin{equation*}
\tag{like FNRP 3.29}
 \boxed{ \phi_{\text{NR}}(E) \propto \frac{1}{\Sigma_t(E) E} } \, ,
\end{equation*}
which is the narrow-resonance (NR) approximation.  

\centering
\includegraphics{figures/sde_NR.pdf}

\end{frame}

%%%%%%%%%%%%%%%%%%%%%%%%%%%%%%%%%%%%%%%%%%%%%%%%%%%%%%%%%%%%%%%%%%%%%%%%%%%%%%
\begin{frame}{How Good is the NR Approximation?}

Consider the reflected sphere we use to model an infinite, homogeneous 
system in OpenMC.  For a 2 MeV source of neutrons, the flux spectrum between
1 eV and 0.1 MeV is shown below based on (a) direct simulation, i.e., ``truth'' and
(b) the NR approximation for a 10-to-1 mixture of ${}^1$H and ${}^{238}$U.


\includegraphics[width=0.95\linewidth]{figures/mc_vs_nr_u238_h1.pdf}


{\bf Hmm.} The curves are scaled to match at 0.1 MeV.  Why does the NR approximation
exceed the simulation value at 1 eV? \pause {\it Answer}: absorption attenuates the 
{\bf slowing down density} $q(E)$ according to FNRP 3.20.

\end{frame}



%%%%%%%%%%%%%%%%%%%%%%%%%%%%%%%%%%%%%%%%%%%%%%%%%%%%%%%%%%%%%%%%%%%%%%%%%%%%%%
\begin{frame}[fragile]{What Happens Below 1 eV?}

Recall the spectrum equation (sans external source):
\begin{align*}
\tag{like FNRP 3.15}
 \Sigma_t(E) \phi(E) = & \textcolor{wcprimary}{\int^{\infty}_0 p(E'\to E)\Sigma_s(E') \phi(E') dE'} \\ 
    &+ \textcolor{wcalerted}{ \chi(E) \int^{\infty}_0 \bar{\nu}(E')\Sigma_f(E') \phi(E') dE'} \, .
\end{align*}

Above 1 eV, we know that $p(E'\to E)$ is well approximated by a uniform, {\bf downscatter}-only distribution
based on a stationary target and isotropic scattering.
However, for neutrons below 1 eV in energy, the thermal motion of surrounding nuclei can lead to appreciable
neutron {\bf upscattering}, i.e., elastic collisions in which the outgoing neutron energy $E$ exceeds its
original energy $E'$.




\end{frame}
 
 
%%%%%%%%%%%%%%%%%%%%%%%%%%%%%%%%%%%%%%%%%%%%%%%%%%%%%%%%%%%%%%%%%%%%%%%%%%%%%%
\begin{frame}[fragile]{Detailed Balance}
 
Consider our infinite, homogeneous system, and suppose it is filled with 
a {\bf purely-scattering medium}.  If we dump a bucket of neutrons of {\it any} 
energies into the system, they have nothing to do but bounce around...forever.
Given a long enough time,  these neutrons will enter 
\only<1>{\textcolor{white}{{\bf thermal equilibrium}}}
\only<2>{{\bf thermal equilibrium}}
with the surrounding nuclei.

\vfill 
\pause 

Once in thermal equilibrium, the neutron population must satisfy, on 
the average, the condition of {\bf detailed balance}:
\begin{equation*}
\tag{FNRP 3.34}
 \Sigma_s(E\to E')\phi(E) = \Sigma_s(E'\to E)\phi(E') \, .
\end{equation*}
In other words, the rate at which neutrons of energy $E$ scatter to $E'$ must be
balanced by the rate at which neutrons of energy $E'$ scatter to $E$.  If this
were not true, the system would not be steady, since neutrons would then 
be moving up or down in energy on the average!
 
\end{frame}
 
%%%%%%%%%%%%%%%%%%%%%%%%%%%%%%%%%%%%%%%%%%%%%%%%%%%%%%%%%%%%%%%%%%%%%%%%%%%%%%
\begin{frame}[fragile]{Maxwell-Boltzmann Distribution}

It turns out that, no matter the specific form of $\Sigma_s(E'\to E)$, 
the flux spectrum that satisfied detailed balance is 
\begin{equation*}
\tag{FNRP 3.35}
 \phi_{\text{MB}}(E) = \frac{E}{(kT)^2} e^{-\frac{E}{kT}} \, ,
\end{equation*}
the famed Maxwell-Boltzmann (or, sometimes, Maxwellian) distribution, where 
$T$ is the temperature of the medium and $k \approx 8.617\cdot 10^{-5}~\si{\electronvolt\per\kelvin}$ is Boltzmann's constant.
Because this is a $\phi_{\text{MB}}(E)$ is a probability density function, its 
integral over all energies must be unity.

\end{frame}

%%%%%%%%%%%%%%%%%%%%%%%%%%%%%%%%%%%%%%%%%%%%%%%%%%%%%%%%%%%%%%%%%%%%%%%%%%%%%%
\begin{frame}[fragile]{What About $\Sigma_s(E'\to E)$?}

Accounting for target motion automatically makes defining $\Sigma_s(E'\to E) = p(E'\to E)\Sigma_s(E')$ more challenging!
In the simplest approximation, target nuclei are assumed to form a {\bf free gas}, and it follows (after much pain 
and suffering) that
\begin{align*}
\tag{D\&H 9-38}
  p(E'\to E) &= \frac{\theta^2}{2E'} e^{-\epsilon-\epsilon'} \times \\
  & \Bigg \{ 
    \Bigg [
      \text{erf}(\theta\sqrt{\epsilon'}-\rho\sqrt{\epsilon})\pm \text{erf}(\theta\sqrt{\epsilon'}+\rho\sqrt{\epsilon})
    \Bigg ] \\
    &+  \Bigg [
      \text{erf}(\theta\sqrt{\epsilon}-\rho\sqrt{\epsilon'})\mp \text{erf}(\theta\sqrt{\epsilon}+\rho\sqrt{\epsilon'})
    \Bigg ] 
    \Bigg \} \, , \quad E' \lessgtr E \, ,
\end{align*}
where $\theta = (A+1)/(2\sqrt{A})$, $\rho = (A-1)/(2\sqrt{A})$, and $\epsilon = E/kT$.
 
\end{frame}
 
 
%%%%%%%%%%%%%%%%%%%%%%%%%%%%%%%%%%%%%%%%%%%%%%%%%%%%%%%%%%%%%%%%%%%%%%%%%%%%%%
\begin{frame}[fragile]{Special Case: $\Sigma_s(E'\to E)$ for ${}^1$H}

For ${}^{1}$H, the analysis simplifies somewhat to produce 

\begin{align*}
\tag{D\&H 9-23}
p(E'\to E) = \frac{1}{E'} 
  \begin{cases}
       \text{erf} \sqrt{\frac{E}{kT}} \, ,  & E' > E \\
       e^{\displaystyle\left(\frac{E'-E}{kT}\right)} \text{erf} \sqrt{\frac{E'}{kT}} \, , &  E' < E \, .
  \end{cases}
\end{align*}
 
\centering
\includegraphics[width=0.9\linewidth]{figures/thermal_scattering_hydrogen.pdf}

 
\end{frame}
 
%%%%%%%%%%%%%%%%%%%%%%%%%%%%%%%%%%%%%%%%%%%%%%%%%%%%%%%%%%%%%%%%%%%%%%%%%%%%%%
\begin{frame}[fragile]{Beyond the Free-Gas Approximation}
 
More generally, thermal-neutron scattering cross sections are represented
using\footnote{See \url{https://t2.lanl.gov/nis/njoy/ther02.html} 
for more information.}

\begin{equation*}
 p(E\to E', \mu) = \frac{1}{2kT} \sqrt{\frac{E'}{E}} e^{-\beta/2} S(\alpha, \beta) \, ,
\end{equation*}
where 
\begin{equation*}
 \alpha = \frac{E' + E -2\mu\sqrt{EE'}}{A kT} \quad \text{and} \quad \beta = \frac{E'-E}{kT} \, 
\end{equation*}
represent a reduced momentum and reduced energy, respectively.

\vfill 

The ``scattering law'' $S(\alpha, \beta)$ is measured experimental and/or estimated 
by molecular dynamics simulations.  Such laws are defined for many important 
moderators and some fuel compositions (e.g., UO$_2$).  
 
\end{frame}

\end{document}

