\documentclass[aspectratio=1610]{beamer}
\usepackage[T1]{fontenc}
\usetheme{wildcat}
\usetikzlibrary{arrows.meta,angles,quotes,calc,intersections,positioning}

\usepackage{amsmath,amssymb,amsfonts}
\usepackage{booktabs}
\usepackage{relsize}
\usepackage{pgfplots}
\pgfplotsset{compat=1.16}
\usepackage{array}
\usepackage{siunitx}
\usepackage{cancel}
\usepackage{jupyter}

\let\oldfootnotesize\footnotesize
\renewcommand*{\footnotesize}{\oldfootnotesize\tiny}

\def\mathdefault#1{#1}
\everymath=\expandafter{\the\everymath\displaystyle}


\title{The Multigroup Method:  \\
       The $k$-Eigenvalue Problem \\
       {\small\it NE 630 - Lecture 14}}

\date{\input{term.txt} \\ {\footnotesize Git SHA: \input{git_sha.txt}}}

\author{Jeremy Roberts}


\definecolor{ksupurple}{HTML}{512888}
\definecolor{orange}{HTML}{CA7C1B}

\begin{document}

\begin{frame}
\titlepage
\end{frame}
 
 
%%%%%%%%%%%%%%%%%%%%%%%%%%%%%%%%%%%%%%%%%%%%%%%%%%%
\begin{frame}{Primary Objective}

Students will be able to 

\vfill

\begin{quote}
\textcolor{wcprimary}{estimate the multiplication factor and multigroup spectrum for 
an infinite, homogeneous system.
}
\end{quote}

\vfill 

\end{frame}

%%%%%%%%%%%%%%%%%%%%%%%%%%%%%%%%%%%%%%%%%%%%%%%%%%%%%%%%%%%%%%%%%%%%%%%%%%%%%%
\begin{frame}{Review: The Multigroup Equations}


\begin{equation*}
\boxed{
 \Sigma_{t,g} \phi_g = 
   \textcolor{wcprimary}{ \sum_{g'=1}^G \Sigma_{s,g\gets g'} \phi_{g'}} + 
   \textcolor{wcalerted}{\chi_g \sum_{g'=1}^G \nu\Sigma_{f,g'}\phi_{g'}} + s_{g} \, , \quad g = 1, 2, \ldots, G 
} \, 
\end{equation*}

{\small 
\begin{equation*}
\phi_g = \int_{E_g}^{E_{g-1}} \phi(E)\, dE 
\qquad 
\Sigma_{x,g} =
\frac{ \displaystyle \int_{E_g}^{E_{g-1}} \Sigma_x(E)\,\phi(E)\, dE }
     { \displaystyle \int_{E_g}^{E_{g-1}} \phi(E)\, dE }
\qquad 
\chi_g = \int_{E_g}^{E_{g-1}} \chi(E) dE 
\end{equation*}

% Group-to-group scattering (incident g' to outgoing g)
\begin{equation*}
\Sigma_{s,\,g \gets g'} =
\frac{ \displaystyle \int_{E_g}^{E_{g-1}} dE \int_{E_{g'}}^{E_{g'-1}} dE'\,
       \Sigma_s(E' \to E)\,\phi(E') }
     { \displaystyle \int_{E_{g'}}^{E_{g'-1}} \phi(E')\, dE' } 
\qquad s_g = \int_{E_g}^{E_{g-1}} s'''_{\text{ext}}(E)\, dE \, .
\end{equation*}
}

\end{frame}



%%%%%%%%%%%%%%%%%%%%%%%%%%%%%%%%%%%%%%%%%%%%%%%%%%%%%%%%%%%%%%%%%%%%%%%%%%%%%%
\begin{frame}[fragile]{Review: Group-to-Group Scattering}

For energies above 1 eV, we can assume no upscattering so that
$\sigma_s(E'\to E) = \sigma_s(E')p(E'\to E)$ where 
\begin{equation*}
 p(E'\to E) = 
 \begin{cases}
   \frac{1}{(1-\alpha)E'} & \alpha E' < E < E' \\
   0 & \text{otherwise} \, .
 \end{cases}
\end{equation*}

Then
\begin{equation*}
 \sigma_{s,\,g \gets g'} =
\frac{ \displaystyle \int_{E_{g'}}^{E_{g'-1}} dE' \frac{\sigma_s(E')\phi(E')}{(1-\alpha)E'}            \int_{ \textcolor{wcprimary}{\max{(E_g, \alpha E')}}}^{\textcolor{wcalerted}{\min{(E_{g-1}, E')}}} dE\,
        }
     { \displaystyle \int_{E_{g'}}^{E_{g'-1}} \phi(E')\, dE' } \, .
\end{equation*}
By convention, if the inner integral (over $E$) has a lower bound $\textcolor{wcprimary}{\max{(E_g, \alpha E')}}$ greater than its upper bound $\textcolor{wcalerted}{\min{(E_{g-1}, E')}}$, the result is zero.

\end{frame}
  
%%%%%%%%%%%%%%%%%%%%%%%%%%%%%%%%%%%%%%%%%%%%%%%%%%%%%%%%%%%%%%%%%%%%%%%%%%%%%%
\begin{frame}[fragile]{Example: Three Groups}

Consider again Prof.~McNeil's tank of water with ${}^{252}$Cf emitting 10 n's \si{\per\cubic\centi\meter\per\second}. 
Write down the set of algebraic equations we can use to find the multigroup fluxes for $E_0 = 10$ MeV, $E_1 = 0.1$ MeV, 
$E_2 = 1$ eV, and $E_3 = 10^{-3}$ eV.

\pause 
\vfill 

{\footnotesize
{\it Answer}:  
\begin{align*}
   \Sigma_{t,1}\phi_1 &= \Sigma_{s,1\gets 1} \phi_1 + s_1 \\
   \Sigma_{t,2}\phi_2 &= \Sigma_{s,2\gets 1} \phi_1 + \Sigma_{s,2\gets 1} \phi_2  \\
   \Sigma_{t,3}\phi_3 &= \Sigma_{s,3\gets 1} \phi_1 + \Sigma_{s,3\gets 2} \phi_2 + \Sigma_{s,3\gets 3} \phi_3  \\
\end{align*}

or

\begin{align*}
   \Sigma_{r,1}\phi_1 &=  s_1 \\
   \Sigma_{r,2}\phi_2 &= \Sigma_{s,2\gets 1} \phi_1  \\
   \Sigma_{r,3}\phi_3 &= \Sigma_{s,3\gets 1} \phi_1 + \Sigma_{s,3\gets 2} \phi_2   \\
\end{align*}
where $\Sigma_{r, g} = \Sigma_{t, g} - \Sigma_{s, g\gets g}$ is the removal cross section.
}


\end{frame}


%%%%%%%%%%%%%%%%%%%%%%%%%%%%%%%%%%%%%%%%%%%%%%%%%%%%%%%%%%%%%%%%%%%%%%%%%%%%%%
\begin{frame}[fragile]{Example: Group-to-Group Scattering}

Assuming energy bounds of $E_0 = 10$ MeV, $E_1 = 0.1$ MeV, 
$E_2 = 1$ eV, and $E_3 = 10^{-3}$ eV, isotropic scattering in the CM system, and 
a constant-in-energy scattering 
cross section $\sigma_s(E) = 20$ b, compute 
$\sigma_{s, g\gets 2}$ for ${}^1$H for $g=1, 2, 3$.

\pause 
\vfill 

{\footnotesize
{\it Answer}:  For a constant $\sigma_s(E) = \sigma_s$, $\phi(E) = 1/E$, and $\alpha = 0$, we have

\begin{equation*}
 \sigma_{s,\,g \gets g'} =
\frac{ \displaystyle \sigma_s \int_{E_{g'}}^{E_{g'-1}} dE' \frac{1}{(E')^2} 
      [\textcolor{wcalerted}{ \min(E_{g-1}, E') } - 
       \textcolor{wcprimary}{ \cancelto{E_g}{\max{(E_g, \alpha E')}} } 
       ] }    
     { \displaystyle \int_{E_{g'}}^{E_{g'-1}} \frac{1}{E'} \, dE' } \, .
\end{equation*}
Here, $\textcolor{wcalerted}{ \min(E_{g-1}, E') } = E'$ if $g=g'$; otherwise, $g < g'$ (for downscatter), and 
$\textcolor{wcalerted}{ \min(E_{g-1}, E') } = E_{g-1}$.  It then follows that
\begin{align*}
 \sigma_{s,1\gets 2} &= 0 \, (\text{no upscatter}) \\
 \sigma_{s,2\gets 2} &= \sigma_0 \frac{\ln(E_1/E_2)+(E_2/E_1 - 1)}{\ln(E_1/E_2)} \approx 0.913\sigma_0 \\
 \sigma_{s,3\gets 2} &= \sigma_0 \frac{(E_2-E_3)(1/E_2 - 1/E_1)}{\ln(E_1/E_2)} \approx 0.087\sigma_0 \, .
\end{align*}
Note that $\sum_{g=1}^{3} \sigma_s{s, g\gets 2} = \sigma_0$.
}


\end{frame}

%%%%%%%%%%%%%%%%%%%%%%%%%%%%%%%%%%%%%%%%%%%%%%%%%%%%%%%%%%%%%%%%%%%%%%%%%%%%%%
\begin{frame}[fragile]{Example: A Source-Driven, Multiplying, Two-Group Problem}

Often, the two-group formulation with $E_0 = 10$ MeV, $E_1 = 1$ eV, and $E_2 = 10^{-3}$
is sufficient (as long as a proper $\phi(E)$ was used!).  For a source of $s_1$ neutrons \si{\per\cubic\centi\meter\per\second},
use the data tabulated below to determine the two-group fluxes $\phi_1$ and $\phi_2$ (i.e., the non-thermal and thermal fluxes).

\vfill 
\begin{tabular}{lll}
\toprule
                        & $g = 1$                    & $g=2$ \\
\midrule
$\Sigma_{\gamma, g}$     & $2.04 \times 10^{-3}$    & $1.91 \times 10^{-2}$    \\
$\Sigma_{f, g}$          & $2.96 \times 10^{-4}$    & $1.38 \times 10^{-2}$    \\
$\bar{\nu}_g$   & $2.49$  &  $2.43$ \\
$\Sigma_{s,1\gets g}$    & $7.21 \times 10^{-1}$  & $0$  \\
$\Sigma_{s,2\gets g}$    & $4.27 \times 10^{-2}$  & $2.72 \times 10^{0}$ \\
$v_g$ (m/s)                     & $1\times 10^{5}$    & $2\times 10^{3}$ \\
\bottomrule
\end{tabular}


\pause 
\vfill 

{\small 
{\it Answer:} $\phi_1 = 1288.5$ \si{\per\centi\meter\squared\per\second}
and $\phi_2 = 1672.3$ \si{\per\centi\meter\squared\per\second}.  See full worked example on Canvas.
}

\end{frame}
 
\end{document}

