\documentclass[aspectratio=1610]{beamer}
\usepackage[T1]{fontenc}
\usetheme{wildcat}
\usetikzlibrary{arrows.meta,angles,quotes,calc,intersections,positioning}

\usepackage{amsmath,amssymb,amsfonts}
\usepackage{booktabs}
\usepackage{relsize}
\usepackage{pgfplots}
\pgfplotsset{compat=1.16}
\usepackage{array}
\usepackage{siunitx}
\usepackage{cancel}
\usepackage{jupyter}

\let\oldfootnotesize\footnotesize
\renewcommand*{\footnotesize}{\oldfootnotesize\tiny}

\def\mathdefault#1{#1}
\everymath=\expandafter{\the\everymath\displaystyle}


\title{Lumps, Piles, and Lattices \\
       {\small\it NE 630 - Lecture 15}}

\date{\input{term.txt} \\ {\footnotesize Git SHA: \input{git_sha.txt}}}

\author{Jeremy Roberts}


\definecolor{ksupurple}{HTML}{512888}
\definecolor{orange}{HTML}{CA7C1B}

\begin{document}

\begin{frame}
\titlepage
\end{frame}
 
 
%%%%%%%%%%%%%%%%%%%%%%%%%%%%%%%%%%%%%%%%%%%%%%%%%%%
\begin{frame}{Primary Objective}

Students will be able to 

\vfill

\begin{quote}
\textcolor{wcprimary}{estimate the largest multiplication factor possible for a 
homogeneous mixture of water and natural uranium and propose ways to increase 
that value+.  
}
\end{quote}

\vfill 

\end{frame}

%%%%%%%%%%%%%%%%%%%%%%%%%%%%%%%%%%%%%%%%%%%%%%%%%%%%%%%%%%%%%%%%%%%%%%%%%%%%%%
\begin{frame}{Review: The Two-Group Equations}

Let $E_0 = 10$ MeV, $E_1 = 1$ eV, and $E_2 = 10^{-3}$ eV.  Then, the 
two-group equations for a \textcolor{wcprimary}{source-driven} system (including multiplication) are
\begin{align*}
\Sigma_{r,1} \phi_1 &= \nu\Sigma_{f,1}\phi_1 + \nu\Sigma_{f,2}\phi_2 + \textcolor{wcprimary}{s_1} \\
 \Sigma_{a,2} \phi_2 &= \Sigma_{s,2\gets 1}\phi_1 + \textcolor{wcprimary}{s_2} \, ,
\end{align*}
where $\Sigma_{r,1} = \Sigma_{t,1}-\Sigma_{s,1\gets 1}$, $\chi_1 = 1$, $\chi_2 = 0$, and
$\Sigma_{s,1\gets 2} = 0$.  Without sources, we introduce the \textcolor{wcalerted}{multiplication
factor $k_{\infty}$} for balance, or
\begin{align*}
 \Sigma_{r,1} \phi_1 &= \frac{\nu\Sigma_{f,1}\phi_1 + \nu\Sigma_{f,2}\phi_2}{\textcolor{wcalerted}{k_{\infty}}}  \\
 \Sigma_{a,2} \phi_2 &= \Sigma_{s,2\gets 1}\phi_1  \, .
\end{align*}

\end{frame}



%%%%%%%%%%%%%%%%%%%%%%%%%%%%%%%%%%%%%%%%%%%%%%%%%%%%%%%%%%%%%%%%%%%%%%%%%%%%%%
\begin{frame}[fragile]{Review: Two Groups in Matrix Form}
 
We can write these equations using matrix notation as
\begin{equation*}
\begin{split}
\overbrace{
 \begin{pmatrix}
  \Sigma_{t,1}-\Sigma_{s,1\gets 1} & \Sigma_{s,1\gets 2}              \\
  \Sigma_{s,2\gets 1}              & \Sigma_{t,2}-\Sigma_{s,2\gets 2}
 \end{pmatrix}
}^{\mathbf{A}}
 &
 \begin{pmatrix}
   \phi_1        \\
   \phi_2
 \end{pmatrix} = \\
 \frac{1}{ \textcolor{wcalerted}{k_{\infty}} } 
\underbrace{
 \begin{pmatrix}
   \chi_1        \\
   \chi_2 
 \end{pmatrix}
 \begin{pmatrix}
    \nu\Sigma_{f,1}  & \nu\Sigma_{f,2}
 \end{pmatrix}
 }_{\mathbf{F}}
 &
 \begin{pmatrix}
   \phi_1        \\
   \phi_2
 \end{pmatrix}
 +
 \textcolor{wcprimary}{
 \begin{pmatrix}
   s_1        \\
   s_2
 \end{pmatrix}
 } \, ,
 \end{split}
\end{equation*}
or, for source-driven systems,
\begin{equation*}
  (\mathbf{A}-\mathbf{F})\boldsymbol{\phi} = \mathbf{s} \, ,
\end{equation*}
and for source-free systems,
\begin{equation*}
   \mathbf{A}^{-1}\mathbf{F} \boldsymbol{\phi} = k_{\infty} \boldsymbol{\phi} \, .
\end{equation*}

{\bf Quick Check}: Write out $\mathbf{F}$ explicitly!
\end{frame}
  
%%%%%%%%%%%%%%%%%%%%%%%%%%%%%%%%%%%%%%%%%%%%%%%%%%%%%%%%%%%%%%%%%%%%%%%%%%%%%%
\begin{frame}[fragile]{Example: Many Groups!}


\end{frame}


%%%%%%%%%%%%%%%%%%%%%%%%%%%%%%%%%%%%%%%%%%%%%%%%%%%%%%%%%%%%%%%%%%%%%%%%%%%%%%
\begin{frame}[fragile]{A Trip Down Memory Lane}

\begin{enumerate}
  \item 1932: James Chadwick discovers the neutron, establishing the missing building block of the nucleus.
  \item 1934: Enrico Fermi and collaborators bombard elements with neutrons, discovering many new radioisotopes and the principle of neutron-induced reactions.
  \item 1938: Otto Hahn and Fritz Strassmann observe barium among the products of neutron bombardment of uranium, providing the first experimental evidence of nuclear fission.
  \item 1939: Lise Meitner and Otto Frisch correctly interpret Hahn and Strassmann’s results as fission, explaining the process in terms of nuclear binding energy and mass defect.
  \item 1939: Niels Bohr and John Wheeler publish a theoretical framework for fission, identifying fissile materials and energy release mechanisms.
  \item \textcolor{wcprimary}{1939–1940: Experimental confirmation of secondary neutrons emitted in fission establishes the possibility of a self-sustaining chain reaction.}
  \item December 2, 1942: Enrico Fermi directs the first controlled, self-sustaining chain reaction in Chicago Pile-1 (CP-1).
\end{enumerate}

\end{frame}

\begin{frame}{Lumps, Piles, and Lattices}
 
(Discussion of Fermi's Patent)
 
\end{frame}


%%%%%%%%%%%%%%%%%%%%%%%%%%%%%%%%%%%%%%%%%%%%%%%%%%%%%%%%%%%%%%%%%%%%%%%%%%%%%%
\begin{frame}[fragile]{Exam 1}

\begin{enumerate}
 \item Reece will have normal session tomorrow.
 \item Practice exam and solution are posted.
 \item Check solutions even if you earned full credit.
 \item 8.5'' x 11'' sheet, front and back, and a calculator
\end{enumerate}

\vfill 
\pause

{\Large Questions?}

\end{frame}
 
\end{document}

