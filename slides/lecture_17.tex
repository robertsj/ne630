\documentclass[aspectratio=1610]{beamer}
\usepackage[T1]{fontenc}
\usetheme{wildcat}
\usetikzlibrary{arrows.meta,angles,quotes,calc,intersections,positioning}

\usepackage{amsmath,amssymb,amsfonts}
\usepackage{booktabs}
\usepackage{relsize}
\usepackage{pgfplots}
\pgfplotsset{compat=1.16}
\usepackage{array}
\usepackage{siunitx}
\usepackage{cancel}
\usepackage{jupyter}

\let\oldfootnotesize\footnotesize
\renewcommand*{\footnotesize}{\oldfootnotesize\tiny}

\def\mathdefault#1{#1}
\everymath=\expandafter{\the\everymath\displaystyle}


\title{Unit Cells and Thermal Reactors  \\
       {\small\it NE 630 - Lecture 17}}

\date{\input{term.txt} \\ {\footnotesize Git SHA: \input{git_sha.txt}}}

\author{Jeremy Roberts}


\definecolor{ksupurple}{HTML}{512888}
\definecolor{orange}{HTML}{CA7C1B}

\begin{document}

\begin{frame}
\titlepage
\end{frame}
 
 
%%%%%%%%%%%%%%%%%%%%%%%%%%%%%%%%%%%%%%%%%%%%%%%%%%%
\begin{frame}{Primary Objective}

Students will be able to 

\vfill

\begin{quote}
\textcolor{wcprimary}{use the four-factor formula to 
characterize (quantitatively and qualitatively) the multiplication 
factor of a thermal-spectrum reactor based on unit-cell quantities. 
}
\end{quote}

\vfill 

\end{frame}

%%%%%%%%%%%%%%%%%%%%%%%%%%%%%%%%%%%%%%%%%%%%%%%%%%%%%%%%%%%%%%%%%%%%%%%%%%%%%%
\begin{frame}{Review: $k_{\infty}$}

Generally, the neutron flux depends on energy {\it and} space, i.e., 

$$
  \phi = \phi(\mathbf{r}, E) \, .
$$

Thus, total reaction-rate densities require integration over both space and energy, e.g.,

$$
  k_{\infty} = \frac{\iiint\limits_{V} \int\limits^{\infty}_0 \nu(E) \Sigma_f(E) \phi(\mathbf{r}, E) d^3 r}
                    { \iiint\limits_{V} \int\limits^{\infty}_0 \Sigma_a(E) \phi(\mathbf{r}, E) d^3 r } \, .
$$


\end{frame}



%%%%%%%%%%%%%%%%%%%%%%%%%%%%%%%%%%%%%%%%%%%%%%%%%%%%%%%%%%%%%%%%%%%%%%%%%%%%%%
\begin{frame}{Review: $k_{\infty}$ and Fast Spectrum Reactors}


For a fast reactor composed of fuel ($f$), coolant ($c$), and structural material ($st$),
the multiplication factor can be approximated by

\begin{equation*}
\tag{FNRP 4.8}
k_{\infty} = \frac{ V_f \nu\bar{\Sigma}_f^f }
                  { V_f \bar{\Sigma}_a^f + V_c \bar{\Sigma}_a^{c} + V_{st} \bar{\Sigma}_a^{st}} \, .
\end{equation*}

What does this approximation assume about $\phi(\mathbf{r}, E)$?  Is that valid?
 

\end{frame}
  
%%%%%%%%%%%%%%%%%%%%%%%%%%%%%%%%%%%%%%%%%%%%%%%%%%%%%%%%%%%%%%%%%%%%%%%%%%%%%%
\begin{frame}[fragile]{$k_{\infty}$ for Thermal Reactors}


For thermal-spectrum reactors, we can no longer factor out an energy-dependent flux spectrum, and, from Eq. (4.18), we have
 
\begin{equation*}
  k_{\infty} = \frac{V_f \int^{\infty}_0 \nu\Sigma^f_f(E) \phi_f(E) dE}
   {V_f \int^{\infty}_0 \Sigma_a^f \phi_f(E) dE 
     + V_m \int^{\infty}_0 \Sigma_a^m(E) \phi_m(E) dE} \, .
  \tag{FNRP 4.18} 
\end{equation*}

Note that this expression {\it does} assume that $\phi(\mathbf{r}, E)$ is spatially uniform within a region, e.g.,

\begin{equation*}
  \phi_f(\mathbf{r}, E) = \phi_f(E), \qquad \mathbf{r} \in V_f \, .
\end{equation*}

If there is spatial variation, then we can still use Eq. (4.18) if we use a volume averaged flux, e.g.,

\begin{equation*}
  \phi_f(E) = \phi_f(\mathbf{r}, E) = \frac{\iiint\limits_{V_f} \phi_f(\mathbf{r}, E) d^3 r}{V_f} \, .
\end{equation*}
\end{frame}



%%%%%%%%%%%%%%%%%%%%%%%%%%%%%%%%%%%%%%%%%%%%%%%%%%%%%%%%%%%%%%%%%%%%%%%%%%%%%%
\begin{frame}[fragile]{Reproduction Factor}

\begin{equation*}
  \eta_T = \frac{\nu\bar{\Sigma}_{fT}^f} {\bar{\Sigma}^f_{aT}}
  \tag{FNRP 4.49}
\end{equation*}

\vfill 

In plain English, $\eta_T$ is...?

\end{frame}


%%%%%%%%%%%%%%%%%%%%%%%%%%%%%%%%%%%%%%%%%%%%%%%%%%%%%%%%%%%%%%%%%%%%%%%%%%%%%%
\begin{frame}[fragile]{Thermal Utilization Factor}

\begin{equation*}
 f = \frac{1}{1 + \zeta \left(V_m \bar{\Sigma}^m_{aT} \big/ V_f \bar{\Sigma}^f_{aT} \right ) } \, ,
 \tag{FNRP 4.48}
\end{equation*}

where 

\begin{equation*}
  \zeta = \frac{\bar{\phi}_{mT}}{\bar{\phi}_{fT}} \, ,
 \tag{FNRP 4.50}
\end{equation*}
is the {\it thermal disadvantage factor}.


\vfill 

In plain English, $f$ is...?

\end{frame}


%%%%%%%%%%%%%%%%%%%%%%%%%%%%%%%%%%%%%%%%%%%%%%%%%%%%%%%%%%%%%%%%%%%%%%%%%%%%%%
\begin{frame}[fragile]{Resonance Escape Probability}

{\small
\begin{equation*}
 p = 1 - \frac{V_f \int_I \Sigma_a^f(E) \phi_f(E) dE} 
              {V_f \left [ \int_T \Sigma_a^f(E) \phi_f(E) dE + \int_T \Sigma_a^f(E) \phi_f(E) dE \right ] 
               + V_m \int_T \Sigma_a^m(E) \phi_m(E) dE } \, .
 \tag{FNRP 4.28} 
\end{equation*}
}

\vfill 

In plain English, $p$ is...?

\end{frame}

%%%%%%%%%%%%%%%%%%%%%%%%%%%%%%%%%%%%%%%%%%%%%%%%%%%%%%%%%%%%%%%%%%%%%%%%%%%%%%
\begin{frame}[fragile]{Fast Fission Factor}

\begin{equation*}
 \varepsilon = \frac{\int_T \nu\Sigma_f^f(E) \phi_f(E) dE + \int_F \nu\Sigma_f^f(E) \phi_f(E) dE}
 {\int_T \nu\Sigma_f^f(E) \phi_f(E) dE} \, .
 \tag{FNRP 4.25}
\end{equation*}

\vfill 

In plain English, $\varepsilon$ is...?

\end{frame}


%%%%%%%%%%%%%%%%%%%%%%%%%%%%%%%%%%%%%%%%%%%%%%%%%%%%%%%%%%%%%%%%%%%%%%%%%%%%%%
\begin{frame}[fragile]{Using Group Fluxes}

In the expressions for $p$ and $\varepsilon$, integrals over energies are 
taken.  With use of effective cross sections, one can write, e.g.,

\begin{equation*}
    V_f \int_T \Sigma^f_a(E) \phi_f(E) dE = V_f \bar{\Sigma}_{aT} \phi_{fT} .
\end{equation*}

{\bf Exercise}: Simplify  Eqs. (4.25) and (4.28) in terms of the effective cross sections and group-wise fluxes.  Here they are again:

{\small
\begin{equation*}
 p = 1 - \frac{V_f \int_I \Sigma_a^f(E) \phi_f(E) dE} 
              {V_f \left [ \int_T \Sigma_a^f(E) \phi_f(E) dE + \int_T \Sigma_a^f(E) \phi_f(E) dE \right ] 
               + V_m \int_T \Sigma_a^m(E) \phi_m(E) dE } \, .
 \tag{FNRP 4.28} 
\end{equation*}

\begin{equation*}
 \varepsilon = \frac{\int_T \nu\Sigma_f^f(E) \phi_f(E) dE + \int_F \nu\Sigma_f^f(E) \phi_f(E) dE}
 {\int_T \nu\Sigma_f^f(E) \phi_f(E) dE} \, .
 \tag{FNRP 4.25}
\end{equation*}
}

\end{frame}


%%%%%%%%%%%%%%%%%%%%%%%%%%%%%%%%%%%%%%%%%%%%%%%%%%%%%%%%%%%%%%%%%%%%%%%%%%%%%%
\begin{frame}[fragile]{Using Two-Group Fluxes}

If we treat all energies above 1 eV as fast (F) and combine all non-fuel (nf)
quantities, the four factors can be written as


\begin{align*}
\tag{FNRP 4.49*}
\eta_T &= \frac{\nu\bar{\Sigma}_{fT}^f} {\bar{\Sigma}^f_{aT}} \\
\tag{FNRP 4.48*}
f &= \frac{1}{1 + \zeta \left(V_{nf} \bar{\Sigma}^{nf}_{aT} \big/ V_f \bar{\Sigma}^f_{aT} \right ) } \\
\tag{FNRP 4.28*}
p &= 1 - \frac{V_f \bar{\Sigma}^f_{aF} \phi^{f}_{F}}
      {V_f \left[ \bar{\Sigma}^f_{aT} \phi^{f}_{T} + \bar{\Sigma}^f_{aF} \phi^{f}_{F} \right ]+ V_{nf} \bar{\Sigma}^{nf}_{aT} \phi^{nf}_{T}} \\
\tag{FNRP 4.25*}
\varepsilon &= \frac{\nu\bar{\Sigma}^f_{fT} \phi^{f}_{T} + \nu\bar{\Sigma}^f_{fF} \phi^{f}_{F} }
   {\nu\bar{\Sigma}^f_{fF} \phi^{f}_{F} } 
\end{align*}


\end{frame}
 
%%%%%%%%%%%%%%%%%%%%%%%%%%%%%%%%%%%%%%%%%%%%%%%%%%%%%%%%%%%%%%%%%%%%%%%%%%%%%%
\begin{frame}[fragile]{Spatial Homogenization}
 
We can preserve reaction rates over multiple volumes just like 
we did over energy ranges.  This process is called {\it spatial homogenization}.
For some reaction $x$ and volumes $A$ and $B$, we seek $\bar{\Sigma}_x$ and $\bar{\phi}$
that satisfies

\begin{equation*}
  V \bar{\Sigma}_x \bar{\phi}= (V^A + V^{B}) \bar{\Sigma}_x \bar{\phi} = V^A  \bar{\Sigma}^A_x  \bar{\phi}^A  + V^{B}  \bar{\Sigma}^{B}_x  \bar{\phi}^{B}  \
\end{equation*}

\vfill 

To preserve the number of neutrons in the combined volume, we need
\begin{equation*}
  (V^A + V^{B})   \bar{\phi} =  V^A \bar{\phi}^A  + V^{B} \bar{\phi}^{B} \longrightarrow  \boxed{ \bar{\phi} = \frac{V^A}{V} \bar{\phi}^A + \frac{V^B}{V} \bar{\phi}^B } \, .
\end{equation*}
It then follows that
\begin{equation*}
 \boxed{\bar{\Sigma}_x = \frac{V^A  \bar{\Sigma}^A_x  \bar{\phi}^A  + V^{B}  \bar{\Sigma}^{B}_x  \bar{\phi}^{B}}
                       { V^A \bar{\phi}^A + V^B \bar{\phi}^B } } \, .
\end{equation*}


% 
% {\bf Example}:  Consider a unit cell in which the fuel is surrounded by a thin shell of metallic cladding (cl).
% We wish to combine the moderator and cladding fluxes, cross sections and volumes into single non-fuel terms
% so that reacion rates
% are preserved.
% Given $V^r$, $\bar{\Sigma}^{r}_{aT}$, $\phi^{r}_{T}$ for $r \in \{m, cl\}$, compute $V^{nf}$, $\bar{\Sigma}^{nf}_{aT}$, $\phi^{nf}_{T}$.
% 
% 
% 
% \vfill 
% \pause
% 
% {\bf Solution}:  To preserve the overall absorption rate, we require 
% \begin{equation*}
%  V^{nf} \phi^{nf} \bar{\Sigma}^{nf}_{aT} = V^{m} \phi^{m} \bar{\Sigma}^{m}_{aT} + V^{cl} \phi^{cl} \bar{\Sigma}^{cl}_{aT} \, .
% \end{equation*}
% To preserve volume, we require $V^{nf} = V^m + V^{cl}$.
% Finally, to preserve neutrons, recall that $\phi = n''' v$, so $V^{nf} n'''_{nf} = V^{m} n'''_m + V^{cl} n'''_{cl}$ implies
% $\phi^{nf}_T = \frac{V^m}{V^{nf}} \phi^m_T + \frac{V^m}{V^{nf}} \phi^m_T


\end{frame}
 
 
 
\end{document}

