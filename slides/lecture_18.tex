\documentclass[aspectratio=1610]{beamer}
\usepackage[T1]{fontenc}
\usetheme{wildcat}
\usetikzlibrary{arrows.meta,angles,quotes,calc,intersections,positioning}

\usepackage{amsmath,amssymb,amsfonts}
\usepackage{booktabs}
\usepackage{relsize}
\usepackage{pgfplots}
\pgfplotsset{compat=1.16}
\usepackage{array}
\usepackage{siunitx}
\usepackage{cancel}
\usepackage{jupyter}

\let\oldfootnotesize\footnotesize
\renewcommand*{\footnotesize}{\oldfootnotesize\tiny}

\def\mathdefault#1{#1}
\everymath=\expandafter{\the\everymath\displaystyle}


\title{Analysis of a LWR Unit Cell  \\
       {\small\it NE 630 - Lecture 18}}

\date{\input{term.txt} \\ {\footnotesize Git SHA: \input{git_sha.txt}}}

\author{Jeremy Roberts}


\definecolor{ksupurple}{HTML}{512888}
\definecolor{orange}{HTML}{CA7C1B}

\begin{document}

\begin{frame}
\titlepage
\end{frame}
 
 
%%%%%%%%%%%%%%%%%%%%%%%%%%%%%%%%%%%%%%%%%%%%%%%%%%%
\begin{frame}{Primary Objective}

Students will be able to 

\vfill

\begin{quote}
\textcolor{wcprimary}{construct a detailed model of a LWR unit cell
and compare the results to theoretical models based on the four-factor formula.
}
\end{quote}

\vfill 

\end{frame}

%%%%%%%%%%%%%%%%%%%%%%%%%%%%%%%%%%%%%%%%%%%%%%%%%%%%%%%%%%%%%%%%%%%%%%%%%%%%%%
\begin{frame}{Review: the four-factor formula}

Recall $k_{\infty} = \eta_T f p \varepsilon$, where
\begin{align*}
\tag{FNRP 4.49*}
\eta_T &= \frac{\nu\bar{\Sigma}_{fT}^f} {\bar{\Sigma}^f_{aT}} \\
\tag{FNRP 4.48*}
f &= \frac{1}{1 + \zeta \left(V_{nf} \bar{\Sigma}^{nf}_{aT} \big/ V_f \bar{\Sigma}^f_{aT} \right ) } \\
\tag{FNRP 4.28*}
p &= 1 - \frac{V_f \bar{\Sigma}^f_{aF} \bar{\phi}^{f}_{F}}
      {V_f \left[ \bar{\Sigma}^f_{aT} \bar{\phi}^{f}_{T} + \bar{\Sigma}^f_{aF} \bar{\phi}^{f}_{F} \right ]+ V_{nf} \bar{\Sigma}^{nf}_{aT} \bar{\phi}^{nf}_{T}} \\
\tag{FNRP 4.25*}
\varepsilon &= \frac{\nu\bar{\Sigma}^f_{fT} \bar{\phi}^{f}_{T} + \nu\bar{\Sigma}^f_{fF} \bar{\phi}^{f}_{F} }
   {\nu\bar{\Sigma}^f_{fT} \bar{\phi}^{f}_{T} } 
\end{align*}

\vfill 

Here, ${}^f$ and ${}^{nf}$ indicate ``fuel'' and ``non-fuel'', $_T$ and $_F$ indicate ``thermal'' and ``non-thermal'' (or $E > 1$ eV), and $\bar{\,}$ indicates averaging over space and/or energy.
\end{frame}



%%%%%%%%%%%%%%%%%%%%%%%%%%%%%%%%%%%%%%%%%%%%%%%%%%%%%%%%%%%%%%%%%%%%%%%%%%%%%%
\begin{frame}{Review: Averaging over Energy and Space}

Suppose we have the two-group fluxes and cross sections 
$\bar{\phi}^R_g$ and $\bar{\Sigma}^R_{x,g}$, $g \in \{1, 2\}$ in some
homogeneous region $R$.  Then the spectrum-averaged, {\it effective} cross section is
\begin{equation*}
\tag{like FNRP 4.5}
 \boxed{
  \bar{\Sigma}^R_{x} = 
    \frac{ \bar{\Sigma}^R_{x,1} \bar{\phi}^R_1 + \bar{\Sigma}^R_{x,2} \bar{\phi}^R_2} 
         { \bar{\phi}^R_1 +\bar{\phi}^R_2 }
  = \frac{ \bar{\Sigma}^R_{x,1} \bar{\phi}^R_1 + \bar{\Sigma}^R_{x,2} \bar{\phi}^R_2}
         { \bar{\phi}^R }} \, .
\end{equation*}
\vfill 
Now, suppose we have $\bar{\Sigma}^R_x$ and $\bar{\phi}^R$ for adjacent regions $R \in \{A, B\}$.  
Then the spatially {\it homogenized} cross section is
\begin{equation*}
\tag{like FNRP 4.17}
 \boxed{
   \bar{\Sigma}_x = 
     \frac{V^A \bar{\Sigma}^A_x  \bar{\phi}^A  + V^{B} \bar{\Sigma}^{B}_x  \bar{\phi}^{B}}
          { V^A \bar{\phi}^A + V^B \bar{\phi}^B } =
     \frac{V^A \bar{\Sigma}^A_x  \bar{\phi}^A  + V^{B} \bar{\Sigma}^{B}_x  \bar{\phi}^{B}}
          { V \bar{\phi} }     
  } \, .
\end{equation*}

These expressions emphasize {\it preserving reaction rates} while simplifying.
\end{frame}
  
%%%%%%%%%%%%%%%%%%%%%%%%%%%%%%%%%%%%%%%%%%%%%%%%%%%%%%%%%%%%%%%%%%%%%%%%%%%%%%
\begin{frame}[fragile]{An Approximate Model for $p$}

Starting with
\begin{align*}
\tag{FNRP 4.28*}
p &= 1 - \frac{V_f \bar{\Sigma}^f_{aI} \phi^{f}_{I}}
      {V_f \left[ \bar{\Sigma}^f_{aT} \phi^{f}_{T} + \bar{\Sigma}^f_{aI} \phi^{f}_{I} \right ]+ 
        V_{m} \bar{\Sigma}^{m}_{aT} \phi^{m}_{T}} \, ,
\end{align*}
assume all neutrons absorbed (denominator) come from the neutrons slowing down in the moderator ($q$),  or
\begin{align*}
\tag{FNRP 4.33*}
  p &\approx 1 - \frac{V_f \bar{\Sigma}^f_{aI} \phi^{f}_{I}}
      {V_{m} \xi^{m} \Sigma^{m}_s E \phi^{m}(E)} \\
\tag{FNRP 4.35*} 
  &= 1 - \frac{V_f  N_{fe}}{V_{m} \xi^{m} \Sigma^{m}_s  } 
         \underbrace{\int_I \frac{ \sigma^f_{a}(E) \phi^{f}(E) }{E \phi^m(E)} dE}_{I^{\text{het}}_a} \\
\tag{FNRP 4.40*} 
  p &\approx \exp \left (- \frac{V_f  N_{fe} I^{\text{het}}_a}{V_{m} \xi^{m} \Sigma^{m}_s  } \right ) \, . 
\end{align*}


\end{frame}



%%%%%%%%%%%%%%%%%%%%%%%%%%%%%%%%%%%%%%%%%%%%%%%%%%%%%%%%%%%%%%%%%%%%%%%%%%%%%%
\begin{frame}[fragile]{Evaluating $I^{\text{het}}$}
 
The heterogeneous resonance integral $I^{\text{het}}_a$ has been measured 
for a variety of cylindrical-fuel lattices and is found to be well
approximated by

\begin{equation*}
 I^{\text{het}}_a  = a + b\sqrt{\frac{4}{\rho_f D}} \, \text{[b]} \, ,
\end{equation*}

where $a$ and $b$ are fitted parameters, $\rho_f$ [g/cm$^3$] is the fuel mass density, and $D$~[cm] is the fuel diameter.

\vfill 

{\bf Example:} From Tabl 4.3, $a = 4.45$ and $b = 26.6$ for UO$_2$ fuel rods.  Estimate $p = \exp \left (- \frac{V_f  N_{fe} I^{\text{het}}_a}{V_{m} \xi^{m} \Sigma^{m}_s  } \right )$ for 4\% enriched PWR fuel assuming that $D = 1$ cm, $P=1.2$ cm, $\rho_{f} = 10$ g/cm$^3$, ${\xi \Sigma_s} = 1.28$. 
 
\pause 

{\bf Ans.} $p \approx 0.65)$
\end{frame}


%%%%%%%%%%%%%%%%%%%%%%%%%%%%%%%%%%%%%%%%%%%%%%%%%%%%%%%%%%%%%%%%%%%%%%%%%%%%%%
\begin{frame}[fragile]{An Approximate Model for $\varepsilon$}
 
\begin{align*}
\tag{FNRP 4.25*}
\varepsilon &= \frac{\nu\bar{\Sigma}^f_{fT} \bar{\phi}^{f}_{T} + \nu\bar{\Sigma}^f_{fF} \bar{\phi}^{f}_{F} }
   {\nu\bar{\Sigma}^f_{fT} \bar{\phi}^{f}_{T} } \\
   &= 1 + \frac{ (1-\tilde{e}) \bar{\nu}^{fe}_{T}\bar{\sigma}^{fe}_{fF}  + \tilde{e} \bar{\nu}^{fi}_{F}\bar{\sigma}^{fi}_{fF}}
                     { \tilde{e} \bar{\nu}^{fi}_{T}\bar{\sigma}^{fi}_{fT}   } \times \frac{\bar{\phi}^{f}_{F}}{\bar{\phi}^{f}_{T}}\\
\tag{FNRP 4.56}
   &\approx 1 + \frac{ (1-\tilde{e})}{\tilde{e}} \frac{\bar{\nu}^{fe} \bar{\sigma}^{fe}_{fF} }
                     {\bar{\nu}^{fi}\bar{\sigma}^{fi}_{fT}   }
\end{align*}

\vfill 
\pause 

{\bf Example:} Assuming $\tilde{e} = 4$\% and $\bar{\nu}=2.4$ for both nuclides, use the data from Table 3.2 to compute $\varepsilon$ using (a) Eq. (4.56) and (b) the middle equation with a fast-to-thermal ratio of 10. 

 
\pause 
{\tiny
{\bf Ans}.  (a) $1+(0.96 \cdot 0.304)/(0.04  \cdot 505) \approx 1.014)$\\
(b) $1+(0.96 \cdot 0.304 + 0.04\cdot 1.22)/(0.04  \cdot 505)\cdot 4 \approx 1.169$.

}

\end{frame}


 
 
\end{document}

