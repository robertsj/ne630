\documentclass[aspectratio=1610]{beamer}
\usepackage[T1]{fontenc}
\usetheme{wildcat}
\usetikzlibrary{arrows.meta,angles,quotes,calc,intersections,positioning}

\usepackage{amsmath,amssymb,amsfonts}
\usepackage{booktabs}
\usepackage{relsize}
\usepackage{pgfplots}
\pgfplotsset{compat=1.16}
\usepackage{array}
\usepackage{siunitx}
\usepackage{cancel}
\usepackage{jupyter}
\usepackage{minted}

\let\oldfootnotesize\footnotesize
\renewcommand*{\footnotesize}{\oldfootnotesize\tiny}

\def\mathdefault#1{#1}
\everymath=\expandafter{\the\everymath\displaystyle}


\title{Reactivity Impacts of \\Xenon and Samarium \\
       {\small\it NE 630 - Lecture 22}}

\date{\input{term.txt} \\ {\footnotesize Git SHA: \input{git_sha.txt}}}

\author{Jeremy Roberts}


\definecolor{ksupurple}{HTML}{512888}
\definecolor{orange}{HTML}{CA7C1B}
\definecolor{skyblue}{RGB}{180,220,255}

\begin{document}

\begin{frame}
\titlepage
\end{frame}
 
 
%%%%%%%%%%%%%%%%%%%%%%%%%%%%%%%%%%%%%%%%%%%%%%%%%%%
\begin{frame}{Primary Objective}

Students will be able to 

\vfill

\begin{quote}
\textcolor{wcprimary}{quantify the impact of xenon and samarium buildup on reactivity.
}
\end{quote}

\vfill 

\end{frame}

%%%%%%%%%%%%%%%%%%%%%%%%%%%%%%%%%%%%%%%%%%%%%%%%%%%%%%%%%%%%%%%%%%%%%%%%%%%%%%
\begin{frame}[fragile]{Review: Reactivity}
 
The multiplication factor ($k$) and the reactivity ($\rho$)
are related by
\begin{equation*}
  \rho = \frac{k - 1}{k} \qquad \pause \longrightarrow \qquad
  k =  \frac{1}{1-\rho} \, .
\end{equation*}

\vfill 
\pause 

Reactivity is given in units of $\Delta k/k$,  $\%\Delta k/k$, pcm, or \$,
and changes in $\rho$ are characterized by {\bf (integral) worths}  ($\Delta \rho$)
and {\bf reactivity coefficients} $\alpha_x = (\partial \rho/\partial x)|_{\text{non-}x\text{ fixed}}$.

 

\end{frame}


%%%%%%%%%%%%%%%%%%%%%%%%%%%%%%%%%%%%%%%%%%%%%%%%%%%%%%%%%%%%%%%%%%%%%%%%%%%%%%
\begin{frame}[fragile]{Reactivity in Time}
 
As a reactor operates (at power), its excess reactivity typically 
descreases because \textcolor{wcprimary}{${}^{235}$U is consumed} and
\textcolor{wcprimary}{parasitic fission products (FPs) are born}.  However,
that excess reactivity must \textcolor{wcalerted}{be balanced by control}.

\vfill 

To represent these processes, define a time-varying multiplication factor as
\begin{equation*}
\tag{FNRP 10.4*}
  k(t) = \frac{ \textcolor{wcprimary}{\nu\bar{\Sigma}_{fT}^F(t)} 
                \cdot p \cdot \varepsilon \cdot P_{\text{NL}}}
    {\bar{\Sigma}_{aT}^F(t)  + \textcolor{wcprimary}{\bar{\Sigma}_{aT}^{\text{FP}}(t)}
      + 
        \varsigma \frac{V_M}{V_F} \left ( \bar{\Sigma}_{aT}^{M} + \textcolor{wcalerted}{\bar{\Sigma}_{aT}^{\text{control}}(t) } \right )  } \, ,
\end{equation*}
where $P_{\text{NL}}$ is the probability that neutrons do not leak from the system.  

\vfill 

During operation, the value of $\textcolor{wcalerted}{\bar{\Sigma}_{aT}^{\text{control}}(t)}$ must be adjusted to maintain $k(t) = 1$.  The equation above implies that \textcolor{wcalerted}{adjustable control} is introduced within the moderator volume, consistent with typical LWRs, whether by way of changing the \textcolor{wcalerted}{rod position} or \textcolor{wcalerted}{soluble poison concentration}.

\end{frame}

 


%%%%%%%%%%%%%%%%%%%%%%%%%%%%%%%%%%%%%%%%%%%%%%%%%%%%%%%%%%%%%%%%%%%%%%%%%%%%%%
\begin{frame}[fragile]{Fission Products}
 

\begin{columns}[T,onlytextwidth]

\begin{column}{0.5\textwidth}
 
Fission of ${}^{235}$U (or ${}^{239}$Pu, ...) produces two
massive fragments called {\bf fission products} (FPs).

\vspace{0.25cm}

Often, FPs are
\begin{itemize}
 \item {\it radioactive} (which mode(s) dominate?)
 \item {\it parasitic 
       neutron absorbers} with large $\sigma_a$ (thermal and/or intermediate)
\end{itemize}

\vspace{0.25cm}

Each fission produces a given FP a fraction 
$\gamma_{fp}$ of the time called the {\bf fission yield}. 

\vspace{0.25cm}
{\bf Ponderable}: $\sum_{fp \, \in \text{all FPs}} \gamma_{fp} = ?$

 
\end{column}

\begin{column}{0.5\textwidth}

\begin{table}
 \begin{tabular}{cc}
  \toprule
  nuclide &  $\gamma_{fp}$ \\
  \midrule
  ${}^{134}$Xe & 0.078\\
  ${}^{133}$Cs & 0.067\\
  ${}^{95}$Mo & 0.065\\
  ${}^{137}$Cs & 0.063\\
  ${}^{135}$I & 0.063\\
  ${}^{99}$Tc & 0.061\\
  \bottomrule
 \end{tabular}
 \caption{Selected fission yields for ${}^{235}$U from the WIMS Library Update project; see \url{https://www-nds.iaea.org/wimsd/fpyield.htm}.}
\end{table}


\end{column}

\end{columns}

\end{frame}


%%%%%%%%%%%%%%%%%%%%%%%%%%%%%%%%%%%%%%%%%%%%%%%%%%%%%%%%%%%%%%%%%%%%%%%%%%%%%%
\begin{frame}[fragile]{Evolution of Fission-Product Concentrations}
 
For a FP produced directly from fission, the rate of change of 
its concentration can be modeled as

\begin{equation*}
 \tag{FNRP 10.7}
 \frac{dN_{fp}}{dt} = -\lambda_{fp} N_{fp}(t) - \overbrace{\bar{\sigma}^{fp}_a N_{fp}(t)}^{\bar{\Sigma}_{a}^{fp} (t)} \phi 
 + \gamma_{fp} \bar{\Sigma}_f \phi \, ,
\end{equation*}
where, as usual, the $\,\bar{}\,$ indicates spectrum averaging over all energies, i.e.,
\begin{equation*}
 \bar{\sigma}_{a}^{fp} = \frac{ \int^{\infty}_{0} \sigma_a^{fp}(E) \phi(E) dE}
 { \int^{\infty}_{0}\phi(E) dE} \, .
\end{equation*}


\end{frame}


%%%%%%%%%%%%%%%%%%%%%%%%%%%%%%%%%%%%%%%%%%%%%%%%%%%%%%%%%%%%%%%%%%%%%%%%%%%%%%
\begin{frame}[fragile]{The Most Important Duo: ${}^{135}$I and ${}^{135}$Xe}
 
\begin{columns}[T,onlytextwidth]

\begin{column}{0.5\textwidth}
 
\vspace{0.25cm}

Of all FPs,  \textcolor{wcprimary}{${}^{135}$Xe} is perhaps most important (see evidence at  right!), but its yield is only  \textcolor{wcprimary}{$\gamma_{Xe} \approx 0.00258$}.  

\vspace{0.5cm}

However, \textcolor{wcprimary}{${}^{135}$Xe} is a daughter of \textcolor{wcalerted}{${}^{135}$I}, whose yield is  \textcolor{wcalerted}{$\gamma_I \approx 0.063$}. 

\vspace{0.5cm}

Neglecting absorption in  \textcolor{wcalerted}{${}^{135}$I}, the evolution equations are 

\end{column}

\begin{column}{0.5\textwidth}

\includegraphics[width=0.95\textwidth]{./figures/xe135_comparison.pdf}
 
\end{column}

\end{columns}
 
\begin{align*}
\tag{FNRP 10.12}
 \textcolor{wcalerted}{\frac{dN_{I}}{dt}} &= 
   \textcolor{wcalerted}{-\lambda_{I} N_{I}(t)} + 
    \textcolor{wcalerted}{\gamma_{I} \bar{\Sigma}_f \phi} \, ,\\
 \tag{FNRP 10.13}
 \textcolor{wcprimary}{\frac{dN_{Xe}}{dt}} &= 
  \textcolor{wcprimary}{-\lambda_{Xe} N_{Xe}(t)} +  \textcolor{wcalerted}{\lambda_{I} N_{I}(t)} \\
    &\quad + \textcolor{wcprimary}{\gamma_{Xe} \bar{\Sigma}_f \phi} -  
    \textcolor{wcprimary}{\bar{\sigma}^{Xe}_a N_{Xe}(t)\phi} \, .
\end{align*}
 
\end{frame}

%%%%%%%%%%%%%%%%%%%%%%%%%%%%%%%%%%%%%%%%%%%%%%%%%%%%%%%%%%%%%%%%%%%%%%%%%%%%%%
\begin{frame}[fragile]{Equilibrium Xenon Worth}
 
Assuming constant $\bar{\Sigma}_f$ and $\phi$, the ${}^{135}$I and ${}^{135}$Xe
concentrations will reach saturation levels that satisfy
\begin{align*}
 0 &= 
   \textcolor{wcalerted}{-\lambda_{I} N_{I}(\infty)} + 
    \textcolor{wcalerted}{\gamma_{I} \bar{\Sigma}_f \phi} \, ,\\
 0 &= 
  \textcolor{wcprimary}{-\lambda_{Xe} N_{Xe}(\infty)} +  \textcolor{wcalerted}{\lambda_{I} N_{I}(\infty)} \\
    &\quad + \textcolor{wcprimary}{\gamma_{Xe} \bar{\Sigma}_f \phi} -  
    \textcolor{wcprimary}{\bar{\sigma}^{Xe}_a N_{Xe}(\infty)\phi} \, 
\end{align*}
or 
\begin{equation*}
   \textcolor{wcalerted}{N_{I}(\infty)}   = \textcolor{wcalerted}{\frac{\gamma_I \bar{\Sigma}_f \phi}  {\lambda_I}} \qquad \text{and} \qquad
    \textcolor{wcprimary}{N_{Xe}(\infty)}  =  \textcolor{wcprimary}{\frac{(\gamma_I + \gamma_{Xe})\bar{\Sigma}_f \phi}
                     { \lambda_{Xe}+\bar{\sigma}^{Xe}_a \phi }}
\end{equation*}

\vfill 

{\bf Example}. Suppose we have $\phi = 2\cdot 10^{14}\, \si{\per\centi\meter\squared\per\second}$, $\bar{\Sigma}_f=0.16 \, \si{\per\centi\meter}$, and $\bar{\sigma}^{Xe} = 2.6\cdot 10^6$ b (averaged over the core.  Compute $\bar{\Sigma}_a^{Xe}$.  What is the equilibrium Xe worth ($\Delta \rho_{Xe}$) if the reactor 
is initially critical with $\bar{\Sigma}_a=0.34\, \si{\per\centi\meter}$?\\
\pause

{\small {\bf Ans.} $\Sigma_a^{Xe} \approx 0.01 \si{\per\centi\meter}$, $\Delta \rho_{Xe} = \bar{\Sigma}_a^{Xe}/\bar{\Sigma}_a \approx -2.9 \, \% \Delta k/k$}


\end{frame}


%%%%%%%%%%%%%%%%%%%%%%%%%%%%%%%%%%%%%%%%%%%%%%%%%%%%%%%%%%%%%%%%%%%%%%%%%%%%%%
\begin{frame}[fragile]{The Xenon Transient: The Problem}
 
The solution of FNRP Eqs.~(10.12) and (10.13) is straightforward for 
a constant flux using normal techniques, but when the flux varies in 
time, numerical techniques are useful.  Almost all tools for solving
systems of initial-value problems require that we write the problem
in the form 
\begin{equation*}
 \frac{d\mathbf{y}}{dt} = \mathbf{f}(t, \mathbf{y}(t)) \, , 
   \qquad \mathbf{y}(0) = \mathbf{y}_0 \, .
\end{equation*}
For our problem, that looks like
\begin{align*}
 \frac{d}{dt} 
 \overbrace{
 \begin{pmatrix}
    N_I \\ N_{Xe} 
 \end{pmatrix}}^{\mathbf{y}(t)} = 
 \overbrace{
 \begin{pmatrix}
     -\lambda_{I}    &  0 \\
      \lambda_{I}   & -\lambda_{Xe} -  \bar{\sigma}^{Xe}_a \phi(t)
 \end{pmatrix} 
 \begin{pmatrix}
    N_I \\ N_{Xe} 
 \end{pmatrix}
 +
 \begin{pmatrix}
   \gamma_{I} \bar{\Sigma}_f \phi \\
   \gamma_{Xe} \bar{\Sigma}_f \phi
 \end{pmatrix}}^{\mathbf{f}(t,\mathbf{y}(t))}
\end{align*}
with initial concentration $N_{I,0}$ and $N_{Xe,0}$.  We'll 
start with the following:
\begin{minted}[frame=single,framesep=5pt]{python}
import numpy as np
import matplotlib.pyplot as plt
from scipy.integrate import solve_ivp
\end{minted} 
\end{frame}

%%%%%%%%%%%%%%%%%%%%%%%%%%%%%%%%%%%%%%%%%%%%%%%%%%%%%%%%%%%%%%%%%%%%%%%%%%%%%%
\begin{frame}[fragile]{The Xenon Transient: The Code}
 
\begin{minted}[frame=single,framesep=5pt]{python}
def dNdt(t, N, ϕ, σa_Xe, Σf):
  γ_I, γ_Xe = 0.0639, 0.00237
  λ_I, λ_Xe = 0.693/(6.7*3600), 0.693/(9.2*3600)
  σa_Xe = σa_Xe*1e-24
  N_I, N_Xe = N
  dN_Idt = -λ_I*N_I + γ_I*Σf*ϕ(t)
  dN_Xdt = -λ_X*N_Xe + λ_I*N_I + (γ_X*Σf-σa_X*N_Xe)*ϕ(t)
  return dN_Idt, dN_Xdt
hour=3600; day = 24*hour;
flux = lambda t: 2e14 if t < 3*day else 0.0
result = solve_ivp(fun=dNdt, t_span=(0.0, 5*day), 
  y0=(0, 0), args=(flux, 2.3e6, 0.16), max_step=1*hour)
t = result.t/3600; N_I, N_Xe = result.y
plt.plot(t, N_I, color=orange, lw=1, label="I")
plt.plot(t, N_Xe, color=purple, lw=1, ls="-.", label="Xe")
plt.xlabel("$t$ (h)"); plt.ylabel("N(t) (cm$^{-3}$)")
plt.legend();
\end{minted} 

\end{frame}


%%%%%%%%%%%%%%%%%%%%%%%%%%%%%%%%%%%%%%%%%%%%%%%%%%%%%%%%%%%%%%%%%%%%%%%%%%%%%%
\begin{frame}[fragile]{The Xenon Transient: The Result}
 
\centering
\includegraphics[width=0.9\textwidth]{./figures/Xe_transient.pdf}

\end{frame}

\end{document}

