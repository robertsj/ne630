\documentclass[aspectratio=1610]{beamer}
\usepackage[T1]{fontenc}
\usetheme{wildcat}
\usetikzlibrary{arrows.meta,angles,quotes,calc,intersections,positioning}

\usepackage{amsmath,amssymb,amsfonts}
\usepackage{booktabs}
\usepackage{relsize}
\usepackage{pgfplots}
\pgfplotsset{compat=1.16}
\usepackage{array}
\usepackage{siunitx}
\usepackage{cancel}
\usepackage{jupyter}
\usepackage{minted}

\let\oldfootnotesize\footnotesize
\renewcommand*{\footnotesize}{\oldfootnotesize\tiny}

\def\mathdefault#1{#1}
\everymath=\expandafter{\the\everymath\displaystyle}


\title{Linear Reactivity Model\\
       {\small\it NE 630 - Lecture 24}}
\date{\input{term.txt} \\ {\footnotesize Git SHA: \input{git_sha.txt}}}

\author{Jeremy Roberts}


\definecolor{ksupurple}{HTML}{512888}
\definecolor{orange}{HTML}{CA7C1B}
\definecolor{skyblue}{RGB}{180,220,255}

\begin{document}

\begin{frame}
\titlepage
\end{frame}
 
 
%%%%%%%%%%%%%%%%%%%%%%%%%%%%%%%%%%%%%%%%%%%%%%%%%%%
\begin{frame}{Primary Objective}

Students will be able to 

\vfill

\begin{quote}
\textcolor{wcprimary}{estimate cycle lengths and discharge burnups using the linear reactivity model.
}
\end{quote}

\vfill 

\end{frame}

%%%%%%%%%%%%%%%%%%%%%%%%%%%%%%%%%%%%%%%%%%%%%%%%%%%%%%%%%%%%%%%%%%%%%%%%%%%%%%
\begin{frame}[fragile]{Review: Reactivity and Depletion}

Our simple model for depletion consists of

\begin{equation*}
\tag{10.27 $\rightarrow$ 10.31}
\begin{split}
N^{\text{U}235}(t) = N^{\text{U}235}(0) e^{-\bar{\sigma}^{\text{U}235}_a \phi t} \, ,
\end{split}
\end{equation*}

\begin{equation*}
\tag{10.27 $\rightarrow$  10.34}
\begin{split}
N^{\text{U}238}(t) = N^{\text{U}238}(0) \, ,
\end{split}
\end{equation*}
and
\begin{equation*}
\tag{10.30 $\rightarrow$  10.37}
\begin{split}
N^{\text{Pu}239}(t) 
  = \frac{\bar{\sigma}^{\text{U}238}_{\gamma}}{\bar{\sigma}^{\text{Pu}239}_a} 
    N^{\text{U}238}(0) \left [1 - e^{-\bar{\sigma}^{\text{Pu}239}_a \phi t} \right ] \, .
\end{split}    
\end{equation*}
These can be used directly to evaluate 

 \begin{equation*}
  \tag{10.39}
  k(t) = \frac{\big(\nu\bar{\sigma}^{\text{U}235}_{fT} N^{\text{U}235}(t) + \nu\bar{\sigma}^{\text{Pu}239}_{fT} N^{\text{Pu}239}(t)\big)\varepsilon p P_{NL}}
  {\bar{\sigma}^{\text{U}235}_{aT} N^{\text{U}235}(t) + \bar{\sigma}^{\text{Pu}239}_{aT} N^{\text{Pu}239}(t) + \Sigma_{aT}^{\text{U}238} + \varsigma \frac{V_m }{V_f}\Sigma_{aT}^m} ]\, ,
\end{equation*}
with which $\rho = (k-1)/k$.

\end{frame}


%%%%%%%%%%%%%%%%%%%%%%%%%%%%%%%%%%%%%%%%%%%%%%%%%%%%%%%%%%%%%%%%%%%%%%%%%%%%%%
\begin{frame}[fragile]{Review: Some Basic Numbers}

Let's do some back-of-the-evelope calculations to get rough (1 or 2 significant figures) numbers to describe a typical PWR. 

\pause
\vfill

{\bf Example 1}: Estimate the total volume of fuel in a $P=3000~\si{\mega\watt_{th}}$ PWR. 
{\it Hint}:~there are $\approx 50,000$ rods roughly $3~\si{\meter}$ in height.

\pause
\vfill

\textcolor{wcprimary}{ {\bf Ans.} Assuming a fuel radius $r \approx 0.5~\si{\centi\meter}$, 
$V \approx (5\times 10^4)\,\pi r^2 h \approx  10^7~\si{\centi\meter^3}$.}

\vfill
\pause

{\bf Example 2}: Suppose the total fuel volume is $V = 10^7~\si{\centi\meter^3}$ and the effective macroscopic fission cross section is 
$\bar{\Sigma}_f = 0.05~\si{\per\centi\meter}$. Estimate the total flux $\phi$ for a $\dot{q}=3000~\si{\mega\watt_{th}}$ reactor.

\vfill
\pause


\textcolor{wcprimary}{ {\bf Ans.} With $P = V\,\phi\,\bar{\Sigma}_f\,\kappa$ and $\kappa \approx 200~\si{\mega\electronvolt} = 3.2\times 10^{-11}~\si{\joule}$, 
$\phi \approx \dfrac{3\times 10^9~\si{\watt}}{(10^7~\si{\centi\meter^3})(0.05~\si{\per\centi\meter})(3.2\times 10^{-11}~\si{\joule})}
\approx 2\times 10^{14}~\si{\per\centi\meter\squared\per\second}$.}

\vfill

\end{frame}
%%%%%%%%%%%%%%%%%%%%%%%%%%%%%%%%%%%%%%%%%%%%%%%%%%%%%%%%%%%%%%%%%%%%%%%%%%%%%%

%%%%%%%%%%%%%%%%%%%%%%%%%%%%%%%%%%%%%%%%%%%%%%%%%%%%%%%%%%%%%%%%%%%%%%%%%%%%%%
\begin{frame}[fragile]{Review: Some Basic Numbers}

{\bf Example 3}: What’s the total fuel mass corresponding to that volume? Assume $\rho_{\mathrm{UO_2}} = 10~\si{\gram\per\centi\meter^3}$.

\vfill
\pause

\textcolor{wcprimary}{{\bf Ans.} $m_{\mathrm{UO_2}} = (10^7~\si{\centi\meter^3})(10~\si{\gram\per\centi\meter^3})
= 10^8~\si{\gram} = 1.0\times 10^5~\si{\kilo\gram}$; the heavy-metal mass is $m_{\mathrm{U}} \approx \tfrac{238}{270}\,m_{\mathrm{UO_2}}
\approx 8.8\times 10^4~\si{\kilo\gram}$.}

\vfill
\pause

{\bf Example 4}: What is the fuel burnup after one day at that flux? After $5$ years?

\vfill
\pause

\textcolor{wcprimary}{{\bf Ans.} Since $P=3000~\si{\mega\watt_{th}}$, the energy in one day is $3000~\si{\mega\watt\day}$; thus 
$B_{1\,\mathrm{day}} \approx \dfrac{3000~\si{\mega\watt\day}}{8.8\times 10^4~\si{\kilogram}}
\approx 0.034~\si{\mega\watt\day\per\kilo\gram}$. Over 5 years at full power, 
$B_{5\,\mathrm{yr}} \approx 5(365) \times 0.034 \approx 62~\si{\mega\watt\day\per\kilo\gram}$.}

\vfill

\end{frame}
%%%%%%%%%%%%%%%%%%%%%%%%%%%%%%%%%%%%%%%%%%%%%%%%%%%%%%%%%%%%%%%%%%%%%%%%%%%%%%

%%%%%%%%%%%%%%%%%%%%%%%%%%%%%%%%%%%%%%%%%%%%%%%%%%%%%%%%%%%%%%%%%%%%%%%%%%%%%%
\begin{frame}[fragile]{A More Realistic Reactivity}

The $\rho(t)$ we computed using the simplified depletion model is reasonable for small 
burnups (or times), but we'll use the more realistic reactivity curve shown below
to explore the linear reactivity model (LRM) and its three assumptions:

\begin{center}
\includegraphics[width=0.9\textwidth]{./figures/realistic_rho.pdf}
\end{center}

\vfill 
{\bf Example}: Using the grid lines to guide you, at 
what burnup does (a) the reactivity go subcritical? (b) $\rho_{\infty} = \rho_L$?

\pause 
\textcolor{wcprimary}{{\small {\bf Ans}. About 30 MWd/kg and 27.5 MWd/kg, respectively.}}

\end{frame}
%%%%%%%%%%%%%%%%%%%%%%%%%%%%%%%%%%%%%%%%%%%%%%%%%%%%%%%%%%%%%%%%%%%%%%%%%%%%%%


%%%%%%%%%%%%%%%%%%%%%%%%%%%%%%%%%%%%%%%%%%%%%%%%%%%%%%%%%%%%%%%%%%%%%%%%%%%%%%
\begin{frame}[fragile]{LRM: Assumption 1}

\begin{center}
{\bf {\large \textcolor{wcprimary}{A reactor core is fueled with $N$ batches of identical fuel.}}}
\end{center} 

\vfill 

{\it Some nomenclature by example}:
\vfill

\begin{itemize}
\item Suppose that $N = 3$. When a reactor like Wolf Creek is refueled, one third of the core is replaced with fresh fuel, one third is ``once burned'', and one third is ``twice burned''.  The fuel that is removed from the core is ``thrice burned''.  
\item We say the reactor is at its {\bf beginning of cycle (BOC)} after refueling and at its  {\bf end of cycle (EOC)} before refueling.  The EOC occurs when the core has no excess reactivity.  The BOC may or may not correspond to the core's maximum excess reactivity (due to burnable poisons).
\item The burnup $B_c$ that takes us from the BOC to the EOC is called the  {\bf cycle length}.  The burnup of the ``thrice burned'' fuel is equal to burnup accumulated over all 3 cycles and is called the  {\bf discharge burnup} or $B_d$.
\end{itemize}

\end{frame}
%%%%%%%%%%%%%%%%%%%%%%%%%%%%%%%%%%%%%%%%%%%%%%%%%%%%%%%%%%%%%%%%%%%%%%%%%%%%%%

%%%%%%%%%%%%%%%%%%%%%%%%%%%%%%%%%%%%%%%%%%%%%%%%%%%%%%%%%%%%%%%%%%%%%%%%%%%%%%
\begin{frame}[fragile]{LRM: Assumption 2}

\begin{center}
{\bf {\large \textcolor{wcprimary}{
The fuel reactivity is linear with burnup: 
 $\boxed{\rho_{\infty}(B) = \rho_0 - aB}$.
}}}
\end{center} 

\vfill 

The $\infty$ indicates that the reactivity is computed for an infinite reactor, e.g., the reflected unit cells that we've been using in OpenMC.  A leakage penalty $\rho_L$ can be accounted for later.

\end{frame}
%%%%%%%%%%%%%%%%%%%%%%%%%%%%%%%%%%%%%%%%%%%%%%%%%%%%%%%%%%%%%%%%%%%%%%%%%%%%%%

%%%%%%%%%%%%%%%%%%%%%%%%%%%%%%%%%%%%%%%%%%%%%%%%%%%%%%%%%%%%%%%%%%%%%%%%%%%%%%
\begin{frame}[fragile]{LRM: Assumption 3}

\begin{center}
{\bf {\large \textcolor{wcprimary}{
The core reactivity is  
 $\boxed{\rho_{\text{core}} = \sum^N_{i=1} w_i \rho_i - \rho_{L}}$ where $\boxed{\sum_{i=1}^N w_i = 1}$.
}}}
\end{center} 

\vfill 

In other words, the core reactivity is a weighted sum of the batch reactivities minus a leakage penalty.  The simplest LRM sets $\boxed{w_i = w = 1/N}$.

\end{frame}
%%%%%%%%%%%%%%%%%%%%%%%%%%%%%%%%%%%%%%%%%%%%%%%%%%%%%%%%%%%%%%%%%%%%%%%%%%%%%%

%%%%%%%%%%%%%%%%%%%%%%%%%%%%%%%%%%%%%%%%%%%%%%%%%%%%%%%%%%%%%%%%%%%%%%%%%%%%%%
\begin{frame}[fragile]{Using LRM to Determine $B_c$ and $B_d$}
 
A primary use of LRM is to determine the {\bf cycle length} and {\bf discharge burnup} for $N$-batch cores.

\vspace{0.25cm}

{\bf Example}: Refer back to example 4 and write down a linear model for the fuel reactivity.  Then, assume $N=2$, $\rho_L=0.05$, and $B_c = 10$ GWd/MTU.  What is the core reactivity at 0 and 10 GWd/MTU? Is this cycle well designed?

\vfill 

\pause 

\textcolor{wcprimary}{{\small {\bf Ans.} Given $\rho = \rho_0 - A  B$, we use the points at $B=20$ and $B=50$ to estimate $A = (-0.1 - 0.1)/(50-20) \approx 0.00667$.  Then the intercept (or initial reactivity) satisfies $\rho(20) = 0.1 = \rho_0 - (0.00667)10$ so that $\rho_0 = 0.233333$.  Finally,the BOC and EOC reactivities are
\begin{align*}
 \rho_{\text{core}}^{\text{BOC}} &= \frac{1}{2} ( [\rho_0 - A\cdot 0] + [\rho_0 - A\cdot 10]) = 0.2 - 0.05 = 0.15 \\
 \rho_{\text{core}}^{\text{EOC}} &= \frac{1}{2} ( [\rho_0 - A\cdot 10] + [\rho_0 - A\cdot 20]) = 0.13 - 0.05 = 0.08
\end{align*}
Because $\rho_{\text{core}}^{\text{EOC}} > 0$, our cycle length should be longer or we're ``throwing away'' fuel!}}

\end{frame}
%%%%%%%%%%%%%%%%%%%%%%%%%%%%%%%%%%%%%%%%%%%%%%%%%%%%%%%%%%%%%%%%%%%%%%%%%%%%%%


%%%%%%%%%%%%%%%%%%%%%%%%%%%%%%%%%%%%%%%%%%%%%%%%%%%%%%%%%%%%%%%%%%%%%%%%%%%%%
\begin{frame}[fragile]{Using LRM to Determine $B_c$ and $B_d$}

{\bf Example cont.}: Now, determine the cycle length and discharge burnup for the two-batch core (such that $\rho_{\text{core}}^{\text{EOC}} = 0$).  How do these values compare to a single-batch core?

\vfill 

\pause 

\textcolor{wcprimary}{{\small {\bf Ans.}  
With $\rho(B)=\rho_0-AB$ (from the previous slide, $A\simeq 0.006667\ \text{(GWd/MTU)}^{-1}$ and $\rho_0\simeq 0.233333$) and a constant leakage penalty $\rho_L=0.05$, a 2-batch core at EOC has one batch at $B_c$ and the other at $2B_c$, so the weighted average is
\[
\bar{\rho}_\infty^{\text{EOC}}
=\frac{1}{2}\!\left[(\rho_0-AB_c)+(\rho_0-A(2B_c))\right]
=\rho_0-\tfrac{3}{2}AB_c.
\]
Setting $\rho_{\text{core}}^{\text{EOC}}=\bar{\rho}_\infty^{\text{EOC}}-\rho_L=0$ gives
\[
B_c=\frac{2(\rho_0-\rho_L)}{3A}
\quad\Rightarrow\quad
B_c\approx \frac{2(0.233333-0.05)}{3(0.006667)}
\approx \boxed{18.33\ \text{GWd/MTU}}.
\]
Thus the discharge burnup is $B_d=2B_c\approx \boxed{36.67\ \text{GWd/MTU}}$.
For $N=1$, $\rho_{\text{core}}^{\text{EOC}}=(\rho_0-AB_c)-\rho_L=0$,
so $B_c^{(1)} =  B_d^{(1)} \approx 27.50\ \text{GWd/MTU}$.
Hence a 2-batch core has a shorter cycle ($B_c^{(2)}=\tfrac{2}{3}B_c^{(1)}$) but a higher discharge burnup ($B_d^{(2)}=\tfrac{4}{3}B_d^{(1)}$; about 33\% higher).
}}
\end{frame}
%%%%%%%%%%%%%%%%%%%%%%%%%%%%%%%%%%%%%%%%%%%%%%%%%%%%%%%%%%%%%%%%%%%%%%%%%%%%%


%%%%%%%%%%%%%%%%%%%%%%%%%%%%%%%%%%%%%%%%%%%%%%%%%%%%%%%%%%%%%%%%%%%%%%%%%%%%%
\begin{frame}[fragile]{$B_c$ and $B_d$ for $N$-batch Cores}

More generally, the EOC reactivity of each batch in an $N$-batch core is

\begin{align*}
  \rho_1 &= \rho_0 -  AB_c \\
  \rho_2 &= \rho_0 - 2AB_c \\
         & \vdots \\
  \rho_N &= \rho_0 - NAB_c \, ,
\end{align*}

and the core EOC reactivity is 

\begin{equation*}
  \rho_{\text{core}} = \frac{1}{N} [ N\rho_0 - (A + 2A + \ldots + NA)B_c ] - \rho_L = 0 \, .
\end{equation*}

For $\rho_L = 0$, we have $B_c^{(N)} = \frac{2}{N+1}\,B_c^{(1)}$.

\vfill 
\pause 

{\bf Example}:  How is $B_d^{(N)}$ related to $B_d^{(1)}$?
\pause 

\textcolor{wcprimary}{{\small {\bf Ans.} $B_d^{(N)} = \frac{2N}{N+1}\,B_d^{(1)}$.}}


\end{frame}
 
\end{document}

