\documentclass[aspectratio=1610]{beamer}
\usepackage[T1]{fontenc}
\usetheme{wildcat}
\usetikzlibrary{arrows.meta,angles,quotes,calc,intersections,positioning}

\usepackage{amsmath,amssymb,amsfonts}
\usepackage{booktabs}
\usepackage{relsize}
\usepackage{pgfplots}
\pgfplotsset{compat=1.16}
\usepackage{array}
\usepackage{siunitx}
\usepackage{cancel}
\usepackage{jupyter}
\usepackage{minted}
\usepackage{xfrac}

\let\oldfootnotesize\footnotesize
\renewcommand*{\footnotesize}{\oldfootnotesize\tiny}

\def\mathdefault#1{#1}
\everymath=\expandafter{\the\everymath\displaystyle}


\title{Neutron Kinetics without Multiplication\\
       {\small\it NE 630 - Lecture 25}}

\date{\input{term.txt} \\ {\footnotesize Git SHA: \input{git_sha.txt}}}

\author{Jeremy Roberts}


\definecolor{ksupurple}{HTML}{512888}
\definecolor{orange}{HTML}{CA7C1B}
\definecolor{skyblue}{RGB}{180,220,255}

\begin{document}

\begin{frame}
\titlepage
\end{frame}
 
 
%%%%%%%%%%%%%%%%%%%%%%%%%%%%%%%%%%%%%%%%%%%%%%%%%%%
\begin{frame}{Primary Objective}

Students will be able to 

\vfill

\begin{quote}
\textcolor{wcprimary}{write down, solve, and explain each term of a first-order, ordinary differential 
equation that models the neutron flux, the neutron density, or the number of 
neutrons in a non-multiplying system as a function of time.
}
\end{quote}

\vfill 

\end{frame}

%%%%%%%%%%%%%%%%%%%%%%%%%%%%%%%%%%%%%%%%%%%%%%%%%%%%%%%%%%%%%%%%%%%%%%%%%%%%%%
\begin{frame}[fragile]{Review: Linear Reactivity Model}

LRM is based on three assumptions:
\vfill
\begin{enumerate}
 \item A reactor core is fueled with $N$ batches of identical fuel.
 \vfill 
 \item The fuel reactivity is linear with burnup: 
 $\boxed{\rho_{\infty}(B) = \rho_0 - AB}$.
 \vfill
 \item The core reactivity is  
 $\boxed{\rho_{\text{core}} = \sum^N_{i=1} w_i \rho_i - \rho_{L}}$ where $\boxed{\sum_{i=1}^N w_i = 1}$. 
\end{enumerate}
\vfill 

Important terms: $N$-batch cycle length $B^{(N)}_c$, discharge burnup $B^{(N)}_d$, leakage penalty $\rho_L$.

\end{frame}

%%%%%%%%%%%%%%%%%%%%%%%%%%%%%%%%%%%%%%%%%%%%%%%%%%%%%%%%%%%%%%%%%%%%%%%%%%%%%
\begin{frame}[fragile]{Review: $B_c$ and $B_d$ for $N$-batch Cores}

The EOC reactivities of each batch in an $N$-batch core are

\begin{align*}
  \rho_1 &= \rho_0 -  AB^{(N)}_c \\
  \rho_2 &= \rho_0 - 2AB^{(N)}_c \\
         & \vdots \\
  \rho_N &= \rho_0 - NAB^{(N)}_c \, ,
\end{align*}

and the core EOC reactivity is

\begin{equation*}
  \rho_{\text{core}} = \frac{1}{N} [ N\rho_0 - (A + 2A + \ldots + NA)B_c ] - \rho_L = 0 \, .
\end{equation*}

For $\rho_L = 0$, we have $B_c^{(N)} = \frac{2}{N+1}\,B_c^{(1)}$.

\vfill 
\pause 

{\bf Example}:  How is $B_d^{(N)}$ related to $B_d^{(1)}$ if $\rho_L = 0$?
\pause 

\textcolor{wcprimary}{{\small {\bf Ans.} $B_d^{(N-\text{batch})} = \frac{2N}{N+1}\,B_d^{(1)}$.}}


\end{frame}


\begin{frame}[fragile]{Revisiting the BIG PICTURE\texttrademark}

In the course introduction, I wrote the following horror:
\begin{align*}
\frac{1}{v}\frac{\partial \phi}{\partial t}
&\;-\;\nabla \!\cdot\! D(\mathbf r,E)\,\nabla \phi(\mathbf r,E,t)
+\Sigma_t(\mathbf r,E)\,\phi(\mathbf r,E,t) \\
&= \int_0^\infty \!\Sigma_s(\mathbf r,E'\!\to\!E)\,\phi(\mathbf r,E',t)\,dE' \\
&+ \frac{\chi(E)}{\textcolor{wcprimary}{k_{\text{eff}}}}\!\int_0^\infty \!\bar\nu(\mathbf r,E')\Sigma_f(\mathbf r,E')\phi(\mathbf r,E',t)\,dE' 
 + \textcolor{wcalerted}{S(\mathbf r,E,t)}\, ,
\end{align*}
 

which is the time-dependent, multigroup diffusion equation.  
\vfill 

As for steady-state problems, we'll have 
$\textcolor{wcalerted}{S(\mathbf r,E,t)}$ for source-driven problems (with or without multiplication) and 
$\textcolor{wcprimary}{k_{\text{eff}}}$ for source-free problems with multiplication.  For time-dependent
problems, $\textcolor{wcprimary}{k_{\text{eff}}}$ can first be computed for a steady-state model and then used
to enforce a steady-state, initial condition.


\end{frame}
%%%%%%%%%%%%%%%%%%%%%%%%%%%%%%%%%%%%%%%%%%%%%%%%%%%%%%%%%%%%%%%%%%%%%%%%%%%%%%

%%%%%%%%%%%%%%%%%%%%%%%%%%%%%%%%%%%%%%%%%%%%%%%%%%%%%%%%%%%%%%%%%%%%%%%%%%%%%%
\begin{frame}[fragile]{The Kinetics Equation}

We'll still ignore space. We won't {\it ignore} energy, but instead integrate it away:

\begin{align*}
\int^{\infty}_0 dE \Bigg \{
\frac{1}{v} \frac{\partial \phi}{\partial t} 
 &+ \Sigma_t(E) \phi( E, t) \\
  &= \int^{\infty}_0 \Sigma_s(E'\to E) \phi(E', t) dE' \\
  &+ \frac{\chi(E)}{k_{\text{eff}}} \int^{\infty}_0 \bar{\nu}(E')\Sigma_f(E') \phi(E', t) dE'  \\
  &+ S(E, t) \Bigg \} \, .
\end{align*}

\end{frame}

\begin{frame}[fragile]{Flux Separability}

If we assume\footnote{In other words, we'll assume the spectrum is independent of time but the overall neutron population is not.  This is an imperfect but reasonably accurate assumption for many systems.} that $\phi(E, t) = \phi(t)\varphi(E)$, then we can write:

\begin{align*}
\frac{d\phi}{dt} \times \int^{\infty}_0  
\left ( \frac{\varphi(E)}{v(E)} \right ) dE
 + \bar{\Sigma}_t \phi(t) 
  = \bar{\Sigma}_s \phi(t)  + \frac{1}{k_{\text{eff}}} \bar{\nu}\bar{\Sigma}_f \phi(t) + S(t)  \, .
\end{align*}
or
\begin{align*}
\tag{25.1}
  \boxed{ \frac{1}{\bar{v}}\frac{d\phi}{dt}
 + \bar{\Sigma}_t \phi(t) 
  = \bar{\Sigma}_s \phi(t)  + \frac{1}{k_{\text{eff}}} \bar{\nu}\bar{\Sigma}_f \phi(t) + S(t)  } \, .
\end{align*}
where %the effective macroscopic cross section and mean neutron speed are
\begin{equation*}
\bar\Sigma_x=\frac{\displaystyle \int_0^\infty \Sigma_x(E)\,\varphi(E)\,dE}{\displaystyle \int_0^\infty \varphi(E)\,dE}
\qquad \text{and} \qquad
\tag{25.2}
\boxed{\;\bar v=\frac{\displaystyle \int_0^\infty \varphi(E)\,dE}{\displaystyle \int_0^\infty \frac{\varphi(E)}{v(E)}\,dE}\;}
\end{equation*}

\end{frame}
%%%%%%%%%%%%%%%%%%%%%%%%%%%%%%%%%%%%%%%%%%%%%%%%%%%%%%%%%%%%%%%%%%%%%%%%%%%%%%

\begin{frame}[fragile]{The Mean Neutron Speed}
 
%$\bar v=\frac{ \int_0^\infty \varphi(E)\,dE}{ \int_0^\infty \frac{\varphi(E)}{v(E)}\,dE}$
The mean speed $\bar{v}$ represents a spectrum-averaged speed, and its numerical values are worth 
quantifying over the thermal, intermediate, and fast energy ranges we've
previously explored.

\vfill 
\pause

{\bf Example}:  Assuming classical physics (i.e., $E = (1/2)mv^2$) applies, 
derive $v(E)~\si{\centi\meter\per\second}$, where $E$ is given in \si{\electronvolt}.

\vfill 
\pause

\textcolor{wcprimary}{ 
\textbf{Ans.} The neutron mass is $m_n = 939.655 \, \si{\mega\electronvolt}/c^2$. Thus
\begin{align*}
 v(E)/c = \beta(E)
        &= \sqrt{ (2 E~\si{\electronvolt} \cdot 10^{-6} \si{\mega\electronvolt\per\electronvolt} ) / (939.655\,  \si{\mega\electronvolt}/c^2)} \\
        &= 4.614\cdot 10^{-5}\sqrt{E/c^2} \, , 
\end{align*}
and, with $c=3\cdot 10^{10}~\si{\centi\meter\per\second}$, we have
\begin{equation*}
 \boxed{v(E) = 1.384\cdot 10^{6}\sqrt{E} \, \si{\centi\meter\per\second} } \, .
\end{equation*}
A helpful sanity check: $v(0.0253~\si{\electronvolt}) \approx 220,000~\si{\centi\meter\per\second} = 2200~\si{\meter\per\second}$.
}

\end{frame}

\begin{frame}[fragile]{The Mean Neutron Speed}
 
{\bf Example}:  Assuming $v(E) = a\sqrt{E} = 1.384\cdot 10^{6}\sqrt{E}~\si{\centi\meter\per\second}$ ($E$ in \si{\electronvolt}), compute the mean neutron speed if 
\begin{enumerate}[label=(\alph*)]
 \item $\varphi(E) = E e^{-E/kT}$, $10^{-3} < E < 1$ eV, $T = 294$ K, $k = 8.6173\cdot 10^{-5}~\si{\electronvolt\per\kelvin}$.
 \item $\varphi(E) = 1/E$, $1 < E < 10^5$ eV
 \item $\varphi(E) = \chi(E)$ (as defined by FNRP 2.31),  $10^5 < E < 10^7$ eV .
 \item $\varphi(E)$, piecewise as above, with $\phi_T=\phi_I=\phi_F$.
\end{enumerate}

\vfill 
\pause

\textcolor{wcprimary}{\small 
  \textbf{Ans.} Either numerical or, with suitable approximations, analytical techniques can be used.  Here, just the numerical 
  results based on {\tt scipy.integrate.quad} are given:
\begin{enumerate}[label=(\alph*)]
 \item $2.498\cdot 10^{5}~\si{\centi\meter\per\second}$  
 \item $7.992\cdot 10^{7}~\si{\centi\meter\per\second}$
 \item $1.513\cdot 10^{9}~\si{\centi\meter\per\second}$
 \item $7.266\cdot 10^{5}~\si{\centi\meter\per\second}$ (notice how heavily weighted thermal neutrons are!)
\end{enumerate}
}

\vfill 


\end{frame}


%%%%%%%%%%%%%%%%%%%%%%%%%%%%%%%%%%%%%%%%%%%%%%%%%%%%%%%%%%%%%%%%%%%%%%%%%%%%%%
\begin{frame}[fragile]{Kinetics Sans Multiplication}

Set $\bar\Sigma_f=0$ in (25.1) to obtain
\begin{equation*}
\tag{25.3}
\frac{1}{\bar v}\frac{d\phi}{dt} + \bar\Sigma_a\,\phi(t) = S(t)\, .
\end{equation*}

\vfill 
\pause 

{\bf Example}: Rewrite (25.3) in terms of the neutron density $n'''(t)$.

\pause

\textcolor{wcprimary}{{\small 
{\bf Ans.}
Recall $\phi = n''' \bar{v}$.  Thus,
\begin{equation*}
\tag{Like FNRP 5.2}
 \sfrac{dn'''}{dt} + \bar\Sigma_a \bar{v} n'''(t)  = S(t) \, ,
\end{equation*}
which is equivalent to FNRP (5.2) both sides are multiplied by the system volume.
}}

\vfill 
\pause 

{\bf Example}: Suppose that $S(t) = S_0 \delta(t)$ and that $n'''(0) = 0$. Find $n'''(t)$.

\pause 

\textcolor{wcprimary}{{\small 
{\bf Ans.}
$n'''(t) = S_0 e^{-\Sigma_a \bar{v} t}$
}}


\end{frame}
%%%%%%%%%%%%%%%%%%%%%%%%%%%%%%%%%%%%%%%%%%%%%%%%%%%%%%%%%%%%%%%%%%%%%%%%%%%%%%

%%%%%%%%%%%%%%%%%%%%%%%%%%%%%%%%%%%%%%%%%%%%%%%%%%%%%%%%%%%%%%%%%%%%%%%%%%%%%%
\begin{frame}[fragile]{The Neutron Lifetime}

If we consider $n(t)$ (or $n'''(t)$) as just found, or, if $S_0 = 0$ and we let $n(0) = n_0$ such that
\begin{equation*}
 n(t) = n_0 e^{-\Sigma_a \bar{v} t} \, ,
\end{equation*}
one may immediately see a similarity to 
\begin{equation*}
    I(x) = I_0 e^{-\Sigma x} \qquad \text{and} \qquad n(t) = n(0) e^{-\lambda t} \, .
\end{equation*}

\pause

All three represent {\it exponential attenuation} in which
\begin{itemize}
 \item $\Sigma^{-1} = \tau$ is the mean(-free) path to collision
 \item $\lambda^{-1} = \bar{t}$ is the mean time to decay
 \pause 
 \item  $\boxed{l_{\infty} = [\bar{\Sigma}_a \bar{v}]^{-1} = \bar{t}}$ is also a mean time, but here it is the time a neutron travels, on the average, before being absorbed (and, hence, ``dying'').
\end{itemize}

\vfill 
\pause 

{\bf Example}. What's $l_{\infty}$ for thermal neutrons in room-temperature water?  Assume $\bar{\sigma}^{\text{H}_2\text{O}}_a = 0.6$ [b].
\pause 
\textcolor{wcprimary}{{\small 
{\bf Ans.}
$N=6.022\cdot 10^{23}/18$, so $\Sigma_a = N\,\bar\sigma_a \approx 2.0\times10^{-2}\ \text{cm}^{-1}$.  For thermal neutrons, $\bar{v}\approx2.5\times10^5$ cm/s.  Thus $l_{\infty} \approx 2.0\times10^{-4} = 200$ $\mu$s.  Compare to  \url{https://journals.aps.org/pr/abstract/10.1103/PhysRev.61.152}.
}}
 

\end{frame}

 
\end{document}

