\documentclass[aspectratio=1610]{beamer}
\usepackage[T1]{fontenc}
\usetheme{wildcat}
\usetikzlibrary{arrows.meta,angles,quotes,calc,intersections,positioning}

\usepackage{amsmath,amssymb,amsfonts}
\usepackage{booktabs}
\usepackage{relsize}
\usepackage{pgfplots}
\pgfplotsset{compat=1.16}
\usepackage{array}
\usepackage{siunitx}
\usepackage{cancel}
\usepackage{jupyter}
\usepackage{minted}
\usepackage{xfrac}
\usepackage{mathtools}

\let\oldfootnotesize\footnotesize
\renewcommand*{\footnotesize}{\oldfootnotesize\tiny}

\def\mathdefault#1{#1}
\everymath=\expandafter{\the\everymath\displaystyle}


\title{Neutron Kinetics with Multiplication\\
       {\small\it NE 630 - Lecture 26}}

\date{\input{term.txt} \\ {\footnotesize Git SHA: \input{git_sha.txt}}}

\author{Jeremy Roberts}


\definecolor{ksupurple}{HTML}{512888}
\definecolor{orange}{HTML}{CA7C1B}
\definecolor{skyblue}{RGB}{180,220,255}

\begin{document}

\begin{frame}
\titlepage
\end{frame}
 
 
%%%%%%%%%%%%%%%%%%%%%%%%%%%%%%%%%%%%%%%%%%%%%%%%%%%
\begin{frame}{Primary Objective}

Students will be able to 

\vfill

\begin{quote}
\textcolor{wcprimary}{write down, solve, and explain each term of a first-order, ordinary differential 
equation that models the neutron flux, the neutron density, or the number of 
neutrons in a multiplying system as a function of time.
}
\end{quote}

\vfill 

\end{frame}

%%%%%%%%%%%%%%%%%%%%%%%%%%%%%%%%%%%%%%%%%%%%%%%%%%%%%%%%%%%%%%%%%%%%%%%%%%%%%%
\begin{frame}[fragile]{Review: Kinetics without Multiplication}

By first eliminating space from the diffusion equation 
and then assuming $\phi(E, t) = \phi(t)\varphi(E)$ for 
some known $\varphi(E)$, we found
\begin{equation*}
\tag{25.1 \to 25.3}
 \frac{1}{\bar{v}}\frac{d\phi}{dt}
 + \bar{\Sigma}_t \phi(t) 
  = \bar{\Sigma}_s \phi(t) 
  + \cancelto{0}{\frac{1}{\textcolor{wcprimary}{k_{\text{eff}}}} \bar{\nu}\bar{\Sigma}_f}  \phi(t) + \textcolor{wcalerted}{S(t)}\, ,
\end{equation*}
or, with $\phi = n''' \bar{v}$,
\begin{equation*}
\tag{Like FNRP 5.2 and 5.7}
 \frac{dn'''}{dt} +  \left ( \frac{1}{l_{\infty}} \right ) n'''(t)  = S(t) \, ,
\end{equation*}
where the mean neutron speed and lifetime are
\begin{equation*}
 \bar v=\frac{\displaystyle \int_0^\infty \varphi(E)\,dE}{\displaystyle \int_0^\infty \frac{\varphi(E)}{v(E)}\,dE}\;
 \quad \text{and} \quad 
 l_{\infty} = \frac{1}{\bar{v} \bar{\Sigma}_a} \, ,
\end{equation*}
and the neutron speed is $v(E) \approx 1.384\cdot 10^{6}\sqrt{E} \, \si{\centi\meter\per\second}$ for $E$ in \si{\electronvolt}.

\end{frame}


%%%%%%%%%%%%%%%%%%%%%%%%%%%%%%%%%%%%%%%%%%%%%%%%%%%%%%%%%%%%%%%%%%%%%%%%%%%%%%
\begin{frame}[fragile]{Review: The Neutron Lifetime}

{\bf Example}. What's $l_{\infty} = \frac{1}{\bar{v} \bar{\Sigma}_a}$ for thermal neutrons in room-temperature water?  Assume $\bar{\sigma}^{\text{H}_2\text{O}}_a = 0.6$ [b].

\pause 
\vspace{0.25cm}

\textcolor{wcprimary}{ 
{\bf Ans.}
$N=6.022\cdot 10^{23}/18$, so $\Sigma_a = N\,\bar\sigma_a \approx 2.0\times10^{-2}\ \text{cm}^{-1}$.  For thermal neutrons, $\bar{v}\approx2.5\times10^5$ cm/s.  Thus $l_{\infty} \approx 2.0\times10^{-4} = 200$ $\mu$s.  Compare to  \url{https://journals.aps.org/pr/abstract/10.1103/PhysRev.61.152}.
}
 

\end{frame}

%%%%%%%%%%%%%%%%%%%%%%%%%%%%%%%%%%%%%%%%%%%%%%%%%%%%%%%%%%%%%%%%%%%%%%%%%%%%%%
\begin{frame}[fragile]{Adding Back Multiplication}

With multiplication, we have\footnote{For source-free, {\it steady-state} problems, the multiplication factor $\textcolor{wcprimary}{k_{\text{eff}}}$ allowed us to adjust the fission rate and enforce a steady state solution.  For kinetics problems, we need no such adjustment!}
\begin{equation*}
\tag{26.1}
  \frac{1}{\bar{v}}\frac{d\phi}{dt}
 + \bar{\Sigma}_t \phi(t) 
  = \bar{\Sigma}_s \phi(t)  +  \bar{\nu}\bar{\Sigma}_f \phi(t) +  S(t)  \, .
\end{equation*}
   

\vfill 

Let $\bar{\Sigma}_a = \bar{\Sigma}_t - \bar{\Sigma}_s$ and $\phi(t) = n'''(t)\bar{v}$ in (26.1). Then, multiply by volume so that
\begin{align*}
  \frac{dn}{dt}
 &= \bar{v} (\bar{\nu}\bar{\Sigma}_f - \bar{\Sigma}_a) n(t) 
  + S(t) \\
 &= \bar{\Sigma}_a\bar{v} \left ( \frac{\bar{\nu}\bar{\Sigma}_f}{\bar{\Sigma}_a} - 1 \right ) n(t) 
  + S(t) \\
 \tag{FNRP 5.12}
 \Aboxed{\frac{dn}{dt} &= \left ( \frac{k_{\infty} - 1}{l_{\infty}} \right ) n(t) 
  + S(t)} \, . 
\end{align*}
 
\end{frame}
%%%%%%%%%%%%%%%%%%%%%%%%%%%%%%%%%%%%%%%%%%%%%%%%%%%%%%%%%%%%%%%%%%%%%%%%%%%%%%

%%%%%%%%%%%%%%%%%%%%%%%%%%%%%%%%%%%%%%%%%%%%%%%%%%%%%%%%%%%%%%%%%%%%%%%%%%%%%%
\begin{frame}[fragile]{Kinetics with Multiplication}

For finite systems with leakage, the text explains how to incorporate the non-leakage probability $P_{NL}$, and the resulting FNRP (5.22) is identical FNRP (5.12) without the $\infty$'s on $k$ and $l$:
\begin{equation*}
\tag{FNRP 5.22}
\frac{dn}{dt}  = \left ( \frac{k - 1}{l} \right ) n(t) + S(t)
\end{equation*}
   
\vfill 
\pause 

{\bf Example}: Let $S(t) = 0$ and $n(0) = n_0$.  Determine $n(t)$.

\pause 
\textcolor{wcprimary}{
{\small {\bf Ans}.  The equation is separable, i.e., $dn/n = [(k-1)/l]dt$, and direct integration yields
\begin{equation*}
  \tag{FNRP 5.24}
  \boxed{ n(t)  =  n_0 e^{\left ( \displaystyle \frac{k - 1}{l}  \right ) t} } \, ,
\end{equation*}
or, recalling, $\rho = (k-1)/k$,
\begin{equation*}
  \tag{FNRP 5.24}
  \boxed{ n(t)  =  n_0 e^{ (\rho/\Lambda) t} } \, ,
\end{equation*}
where $\Lambda = l/k$ is the {\bf prompt generation time}.
}}

\pause 
\vfill 


 
\end{frame}
%%%%%%%%%%%%



%%%%%%%%%%%%%%%%%%%%%%%%%%%%%%%%%%%%%%%%%%%%%%%%%%%%%%%%%%%%%%%%%%%%%%%%%%%%%%
\begin{frame}[fragile]{Kinetics of Source-Driven, Subcritical Systems}


When $S(t) \neq 0$, we go from

\begin{equation*}
   \frac{d}{dt} \left ( n(t) e^{-\left(\frac{k-1}{l}\right )t} \right ) = S(t) e^{-\left(\frac{k-1}{l}\right )t} .
\end{equation*}

to

\begin{equation*}
   n(t') =  \frac{\displaystyle\int^{t'}_0  \left( S(t) e^{-\left(\frac{k-1}{l}\right )t} \right ) dt + n(0)}
                 {e^{-\left(\frac{k-1}{l}\right )t'}}
\end{equation*}

If $S(t) = S_0$, then

\begin{equation*}
   n(t) = \frac{l S_0}{k-1}
     \left (  e^{\frac{k-1}{l}t} - 1 \right )
     + n(0) e^{\frac{k-1}{l}t} \, .
\end{equation*}

If we further set $n(0) = 0$, we get FNRP (5.26).

\end{frame}


%%%%%%%%%%%%%%%%%%%%%%%%%%%%%%%%%%%%%%%%%%%%%%%%%%%%%%%%%%%%%%%%%%%%%%%%%%%%%%
\begin{frame}[fragile]{Revisiting Subcritical Multiplication}

Let $n(0) = 0$, $S_0 = 1$, and $l_{\infty} = 10^{-2}$.  What happens to $n(t)$ for $0 \leq t \leq 10$ s as $k_{\infty}$ increases from 0 to 1?\footnote{Recall that, for an infinite, homogeneous medium, $\phi = S_0/(\bar{\Sigma}_a-\bar{\nu}\bar{\Sigma}_f) = (S_0/\bar{\Sigma}_a)/(1-k_{\infty})$!}

\resizebox{0.95\textwidth}{!}{\input{figures/subcrit_mult.pgf}}

\end{frame}


\begin{frame}[fragile]{Appendix: Solve $dn/dt = [(k-1)/l]n(t)$}

{\bf Problem.} Let $S(t) = 0$ and $n(0) = n_0$. Determine $n(t)$.

\vfill

\begin{columns}[T]
  \begin{column}{0.48\textwidth}
  {\small
  {\bf 1. Write Down the Equations}  
  \begin{equation*}
   \frac{dn}{dt} =
    \left(\frac{k - 1}{l}\right)n(t),
    \quad n(0) = n_0 \, .
  \end{equation*}
  
  \vspace{0.25cm}
  
  {\bf 2. Identify Type and Familiar Form}  \\
  \vspace{0.25cm}

  This first-order, initial-value problem is {\bf separable}, and we can write
  \begin{equation*}
     \frac{dn}{n(t)} =
     \left(\frac{k - 1}{l}\right) dt \, .
  \end{equation*}
  }
  \end{column}
  %
  \begin{column}{0.48\textwidth}
  {\small
  {\bf 3. Integrate Using Appropriate Technique}
  \begin{align*}
    \int^{n(t)}_{n(0)} \frac{dn}{n(t')} &=
     \left(\frac{k - 1}{l}\right)
     \int^{t}_{0} dt' \\[4pt]
    \ln\!\left(\frac{n(t)}{n(0)}\right)
      &= \left(\frac{k - 1}{l}\right)t \\[4pt]
    \Rightarrow \quad n(t)
      &= n(0)\, e^{\left(\displaystyle\frac{k - 1}{l}\right)t}
  \end{align*}
  {\bf 4. Apply Initial Conditions}
  \begin{equation*}
    \boxed{ n(t)  =  n_0 e^{\left ( \displaystyle\frac{k - 1}{l}  \right ) t} }
  \end{equation*}
  }
  \end{column}
\end{columns}

\end{frame}

\end{document}

