\documentclass[aspectratio=1610]{beamer}
\usepackage[T1]{fontenc}
\usetheme{wildcat}
\usetikzlibrary{arrows.meta,angles,quotes,calc,intersections,positioning}

\usepackage{amsmath,amssymb,amsfonts}
\usepackage{booktabs}
\usepackage{relsize}
\usepackage{pgfplots}
\pgfplotsset{compat=1.16}
\usepackage{array}
\usepackage{siunitx}
\usepackage{cancel}
\usepackage{jupyter}
\usepackage{minted}
\usepackage{xfrac}
\usepackage{mathtools}

\let\oldfootnotesize\footnotesize
\renewcommand*{\footnotesize}{\oldfootnotesize\tiny}

\def\mathdefault#1{#1}
\everymath=\expandafter{\the\everymath\displaystyle}


\title{Step Insertions of Reactivity \\
       {\small\it NE 630 - Lecture 28}}

\date{\input{term.txt} \\ {\footnotesize Git SHA: \input{git_sha.txt}}}

\author{Jeremy Roberts}


\definecolor{ksupurple}{HTML}{512888}
\definecolor{orange}{HTML}{CA7C1B}
\definecolor{skyblue}{RGB}{180,220,255}

\begin{document}

\begin{frame}
\titlepage
\end{frame}
 
 
%%%%%%%%%%%%%%%%%%%%%%%%%%%%%%%%%%%%%%%%%%%%%%%%%%%
\begin{frame}{Primary Objective}

Students will be able to 

\vfill

\begin{quote}
\textcolor{wcprimary}{solve the kinetics equations subject to step insertions of reactivity and to predict neutron and precursor densities and reactor periods from those solutions and approximations thereof. 
}
\end{quote}

\vfill 

\end{frame}

%%%%%%%%%%%%%%%%%%%%%%%%%%%%%%%%%%%%%%%%%%%%%%%%%%%%%%%%%%%%%%%%%%%%%%%%%%%%%%
\begin{frame}[fragile]{Review: The Point Kinetics Equations}

The time evolution of the neutron population (or density!) is described by
\begin{equation*}
  \tag{FNRP 5.47}
  \frac{dn}{dt} = 
     \left ( \frac{\rho(t) - \beta}{\Lambda} \right ) n(t) + 
     \sum^N_{i=1} \lambda_i C_i (t) + S(t) \, ,
\end{equation*}
while the precursor populations (or densities!) evolve according to
\begin{align*}
  \tag{FNRP 5.48}
  \frac{d C_i}{dt} = -\lambda_i C_i(t) + \frac{\beta_i}{\Lambda} n(t) \, \quad i = 1, 2, \ldots, N \, ,
\end{align*}
where $\beta = \sum_i \beta_i$ is the total delayed neutron precursor fraction.  
These delayed neutrons increase the average neutron lifetime\footnote{Without delayed neutrons, the growth of a neutron population follows $e^{[(k-1)/l]t}$, so increasing $l$ even a little slows that growth a lot!} according to
\begin{align*}
\tag{FNRP 5.36}
 \bar{l} &= (1-\beta)l + \beta l_d \\
         &= l + \beta/\lambda \, .
\end{align*}

\end{frame}

 

%%%%%%%%%%%%%%%%%%%%%%%%%%%%%%%%%%%%%%%%%%%%%%%%%%%%%%%%%%%%%%%%%%%%%%%%%%%%%%
\begin{frame}[fragile]{Review: A Single Precursor Group}

One can reduce the $N$ precursor equations to one by defining
\begin{equation*}
\tag{FNRP 5.31 \& 5.34}
 \beta = \sum^N_{i=1} \beta_i \qquad \text{and} \qquad 
\frac{1}{\lambda} = \left(\frac{1}{\beta} \sum_{i=1}^N \frac{\beta_i}{\lambda_i}\right) \, .
\end{equation*}

Then the PKEs reduce to
\begin{align*}
 \frac{dn}{dt} = \frac{\rho(t)-\beta}{\Lambda}n(t) + \lambda C(t) \qquad \text{and} \qquad
 \frac{dC}{dt} = -\lambda C(t) + \frac{\beta}{\Lambda} n(t) \, .
\end{align*}


\vfill 
\pause

{\bf Example}: Assume a ${}^{235}$U-fueled reactor is initially critical, i.e., $\rho(t<0)=0$, with $n(0) = 10^9$.  
With $\Lambda = 10^{-4}$ and $\rho(t\geq 0)=\rho_0=0.25\$$, write down the equations, with numerical
coefficients, needed to find $n(t)$ and $C(t)$.

\pause 
\textcolor{wcprimary}
{{\small
{\bf Ans}.  Using data from Table 5.1, $\beta = 0.00650$ and $\lambda = 0.07664$.  Then, FNRP (5.29) gives $C(0) = \beta n(0) / (\lambda \Lambda) \approx 8.48 \cdot 10^{11}$.  Finally, $\rho_0 = 0.25~(\$) = 0.25\beta~(\Delta k/k)=  162.5~(\text{pcm})$.
}}
\end{frame}

%%%%%%%%%%%%%%%%%%%%%%%%%%%%%%%%%%%%%%%%%%%%%%%%%%%%%%%%%%%%%%%%%%%%%%%%%%%%%%

%%%%%%%%%%%%%%%%%%%%%%%%%%%%%%%%%%%%%%%%%%%%%%%%%%%%%%%%%%%%%%%%%%%%%%%%%%%%%%
\begin{frame}[fragile]{Step Insertions and the In-Hour Equation}


\begin{columns}[T]

\begin{column}{0.4\textwidth}

 

Whether we have $N$ precursor groups or just one group, the PKEs can be solved
analytically only for special cases in which $\rho(t)$ and $S(t)$ are simple functions of
time (e.g., a constant or a sinusoid).  

\vspace{0.35cm}

For a constant, {\bf step insertion of reactivity} $\rho(t) = \rho$, the PKEs have {\it constant coefficients}.

\vspace{0.35cm}

If we further take $S(t) = 0$, then the solutions (i.e., $n(t)$, $C_i(t)$) must be sums of exponential functions 
with different time constants.
\end{column}

\begin{column}{0.58\textwidth}
 
In other words,
\begin{align*}
   n(t)   &= \sum_{j=1}^{N+1} A_j e^{\omega_j t} \\
   C_i(t) &= \sum_{j=1}^{N+1} B_{ij} e^{\omega_j t} \, ,  \,i = 1, \ldots, N \, ,
\end{align*}

\vspace{-0.25cm}

whose time constants $\omega_j$ satisfy

\vspace{-0.25cm}

\begin{equation*}
\tag{FNRP 5.53}
   \boxed{\rho = \left ( \Lambda + \sum_{i=1}^N \frac{\beta_i}{\omega + \lambda_i} \right ) \omega} \, .
\end{equation*}
This result is the {\bf inhour equation} and represents a polynomial of degree $N+1$ whose 
roots are the $\omega_j$.
 
\end{column}

\end{columns}

\end{frame}
%%%%%%%%%%%%


%%%%%%%%%%%%%%%%%%%%%%%%%%%%%%%%%%%%%%%%%%%%%%%%%%%%%%%%%%%%%%%%%%%%%%%%%%%%%%
\begin{frame}[fragile]{The Reactor Period}
 
\begin{columns}[T]

\begin{column}{0.6\textwidth}

\resizebox{0.99\textwidth}{!}{\input{figures/traces.pgf}}

\end{column}

\begin{column}{0.4\textwidth}

Shown at the left are {\it traces} for several step insertions of reactivity.  
In all cases, after the immediate {\it jump} or {\it drop}, the solution follows
\begin{equation*}
   n(t) \approx A_1 e^{\omega_1 t} = A_1 e^{t/T} \, ,
\end{equation*}
where $T$ is the {\bf (stable) reactor period}.
 
\vspace{0.25cm}

{\bf Example}: Estimate the period {\it graphically} for $\rho = 0.2\beta$, $\rho = \rho_A$, and $\rho = \rho_B$.
Then, use the inhour equation to estimate $\rho_A$ and $\rho_B$.

\pause 

\textcolor{wcprimary}{
{\bf Ans.}
}

\end{column}

\end{columns}

\end{frame}

%%%%%%%%%%%%%%%%%%%%%%%%%%%%%%%%%%%%%%%%%%%%%%%%%%%%%%%%%%%%%%%%%%%%%%%%%%%%%%
\begin{frame}[fragile]{Step Insertions with One Precursor Group}
 
{\small

Write the PKEs as
\begin{equation*}
 \frac{d}{dt}  
   \overbrace{\left [ 
     \begin{array}{c} n(t) \\ 
                      C(t) 
      \end{array} \right ]}^{\mathbf{y}(t)}
 =
   \overbrace{\left [ \begin{array}{cc}
       (\rho-\beta)/\Lambda    & \lambda \\
       \beta /\Lambda          & -\lambda \\
   \end{array} \right ]}^{\mathbf{M}} 
   \left [ \begin{array}{c} n(t) \\ C(t) \end{array} \right ] \, , \quad \left [ 
     \begin{array}{c} n(0) \\ 
                      C(0) 
      \end{array} \right ] = \left [ 
     \begin{array}{c} n_0 \\ 
                      C_0
      \end{array} \right ] \, .
\end{equation*}

The eigenvalues $\omega$ of $\mathbf{M}$ satisfy
\begin{equation*}
 \overbrace{1}^{a} \omega^2 + 
           \overbrace{\left (\lambda - \frac{\rho - \beta}{\Lambda} \right )}^{b} \omega + 
            \overbrace{\left ( -\frac{\lambda \beta}{\Lambda} -\frac{\lambda(\rho - \beta)}{\Lambda} \right )}^{c} = 0  \longrightarrow \boxed{\omega_{1/2} = \frac{-b\pm \sqrt{b^2 - 4ac}}{2a}} \, .
\end{equation*}

The solutions are then
\begin{equation*}
  n(t) = A_1 e^{\omega_1 t} + A_2 e^{\omega_2 t} \qquad \text{and} \qquad 
  C(t) = B_1 e^{\omega_1 t} + B_2 e^{\omega_2 t} \, .
\end{equation*}

We need to find the four unknowns $A_1$, $A_2$, $B_1$, and $B_2$ noting that
\begin{itemize}
 \item the initial conditions give two equations.
 \item the PKEs give two more equations, and setting $t = 0$ eliminates the exponentials!
\end{itemize}

}
 
\end{frame}

%%%%%%%%%%%%%%%%%%%%%%%%%%%%%%%%%%%%%%%%%%%%%%%%%%%%%%%%%%%%%%%%%%%%%%%%%%%%%%
\begin{frame}[fragile]{Step Insertions with One Precursor Group}
 

\begin{columns}[T]

\begin{column}{0.6\textwidth}

\begin{minted}[]{python}
from math import sqrt
def step_insert(ρ, β, Λ, λ, n0):
  c0 = β*n0/(λ*Λ)
  a = 1
  b = λ*(1-(ρ-β)/(λ*Λ))
  c = -1*(λ*(ρ-β)/Λ + λ*β/Λ)
  d = sqrt(b**2-4*a*c)
  ω1 = (-b+d)/(2*a)
  ω2 = (-b-d)/(2*a)
  A1 = (ρ*n0/Λ - n0*ω2)/(ω1-ω2)
  A2 = n0 - A1
  B1 = (-β*n0*ω2/(λ*Λ))/(ω1-ω2)
  B2 = β*n0/(λ*Λ) - B1
  return ω1, ω2, A1, A2, B1, B2
\end{minted} 

\end{column}


\begin{column}{0.4\textwidth}

$\rho=-0.5\beta$, $\Lambda = 5\cdot 10^{-5}$

\resizebox{0.99\textwidth}{!}{\input{figures/prompt_drop_illustration.pgf}}

\end{column}

\end{columns}

 
\end{frame}

%%%%%%%%%%%%%%%%%%%%%%%%%%%%%%%%%%%%%%%%%%%%%%%%%%%%%%%%%%%%%%%%%%%%%%%%%%%%%%
\begin{frame}[fragile]{Prompt Jump/Drop Approximation}

As the traces illustrate, after a small insertion $\Delta \rho \ll \beta$, the population swiftly jumps or drops.  On the linear scale, the population appears to jump immediately from $n_0$ to some new value $n_1$.  The ratio of this new value to the initial value satisfies 
\begin{equation*}
\boxed{\frac{n_1}{n_0} = \frac{\beta - \rho_0}{\beta - \rho_1}} \, ,
  \tag{Prompt Jump Approximation}
\end{equation*}
where $\rho_0$ is the initial reactivity (usually zero, but $|\rho_0| \ll \beta$ always) and the new reactivity $\rho_1 = \rho_0 + \Delta \rho$.

\vfill 

{\bf Example}. Use the prompt jump approximation to verify
the $\pm 0.2\beta$ traces.  What is $\rho \pm 0.2\beta$ in \$? pcm?

\end{frame}

%%%%%%%%%%%%%%%%%%%%%%%%%%%%%%%%%%%%%%%%%%%%%%%%%%%%%%%%%%%%%%%%%%%%%%%%%%%%%%
\begin{frame}[fragile]{Periods for Very Small and Very Large Insertions}

A ``small'' reactivity is any $|\rho| \ll \beta$.  In this case,                        

\begin{equation*}
\tag{FNRP 5.57}
 T \approx \frac{\beta}{\rho \lambda}\, .
\end{equation*}

For reactivities $\rho > \beta$, the neutron population grows exponentially with prompt neutrons alone and, hence, the period is very small and well approximated by

\begin{equation*}
\tag{FNRP 5.58}
 T \approx \frac{\Lambda}{\rho - \beta} \, .
\end{equation*}

Finally, for large {\it negative} insertions, $T \approx 1/\lambda$, or, for multiple precursor groups, $T \approx 1/\lambda_1$, where $\lambda_1$ corresponds to the longest lives group.

\end{frame}

\end{document}

