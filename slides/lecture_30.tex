\documentclass[aspectratio=1610]{beamer}
\usepackage[T1]{fontenc}
\usetheme{wildcat}
\usetikzlibrary{arrows.meta,angles,quotes,calc,intersections,positioning}

\usepackage{amsmath,amssymb,amsfonts}
\usepackage{booktabs}
\usepackage{relsize}
\usepackage{pgfplots}
\pgfplotsset{compat=1.16}
\usepackage{array}
\usepackage{siunitx}
\usepackage{cancel}
\usepackage{jupyter}
\usepackage{minted}
\usepackage{xfrac}
\usepackage{mathtools}

\let\oldfootnotesize\footnotesize
\renewcommand*{\footnotesize}{\oldfootnotesize\tiny}

\def\mathdefault#1{#1}
\everymath=\expandafter{\the\everymath\displaystyle}


\title{Review of Some Math \\
       {\small\it NE 630 - Lecture 30}}

\date{\input{term.txt} \\ {\footnotesize Git SHA: \input{git_sha.txt}}}

\author{Jeremy Roberts}


\definecolor{ksupurple}{HTML}{512888}
\definecolor{orange}{HTML}{CA7C1B}
\definecolor{skyblue}{RGB}{180,220,255}

\begin{document}

\begin{frame}
\titlepage
\end{frame}
 
 
%%%%%%%%%%%%%%%%%%%%%%%%%%%%%%%%%%%%%%%%%%%%%%%%%%%
\begin{frame}{Primary Objective}

Students will be able to 

\vfill

\begin{quote}
\textcolor{wcprimary}{define, recognize, and solve a variety of two-point boundary-value problems.
}
\end{quote}

\vfill 

Note: these slides represent a condensed summary of the in-class work we did to 
review this fundamental topic.

\end{frame}


%%%%%%%%%%%%%%%%%%%%%%%%%%%%%%%%%%%%%%%%%%%%%%%%%%%%%%%%%%%%%%%%%%%%%%%%%%%%%%
\begin{frame}[fragile]{The Quiz}
 
\noindent {\bf Problem 1}.  Consider the following equation:
\begin{equation*}
  \frac{d^2 y}{dx^2} + p(x)y(x) = q(x) \, .
\end{equation*}
List every feature of this equation that you can recall from a course like MATH 340.

\vfill 


\noindent{\bf Problem 2}. Suppose we set $p(x) = p$ (a constant) and $q(x) = 0$ above.  Dig deep and determine the {\it general} solution $y(x)$.  {\it Hint}: it should involve some arbitrary constants of integration---how many?

 
\vfill 

\noindent{\bf Problem 3}. Now, suppose $p(x) = p$ (a constant) and $q(x) = e^x$ above. How would you find $y(x)$ in this case?  Go ahead and do it if you can!



\end{frame}

%%%%%%%%%%%%%%%%%%%%%%%%%%%%%%%%%%%%%%%%%%%%%%%%%%%%%%%%%%%%%%%%%%%%%%%%%%%%%%
\begin{frame}[fragile]{The Review}
 
$y'' + p(x)y(x) = q(x)$ is a linear, second-order, ordinary differential equation with variable coefficients, and, assuming $q(x) \neq 0$, it is inhomogeneous. 

\vfill 

If we set $p(x) = p$ (a constant) and $q(x) = 0$, the equation becomes homogeneous with constant coefficients.  Constant coefficients should (almost always) suggest a solution of the form $e^{rx}$.  Drop this into $y'' + py(x) = 0$, find that $r^2 + p = 0$, and reason that $y(x)$ is some combination of $e^{i\sqrt{p}x}$ and $e^{-i\sqrt{p}x}$, i.e., $y(x) = y_H(x) = C_1 e^{i\sqrt{p}x} + C_2 ^{-i\sqrt{p}x}$, where $_H$ indicates a solution to a homogeneous problem.  We can do nothing more without further information!

 
\vfill 

If we keep constant $p$ but set $q(x) = e^x$, we first seek a $y_P(x)$ that looks like $q(x)$, e.g., $y_P(x) = C_3 e^{x}$.  Drop that into $y'' + py(x) = e^x$ to find $C_3 e^{x} + p e^x = e^x$, which means that $C_3 + p = 1$ or $C_3 = 1 -p$.  Then the full (or what I'd call {\it general}) solution is $y(x) = y_H(x) + y_P(x)$ or $y(x) =  C_1 e^{i\sqrt{p}x} + C_2 ^{-i\sqrt{p}x} + (1-p) e^x$.  Again, we can do nothing more without further information!


\end{frame}

%%%%%%%%%%%%%%%%%%%%%%%%%%%%%%%%%%%%%%%%%%%%%%%%%%%%%%%%%%%%%%%%%%%%%%%%%%%%%%
\begin{frame}[fragile]{The (Maybe) New}
 
The further information comes from {\it conditions} at two {\it boundaries} (i.e., {\bf boundary conditions} or BCs).  Maybe $y(x)$ is a temperature or neutron flux and we know it at $x = 0$ and $x = L$, e.g., $y(0) = 0$ and $y(L) = 1$.  If $q(x) = 0$, our solution must then satisfy $y(0) = C_1 + C_2 = 0$ and $y(L) =  C_1 e^{i\sqrt{p}L} + C_2 ^{-i\sqrt{p}L}$, or
\begin{equation*}
 \begin{bmatrix}
  1 & 1 \\
  e^{i\sqrt{p}L} & e^{-i\sqrt{p}L}
 \end{bmatrix}
 \begin{bmatrix}
  C_1 \\
  C_2
 \end{bmatrix} =  \begin{bmatrix}
  0 \\
  1
 \end{bmatrix} \, .
\end{equation*}
Often, solving ODEs ``all the way'' results in algebra!  Because  $y(L) \neq 0$, this problem is not homogeneous (i.e., the ODE and the BCs must have all terms proportional to $y(x)$ or its derivatives; the constant 1 breaks that).  If instead we set $y(L) = 0$, the problem {\it is} homogeneous, and we're left with this condundrum:
$C_1 + C_2 = 0$, $C_1 e^{i\sqrt{p}L} + C_2 ^{-i\sqrt{p}L} = 0$, and, thus
\begin{equation*}
 C_2  e^{i\sqrt{p}L} = C_2  e^{-i\sqrt{p}L} \, !?!
\end{equation*}
Just like $k_{\infty}$ helped us when we couldn't find $\phi_1$, we're going to need to find which values of $\sqrt{p}$  satisfy this {\it eigenvalue} problem.


\end{frame}



\end{document}

