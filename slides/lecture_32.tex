\documentclass[aspectratio=1610]{beamer}
\usepackage[T1]{fontenc}
\usetheme{wildcat}
\usetikzlibrary{arrows.meta,angles,quotes,calc,intersections,positioning}

\usepackage{amsmath,amssymb,amsfonts}
\usepackage{booktabs}
\usepackage{relsize}
\usepackage{pgfplots}
\pgfplotsset{compat=1.16}
\usepackage{array}
\usepackage{siunitx}
\usepackage{cancel}
\usepackage{jupyter}
\usepackage{minted}
\usepackage{xfrac}
\usepackage{mathtools}

\let\oldfootnotesize\footnotesize
\renewcommand*{\footnotesize}{\oldfootnotesize\tiny}

\def\mathdefault#1{#1}
\everymath=\expandafter{\the\everymath\displaystyle}


\title{One-Speed Diffusion in \\ a Homogeneous Slab\\
       {\small\it NE 630 - Lecture 32}}

\date{\input{term.txt} \\ {\footnotesize Git SHA: \input{git_sha.txt}}}

\author{Jeremy Roberts}


\definecolor{ksupurple}{HTML}{512888}
\definecolor{orange}{HTML}{CA7C1B}
\definecolor{skyblue}{RGB}{180,220,255}

% --- Shortcuts ---
\newcommand{\br}{\mathbf{r}}
\newcommand{\bO}{\hat{\Omega}}
\newcommand{\bn}{\hat{\mathbf{n}}}
\newcommand{\jj}{\mathbf{j}}
\newcommand{\angflux}{\psi}
\newcommand{\dV}{\,d^3\br}
\newcommand{\dE}{\,dE}
\newcommand{\dO}{\,d\hat{\Omega}}
\newcommand{\ds}{\,d\mathbf{S}}
\newcommand{\mnote}[1]{\marginpar{\raggedright\footnotesize\color{ksupurple}{#1}}}

% --- Common shortcuts ---
\newcommand{\flux}{\phi}
\newcommand{\JJ}{\mathbf{J}}
\newcommand{\dd}{\,\mathrm{d}}
\newcommand{\SigT}{\Sigma_t}
\newcommand{\SigA}{\Sigma_a}
\newcommand{\SigS}{\Sigma_s}
\newcommand{\SigF}{\Sigma_f}
\newcommand{\nuSigF}{\nu\Sigma_f}
\newcommand{\D}{D}
\newcommand{\Ldiff}{L}
\newcommand{\xhat}{\hat{\mathbf{x}}}
\newcommand{\rhat}{\hat{\mathbf{r}}}

\begin{document}

\begin{frame}
\titlepage
\end{frame}
 
 
%%%%%%%%%%%%%%%%%%%%%%%%%%%%%%%%%%%%%%%%%%%%%%%%%%%
\begin{frame}{Primary Objective}

Students will be able to 

\vfill

\begin{quote}
\textcolor{wcprimary}{develop solutions for the one-speed diffusion equation in 
slab geometry for single-region systems with constant $\D$, $\SigA$, and a uniform volumetric source $S(x)$
}
\end{quote}


\end{frame}


%%%%%%%%%%%%%%%%%%%%%%%%%%%%%%%%%%%%%%%%%%%%%%%%%%%%%%%%%%%%%%%%%%%%%%%%%%%%%%
\begin{frame}[fragile]{Review: from Continuity to Diffusion}
 
Last time, we developed the neutron transport equation and integrated it 
over all angles to produce the neutron continuity equation
\begin{align*}
  \frac{1}{v}\,\frac{\partial \flux}{\partial t}
  &+ \nabla\!\cdot\! \JJ(\br,E,t)
   +\Sigma_t(\br,E)\, \flux(\br,E,t) = \overbrace{\textcolor{wcprimary}{Q(\mathbf{r}, E, t)}}^{\text{in-scatter + source}} \, ,
  \tag{4--79}
\end{align*}
where $\phi$ is the (scalar) flux, $\mathbf{J}$ is the current density, and 
the {\it emission density} $Q$ represents all \textcolor{wcprimary}{gains}.
Closure of the continuity equation  comes from Fick's law, 
\begin{equation*}
\tag{FNRP 6.11}
   \mathbf{J} = -D \nabla \phi \, ,
\end{equation*}
substitution of which into (4--79) produces the {\bf neutron diffusion equation}
\begin{align*}
  \frac{1}{v}\,\frac{\partial \flux}{\partial t}
  - \nabla\!\cdot\! D(\br,E) \nabla \phi(\br,E,t)
   +\Sigma_t(\br,E)\, \flux(\br,E,t)
  = Q(\mathbf{r}, E, t)\, .
\end{align*}
 
\end{frame}

%%%%%%%%%%%%%%%%%%%%%%%%%%%%%%%%%%%%%%%%%%%%%%%%%%%%%%%%%%%%%%%%%%%%%%%%%%%%%%
\begin{frame}[fragile]{Review: Steady-State, One-Speed Diffusion}


Now that we have space, it's time to say goodbye to $t$ and to 
replace $E$ by $g$:
\begin{align*}
 - \nabla\!\cdot\! D_g(\br) \nabla \phi_g(\br)
   + \Sigma_{tg}(\br)\, \flux_g(\br)
  = \sum_{g'=1}^{G} \Sigma_{sg\gets g'} \phi_{g'}(\br) +  S_g(\br) \, .
\end{align*}

{\bf Example}: For $G=2$ and no upscatter, show that
\begin{align*}
 - \nabla\!\cdot\! D_1(\br) \nabla \phi_1(\br)
   + \Sigma_{r,1}(\br)\, \flux_1(\br)
  &=  S_1(\br,t) \\ 
 - \nabla\!\cdot\! D_2(\br) \nabla \phi_2(\br)
   + \Sigma_{a,2}(\br)\, \flux_2(\br)
  &=  \Sigma_{s,2\gets 1} \phi_{1}(\br) + S_2(\br) 
\end{align*}
\vfill 

What if $G=1$ and we introduce multiplication?
\pause 
\begin{align*}
\tag{FNRP 6.12}
\boxed{ - \nabla\!\cdot\! D(\br) \nabla \phi(\br)
   + \Sigma_{a}(\br)\, \flux(\br) 
  =  S(\br) + \bar{\nu}\Sigma_{f}(\br)\, \flux(\br) 
}
\end{align*}


\end{frame}


\begin{frame}[fragile]{Applying $\nabla$ and $\nabla^2$}

Much like integration over $4\pi$ hides some detail, so, too, does $\nabla\!\cdot\! D(\br) \nabla \phi(\br)$!

\vfill 

For Cartesian (or $xyz$) coordinates, the gradient and divergence operators act on scalar and vector functions according to
\begin{align*}
\nabla \phi(\mathbf{r})
= \mathbf{i} \,\frac{\partial \phi}{\partial x}
 + \mathbf{j}  \,\frac{\partial \phi}{\partial y}
 + \mathbf{k}  \,\frac{\partial \phi}{\partial z} \quad \text{and} \quad
\nabla\!\cdot\!\mathbf{J}(\mathbf{r})
= \frac{\partial J_x}{\partial x}
 + \frac{\partial J_y}{\partial y}
 + \frac{\partial J_z}{\partial z} \, ,
\end{align*}
so that
\begin{align*}
\nabla\!\cdot\!\big(D(\mathbf{r})\,\nabla \phi(\mathbf{r})\big)
&= \frac{\partial}{\partial x}\!\left(D\,\frac{\partial \phi}{\partial x}\right)
 + \frac{\partial}{\partial y}\!\left(D\,\frac{\partial \phi}{\partial y}\right)
 + \frac{\partial}{\partial z}\!\left(D\,\frac{\partial \phi}{\partial z}\right) \\[4pt]
&= D(\mathbf{r})\,\nabla^2 \phi(\mathbf{r})
  + \big(\nabla D(\mathbf{r})\big)\!\cdot\!\big(\nabla \phi(\mathbf{r})\big).
\end{align*}

\vfill 


\end{frame}



%%%%%%%%%%%%%%%%%%%%%%%%%%%%%%%%%%%%%%%%%%%%%%%%%%%%%%%%%%%%%%%%%%%%%%%%%%%%%%
\begin{frame}[fragile]{The Homogeneous Slab}
 
For homogeneous regions (i.e., $\Sigma_a(\mathbf{r}) = \Sigma_a$ and $D(\mathbf{r}) = D$),
the one-speed diffusion equation can be written as 
\begin{align*}
  -D \left (\frac{\partial^2 \phi}{\partial x^2} +
            \frac{\partial^2 \phi}{\partial y^2} + 
            \frac{\partial^2 \phi}{\partial z^2} \right )
  + \Sigma_{a}\flux(\br) 
  = \bar{\nu}\Sigma_{f} \flux(\br)  +  S(\br) \, .
\end{align*}

\vfill 

Now, suppose this region is very (infinitely) large in the $y$ and $z$ directions but finite along the $x$.  If $S(\mathbf{r}) = S(x)$, then everything in this {\it homogeneous slab} is uniform along the $y$ and $z$ directions, and, hence, we should have $\phi(\mathbf{r}) = \phi(x)$.  Thus, the one-speed diffusion equation in slab geometry is 
\begin{align*}
  -D \frac{d^2 \phi}{d x^2}
  + \Sigma_{a}\flux(x) 
  = \bar{\nu}\Sigma_{f} \flux(x)  +  S(x) \, .
\end{align*}
\vfill 

This is a second-order, inhomogeneous, ordinary differential equation with constant coefficients.

\end{frame} 

%%%%%%%%%%%%%%%%%%%%%%%%%%%%%%%%%%%%%%%%%%%%%%%%%%%%%%%%%%%%%%%%%%%%%%%%%%%%%%
\begin{frame}[fragile]{The Diffusion Coefficient $D$}

The {\bf diffusion coefficient} is defined\footnote{through the formal derivation of the diffusion equation from the transport equation} as
\begin{equation*}
 \D \;=\; \frac{1}{3\,\Sigma_{\mathrm{tr}}} \, ,
\end{equation*}
where $\Sigma_{\mathrm{tr}} \equiv \SigT - \bar{\mu}\,\SigS$ is the {\bf transport cross section}, and 
$\bar{\mu}$ is the average cosine of the scattering angle in the {\it laboratory} system. 

\vfill 

For isotropic elastic scattering in the 
{\it center-of-mass system} and a single 
nuclide of mass number $A$,  
\begin{equation*}
\tag{FNRP 2.X}
 \bar{\mu} = \frac{2}{3A} \, .
\end{equation*}
When neither $\Sigma_{\mathrm{tr}}$ nor $\bar{\mu}$ is specified, we set $\Sigma_{\mathrm{tr}} \approx \Sigma_{t}$.

\vfill 

{\bf Ponderable}: What does $\Sigma_{\mathrm{tr}} \approx \Sigma_{t}$ imply about scattering?
 

\end{frame}

%%%%%%%%%%%%%%%%%%%%%%%%%%%%%%%%%%%%%%%%%%%%%%%%%%%%%%%%%%%%%%%%%%%%%%%%%%%%%%
\begin{frame}[fragile]{The Diffusion Length $L$ and A First Example}

To simplify things a bit, let's define the {\bf diffusion length}
\begin{equation*}
 L = \sqrt{\frac{D}{\Sigma_a}} \, , 
\end{equation*}
so that the diffusion equation sans multiplication becomes
\begin{align*}
   \frac{d^2 \phi}{d x^2}
  - \frac{1}{L^2}\flux(x) 
  = -\frac{1}{D} S(x) \, .
\end{align*}

{\bf Example}. Consider a slab of thickness $a$ with known $\SigA$ and $\D$. If there is a uniform source $S_0$ throughout, determine $\flux(x)$ in $0\le x\le a$ subject to
the boundary conditions\footnote{We'll see other boundary conditions next time, but the zero-flux conditions are an approximate way to model a slab surrounded by ``nothing'' (i.e., a vacuum}
\begin{equation*}
 \flux(0) = 0  \qquad \text{and} \qquad \flux(a) = 0 \, .
\end{equation*}

\end{frame}


\begin{frame}[fragile]{A First Example: Solution}


\textbf{Step 1 (Write the Equations)}. We have (with $L=\sqrt{D/\Sigma_a}$)
\begin{equation*}
\tag{ODE and BCs}
 \frac{d^2\flux}{dx^2} - \frac{1}{\Ldiff^2}\,\flux = -\frac{S_0}{\D}  \quad \text{subject to} \quad  \flux(0) = 0  \,\, \text{and} \,\, \flux(a) = 0 \, .
\end{equation*}
\textbf{Step 2 (Homogeneous Solution)}. Try $\flux_h(x)=e^{\lambda x}$ to obtain $\lambda=\pm \Ldiff^{-1}$, and, hence,
\begin{equation*}
 \flux_h(x)=C_1 e^{x/\Ldiff}+C_2 e^{-x/\Ldiff}.
\end{equation*}
\textbf{Step 3 (Particular Solution)}. Since $S_0$ is constant, take $\flux_p(x)=C_3$; substitution into ODE gives $C_3=L^2 S_0 /D = S_0/\SigA$.\\[4pt]
\textbf{Step 4 (Apply BCs)}. Plugging the general solution $\phi_h+\phi_p$ into the BCs leads to
\begin{align*}
\tag{left BC}
 \flux(0) &=\frac{S_0}{\SigA}+C_1 e^{0/\Ldiff}+C_2 e^{-0/\Ldiff} = 0\\
\tag{right BC}
 \flux(a) &=\frac{S_0}{\SigA}+C_1 e^{a/\Ldiff}+C_2 e^{-a/\Ldiff} = 0 \, ,
\end{align*}
which represent two equations for the two unknowns, $C_1$ and $C_2$.  {\bf Find them!}

\end{frame}


%%%%%%%%%%%%%%%%%%%%%%%%%%%%%%%%%%%%%%%%%%%%%%%%%%%%%%%%%%%%%%%%%%%%%%%%%%%%%%
\begin{frame}[fragile]{A First Example: Verification and Inspection}

We can write the solution compactly\footnote{We can define any two independent, linear combinations 
of $e^{x/L}$ and $e^{-x/L}$, say $f_{\pm}(x) =  (e^{x/L} \pm e^{-x/L})/2$ 
instead define $\phi_h(x) = \tilde{C}_1 f_+(x) + \tilde{C}_2 f_-(x)$.  For many slab problems, 
we'll find that these functions $f_+(x) = \cosh(x/L)$ and $f_-(x) =\sinh(x/L)$ can produce
simpler solutions.
}
\begin{equation*}
 \tag{***}
 \flux(x)=\frac{S_0}{\SigA}\left[1-\frac{\cosh\!\big(\tfrac{x-a/2}{\Ldiff}\big)}{\cosh\!\big(\tfrac{a}{2\Ldiff}\big)}\right] \, .
 \label{eq:ex1sol}
\end{equation*}
To verify this is correct, we should 
\begin{enumerate}
 \item Plug (***) into the ODE.
 \item Plug (***) into the BCs.
 \item Plot $\flux(x)$ to inspect the shape and boundary values.  
\end{enumerate} 
When we aren't given numerical values for $S_0$, $\Sigma_a$, and $D$, we can simply assume reasonable values. 
%We've computed enough $\Sigma$'s to have a feel for (or, at least, examples of) $\Sigma_a$, $\Sigma_s$, and $\Sigma_t$.  

\vfill 

{\bf Example}: Plot $\phi(x)$ with $\SigA=0.1$ \si{\per\centi\meter}, $\D=1$ \si{\centi\meter}, $S_0=1$ \si{\per\centi\meter\per\second} for (a) $a=10$ \si{\centi\meter} and (b) $a=100$ \si{\centi\meter}.  What's the maximum possible flux for any $a$? 

\end{frame}


\end{document}

