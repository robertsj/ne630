\documentclass[aspectratio=1610]{beamer}
\usepackage[T1]{fontenc}
\usetheme{wildcat}
\usetikzlibrary{arrows.meta,angles,quotes,calc,intersections,positioning}

\usepackage{amsmath,amssymb,amsfonts}
\usepackage{booktabs}
\usepackage{relsize}
\usepackage{pgfplots}
\pgfplotsset{compat=1.16}
\usepackage{array}
\usepackage{siunitx}
\usepackage{cancel}
\usepackage{jupyter}
\usepackage{minted}
\usepackage{xfrac}
\usepackage{mathtools}

\let\oldfootnotesize\footnotesize
\renewcommand*{\footnotesize}{\oldfootnotesize\tiny}

\def\mathdefault#1{#1}
\everymath=\expandafter{\the\everymath\displaystyle}


\title{Criticality in Slabs, Cylinders, and Spherical Elephants \\
       {\small\it NE 630 - Lecture 35}}

\date{\input{term.txt} \\ {\footnotesize Git SHA: \input{git_sha.txt}}}

\author{Jeremy Roberts}


\definecolor{ksupurple}{HTML}{512888}
\definecolor{orange}{HTML}{CA7C1B}
\definecolor{skyblue}{RGB}{180,220,255}

% --- Shortcuts ---
\newcommand{\br}{\mathbf{r}}
\newcommand{\bO}{\hat{\Omega}}
\newcommand{\bn}{\hat{\mathbf{n}}}
\newcommand{\JJ}{\mathbf{J}}
\newcommand{\jj}{\mathbf{j}}
\newcommand{\flux}{\phi}
\newcommand{\angflux}{\psi}
\newcommand{\dV}{\,d^3\br}
\newcommand{\dE}{\,dE}
\newcommand{\dO}{\,d\hat{\Omega}}
\newcommand{\ds}{\,d\mathbf{S}}


\begin{document}

\begin{frame}
\titlepage
\end{frame}
 
 
%%%%%%%%%%%%%%%%%%%%%%%%%%%%%%%%%%%%%%%%%%%%%%%%%%%
\begin{frame}{Primary Objective}

Students will be able to 

\vfill

\begin{quote}
\textcolor{wcprimary}{define and evaluate criticality conditions for slab
and cylindrical systems.
}
\end{quote}


\end{frame}


%%%%%%%%%%%%%%%%%%%%%%%%%%%%%%%%%%%%%%%%%%%%%%%%%%%%%%%%%%%%%%%%%%%%%%%%%%%%%%

\begin{frame}[fragile]{Review: Diffusion with Multiplication}

To solve
\begin{equation*}
  - \frac{d^2 \phi}{dx^2} + \frac{1 - k_{\infty}}{L^2} \phi(x) = \frac{S(x)}{D} \, ,
\end{equation*}
we let
\begin{align*}
 \textcolor{wcprimary}{\alpha^2} =  \frac{1-k_{\infty}}{L^2} \, , \quad \textcolor{wcprimary}{k_{\infty} < 1} 
 \qquad \text{and} \qquad 
 \overbrace{\textcolor{wcalerted}{B^2} =  \frac{k_{\infty}-1}{L^2}}^{\text{material buckling}} \, , \quad  \textcolor{wcalerted}{k_{\infty} > 1} \, ,
\end{align*}
so that 
\begin{align*}
 \phi_h(x) = 
 \begin{cases}
  \textcolor{wcprimary}{C_1 e^{\alpha x} + C_2 e^{-\alpha x}} \quad \text{-or-} & \\
  \textcolor{wcprimary}{  C_1 \cosh(\alpha x) + C_2 \sinh(\alpha x)} & \textcolor{wcprimary}{k_{\infty}} < 1 \\
   \textcolor{wcalerted}{ C_1 \cos(B x) + C_2 \sin(B x)} & \textcolor{wcalerted}{ k_{\infty} > 1} \, .
 \end{cases}
\end{align*}
The 5-step process is otherwise unchanged: find $\phi_p(x)$, 
apply BC's and CC's to determine constants, and verify!

\end{frame}

%%%%%%%%%%%%%%%%%%%%%%%%%%%%%%%%%%%%%%%%%%%%%%%%%%%%%%%%%%%%%%%%%%%%%%%%%%%%%%
\begin{frame}[fragile]{A Critical Breakdown}

We made three important observations about $\phi(x)$ while analyzing a source-driven, 
multiplying slab subject to zero-flux conditions: 
\begin{enumerate}
 \item it increases quickly as $B \to \pi/a$
 \item it blows up at $B = \pi/a$ (due to $\cos(Ba/2)$ vanishing in a denominator)
 \item it becomes {\it negative} as $B$ is increased beyond $\pi/a$.
\end{enumerate}


 \pause
This condition can be written and understood as follows:
\begin{align*}
 B^2  = \frac{k_{\infty}-1}{L^2} &= \left ( \frac{\pi}{a} \right )^2 = B_g^2 = \text{geometric buckling} \\
 \tag{a}
 \longrightarrow 1 = \frac{k_{\infty}}{1 + L^2  \left ( \frac{\pi}{a} \right )^2}
                   &= k_{\infty} P_{\text{NL}} \\
 \tag{b}
                   &= \frac{\bar{\nu}\Sigma_f}{\Sigma_a + D  \left ( \frac{\pi}{a} \right )^2 } \\
 \tag{c}
                   &= k_{\text{eff}}
\end{align*}
All three statements (a), (b), and (c) correspond to {\bf criticality}.  Just as in real critical 
systems, we can only have a steady-state flux if there is no source!
 
\end{frame}



%%%%%%%%%%%%%%%%%%%%%%%%%%%%%%%%%%%%%%%%%%%%%%%%%%%%%%%%%%%%%%%%%%%%%%%%%%%%%%
\begin{frame}[fragile]{In Search of $k_{\text{eff}}$}
 
When $B^2 \neq B_g^2$, we resort to the same trick we used 
for infinite, homogeneous systems: scale $\bar{\nu}\Sigma_f$ by $k_{\text{eff}}$ to obtain
\begin{equation*}
  -D \frac{d^2 \phi}{dx^2} + \Sigma_s \phi(x) = \frac{\bar{\nu}\Sigma_f}{k_{\text{eff}}} \phi(x) \, ,
\end{equation*}
or
\begin{equation*}
  \frac{d^2 \phi}{dx^2} + \overbrace{\frac{k_{\infty}/k_{\text{eff}} -1}{L^2}}^{B^2} \phi(x) = 0 \, .
\end{equation*}
In other words, $k_{\text{eff}}$ adjusts $k_{\infty}$ (or $\bar{\nu}\Sigma_f$) so that $B^2 = B^2_g$.  
Armed with this ``fission knob'', our goals will be to determine
\begin{enumerate}
 \item $k_{\text{eff}}$ given material and geometry properties
 \item the value of some material or geometry property (e.g., enrichment or slab width) that produces a desired $k_{\text{eff}}$ (usually 1!)
\end{enumerate}


\end{frame} 

%%%%%%%%%%%%%%%%%%%%%%%%%%%%%%%%%%%%%%%%%%%%%%%%%%%%%%%%%%%%%%%%%%%%%%%%%%%%%%
\begin{frame}[fragile]{Interlude: Some Critical Thinking}

{\bf Example 1}. Consider a homogeneous, multiplying slab of width $a$ subject to zero-flux
boundary conditions.  If $D=1$, $\Sigma_a = 0.1$, and $k_{\infty} = 1.5$,
what value of $a$ makes the reactor critical?

\pause 
{\small \textcolor{wcprimary}{ {\bf Ans.}  $1 = k_{\infty}/(1+L^2 (\pi/a)^2)$ or
$a = \sqrt{ (\pi^2 L^2)(k_{\infty}-1)} \approx 14.05$ cm.
}}

\vfill 
\pause

{\bf Example 2}.  If we replace $\phi(\pm a/2) = 0$ with $J^{\pm}(\mp a/2) = 0$, how
would that value of $a$ change?

\pause 
{\small \textcolor{wcprimary}{ {\bf Ans.}  Zero-flux conditions impose negative incident currents, so $a$ should be {\it smaller} than above since we remove the negative contributions.  Prove it!
}}

\vfill 
\pause

{\bf Example 3}.  If we modify the material in Example 1 so that $k_{\infty} = 0.99$, how does 
the critical value of $a$ change?

\pause 
{\small \textcolor{wcprimary}{ {\bf Ans.}  If the slab ain't critical {\it without} leakage, it ain't critical {\it with} leakage!
}}

\end{frame}

%%%%%%%%%%%%%%%%%%%%%%%%%%%%%%%%%%%%%%%%%%%%%%%%%%%%%%%%%%%%%%%%%%%%%%%%%%%%%%
\begin{frame}[fragile]{Inside the Volume}
 

Let $s(\br, E, \bO, t)$ be a source density, so that $s(\cdots)\,\dV\,\dE\,\dO$ is the number of neutrons 
added to the volume per second. Then
\begin{equation*}
\tag{4--26}
 \textcolor{wcprimary}{\text{(1)}  = \left[ \int_V s(\br,E,\bO,t)\,\dV \right] \dE\,\dO} \, .
\end{equation*}

\vfill 

Given $\Sigma_t(\br,E)$,
the total interaction rate in the volume is
\begin{equation*}
\tag{4--29}
  \textcolor{wcalerted}{\text{(5)} = \left[ \int_V \Sigma_t(\br,E)\, \angflux(\br,E,\bO,t)\,\dV \right] \dE\,\dO } \, .
\end{equation*}
This assumes an {\it isotropic} medium in which $\Sigma_t$ does {\it not} depend on $\bO$.
Finally, {\it in-scattering} from $(E',\bO')$ to $(E,\bO)$ contributes
\begin{equation}
\tag{4--31}
   \textcolor{wcprimary}{\text{(3)} = \int_V \left[ \int_0^\infty \!\!\int_{4\pi} \Sigma_s(\br, E'\!\to\!E, \bO'\!\to\!\bO)\,
  \angflux(\br, E', \bO', t) \dO' \dE' \right] \dV } \, .
\end{equation}

\end{frame}

\begin{frame}[fragile]{Through the Surface}

 The rate at which neutrons are gained or lost through a surface element is
\begin{equation}
\tag{4--32}
  \jj(\br,E,\bO,t)\cdot \ds \;=\; \bO\, \angflux(\br,E,\bO,t)\cdot \ds.
\end{equation}
Integrating over the entire surface, the {\it net} number \emph{leaving} is
\begin{equation*}
   \textcolor{wcalerted}{\text{(4)}}  - \textcolor{wcprimary}{\text{(2)}} = \int_S \bO\, \angflux(\br,E,\bO,t)\cdot \ds .
  \tag{4--33}
\end{equation*}
Recall Gauss' (divergence) theorem, i.e.,
\begin{equation*}
\tag{4--34}
  \int_S \mathbf{F}(\br)\cdot \ds \;=\; \int_V \nabla\!\cdot\! \mathbf{F}(\br)\, \dV \, ,
\end{equation*}
which, because $\bO$ is independent of $\br$, lets us write
\begin{equation*}
 \textcolor{wcalerted}{\text{(4)}}  - \textcolor{wcprimary}{\text{(2)}} 
 = \int_S \bO\, \angflux\cdot \ds
  \;=\; \int_V \nabla\!\cdot\!\big(\bO\, \angflux\big)\, \dV
  \;=\; \textcolor{wcalerted}{ \int_V \bO\cdot \nabla \angflux \, \dV } \, .
\end{equation*}

\end{frame}


\begin{frame}[fragile]{Shrinking $V$ for Differential Balance}

If $V$ is arbitrary, then we collect the terms, take $V \to 0$ to eliminate the integrals,
and welcome our new friend, the {\bf neutron transport equation}:
\begin{align*}
  \frac{1}{v}\,\frac{\partial \angflux}{\partial t}
  +  \textcolor{wcalerted}{ \bO\cdot\nabla \angflux}
  &+   \textcolor{wcalerted}{ \Sigma_t(\br,E)\, \angflux(\br,E,\bO,t)}
   = \\
  & \textcolor{wcprimary}{\int_0^\infty \!\!\int_{4\pi}
  \Sigma_s(\br, E'\!\to\!E, \bO'\!\to\!\bO)\,
  \angflux(\br,E',\bO',t)\, \dO'\, \dE'} \\
  & + \textcolor{wcprimary}{ s(\br,E,\bO,t) }.
  \tag{4--40}
\end{align*}

\vfill 
{\bf Food for thought}: what term would we add to include multiplication (fission)?

\end{frame}


\begin{frame}[fragile]{Eliminating $\bO$}

While scary, the various terns comprising the NTE are familiar enough, but the dependence on $\bO$ remains a challenge for another day.  Integrating the NTE over $4 \pi$ leads to

\begin{align*}
  \frac{1}{v}\,\frac{\partial \flux}{\partial t}
  &+ \nabla\!\cdot\! \JJ(\br,E,t)
   +\Sigma_t(\br,E)\, \flux(\br,E,t)\\
  &=
  \int_0^\infty \Sigma_s(\br, E'\!\to\!E)\, \flux(\br,E',t)\, \dE'
  + S(\br,E,t) \, ,
  \tag{4--79}
\end{align*}
which is called the {\bf neutron continuity equation} (or NCE). 

\vfill 
\pause  

{\bf Ponderable}.  Have we simplified our life? \pause
Unfortunately, we went from the NTE with one unknown ($\psi$) to the new NCE 
with two unknowns ($\phi$ and $\mathbf{J}$)!


\end{frame}

\begin{frame}[fragile]{Finding Closure}

We can't really do much with one equation and the unknowns $\phi$ and $\mathbf{J}$.  
The fix is a simple one: assume that the flow points where the neutron density, or,
rather, flux falls:
\begin{equation*}
   \mathbf{J} \propto -\nabla \phi \, .
\end{equation*}
{\bf Ponderable}: How do we go from $\propto$ to $=$?
\pause 
Let $D$ be that constant of proportionality so that
\begin{equation*}
\tag{FNRP 6.11}
   \mathbf{J} = -D \nabla \phi \, ,
\end{equation*}
which is known as {\bf Fick's law}.  Substitution of Fick's law 
into the NCE yields

\begin{align*}
  \frac{1}{v}\,\frac{\partial \flux}{\partial t}
  &- \nabla\!\cdot\! D(\br,E) \nabla \phi(\br,E,t)
   +\Sigma_t(\br,E)\, \flux(\br,E,t)\\
  &=
  \int_0^\infty \Sigma_s(\br, E'\!\to\!E)\, \flux(\br,E',t)\, \dE'
  + S(\br,E,t) \, ,
\end{align*}
which is, of course, a long lost friend!  

\end{frame}


\begin{frame}[fragile]{Simplify, Simplify}

Now that we have space, it's time to say goodbye to $t$ and to 
replace $E$ by $g$:

\begin{align*}
 - \nabla\!\cdot\! D_g(\br) \nabla \phi_g(\br)
   + \Sigma_{tg}(\br)\, \flux_g(\br)
  = \sum_{g'=1}^{G} \Sigma_{sg\gets g'} \phi_{g'}(\br) +  S_g(\br) \, .
\end{align*}

{\bf Example}: For $G=2$ and no upscatter, show that
\begin{align*}
 - \nabla\!\cdot\! D_1(\br) \nabla \phi_1(\br)
   + \Sigma_{r,1}(\br)\, \flux_1(\br)
  &=  S_1(\br,t) \\ 
 - \nabla\!\cdot\! D_2(\br) \nabla \phi_2(\br)
   + \Sigma_{a,2}(\br)\, \flux_2(\br)
  &=  \Sigma_{s,2\gets 1} \phi_{1}(\br) + S_2(\br) 
\end{align*}
What if $G=1$ and we introduce multiplication?
\pause 

\begin{align*}
\tag{FNRP 6.12}
\boxed{ - \nabla\!\cdot\! D(\br) \nabla \phi(\br)
   + \Sigma_{a}(\br)\, \flux(\br) 
  =  S(\br) + \bar{\nu}\Sigma_{f}(\br)\, \flux(\br) 
}
\end{align*}

\end{frame}

\begin{frame}[fragile]{Onto Slab Geometry}

Much like integration over $4\pi$ hides some detail, so, too, does $\nabla\!\cdot\! D(\br) \nabla \phi(\br)$.  Recall that, for Cartesian (or $xyz$) coordinates, the gradient and divergence operators act on scalar and vector functions according to
\begin{align*}
\nabla \phi(\mathbf{r})
= \mathbf{i} \,\frac{\partial \phi}{\partial x}
 + \mathbf{j}  \,\frac{\partial \phi}{\partial y}
 + \mathbf{k}  \,\frac{\partial \phi}{\partial z} \quad \text{and} \quad
\nabla\!\cdot\!\mathbf{J}(\mathbf{r})
= \frac{\partial J_x}{\partial x}
 + \frac{\partial J_y}{\partial y}
 + \frac{\partial J_z}{\partial z} \, ,
\end{align*}
so that
\begin{align*}
\nabla\!\cdot\!\big(D(\mathbf{r})\,\nabla \phi(\mathbf{r})\big)
&= \frac{\partial}{\partial x}\!\left(D\,\frac{\partial \phi}{\partial x}\right)
 + \frac{\partial}{\partial y}\!\left(D\,\frac{\partial \phi}{\partial y}\right)
 + \frac{\partial}{\partial z}\!\left(D\,\frac{\partial \phi}{\partial z}\right) \\[4pt]
&= D(\mathbf{r})\,\nabla^2 \phi(\mathbf{r})
  + \big(\nabla D(\mathbf{r})\big)\!\cdot\!\big(\nabla \phi(\mathbf{r})\big).
\end{align*}

If we make the bold assumption that $\phi(\mathbf{r}) = \phi(x)$, we find 
\begin{align*}
\boxed{ - \frac{d}{dx} \left ( D(x) \frac{d\phi}{dx}   \right )
   + \Sigma_{a}(x)\, \flux(x) 
  =  \bar{\nu}\Sigma_{f}(x)\, \flux(x) +  S(x) 
} \, ,
\end{align*}
the one-speed diffusion equation in slab geometry.
\end{frame}




\end{document}

