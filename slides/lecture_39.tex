\documentclass[aspectratio=1610]{beamer}
\usepackage[T1]{fontenc}
\usetheme{wildcat}
\usetikzlibrary{arrows.meta,angles,quotes,calc,intersections,positioning}

\usepackage{amsmath,amssymb,amsfonts}
\usepackage{booktabs}
\usepackage{relsize}
\usepackage{pgfplots}
\pgfplotsset{compat=1.16}
\usepackage{array}
\usepackage{siunitx}
\usepackage{cancel}
\usepackage{jupyter}
\usepackage{minted}
\usepackage{xfrac}
\usepackage{mathtools}

\let\oldfootnotesize\footnotesize
\renewcommand*{\footnotesize}{\oldfootnotesize\tiny}

\def\mathdefault#1{#1}
\everymath=\expandafter{\the\everymath\displaystyle}


\title{Neutrons in Space \\
       {\small\it NE 630 - Lecture 31}}

\date{\input{term.txt} \\ {\footnotesize Git SHA: \input{git_sha.txt}}}

\author{Jeremy Roberts}


\definecolor{ksupurple}{HTML}{512888}
\definecolor{orange}{HTML}{CA7C1B}
\definecolor{skyblue}{RGB}{180,220,255}

% --- Shortcuts ---
\newcommand{\br}{\mathbf{r}}
\newcommand{\bO}{\hat{\Omega}}
\newcommand{\bn}{\hat{\mathbf{n}}}
\newcommand{\JJ}{\mathbf{J}}
\newcommand{\jj}{\mathbf{j}}
\newcommand{\flux}{\phi}
\newcommand{\angflux}{\psi}
\newcommand{\dV}{\,d^3\br}
\newcommand{\dE}{\,dE}
\newcommand{\dO}{\,d\hat{\Omega}}
\newcommand{\ds}{\,d\mathbf{S}}


\begin{document}

\begin{frame}
\titlepage
\end{frame}
 
 
%%%%%%%%%%%%%%%%%%%%%%%%%%%%%%%%%%%%%%%%%%%%%%%%%%%
\begin{frame}{Primary Objective}

Students will be able to 

\vfill

\begin{quote}
\textcolor{wcprimary}{show that neutrons are conserved (or not!) in a spatial 
volume given the neutron flux and appropriate cross-section data.
}
\end{quote}

\vfill 

These notes represent a condensed summary of D\&H, Ch.~4 with connections 
to Ch.~6 of FNRP and the diffusion equation.  Equations numbered like 
(4--8) are from D\&H, but I've modified nomenclature slightly in some 
places to match what we've previously adopted.


\end{frame}


%%%%%%%%%%%%%%%%%%%%%%%%%%%%%%%%%%%%%%%%%%%%%%%%%%%%%%%%%%%%%%%%%%%%%%%%%%%%%%

\begin{frame}[fragile]{Characterizing Neutron Fields}
 
Let the \textbf{angular density} of neutrons be described by
\begin{equation}
  n(\br, E, \bO, t)\, \dV\, \dE\, \dO \quad [ \si{\per\centi\meter\cubed\per\electronvolt\per\steradian} ] \, ,
  \tag{4--8}
\end{equation}
which is the expected number of neutrons in the differential \emph{phase-space} volume
centered at $(\br, E, \bO)$ at time $t$. The direction unit vector is
$\bO = \mathbf{v}/\lVert\mathbf{v}\rVert$ where $\mathbf{v}$ is the neutron velocity.
 
% FIG 1. PHASE SPACE diagram

\vfill

The \textbf{angular flux} and \textbf{angular current} are
\begin{align}
  \tag{4--9}
  \angflux(\br, E, \bO, t) = v\, n(\br, E, \bO, t)  \quad  [\si{\per\centi\meter\squared\per\electronvolt\per\steradian\per\second}] \, ,
\end{align}
and
\begin{align}
  \tag{4--10}
  \jj(\br, E, \bO, t) = \bO\, \angflux(\br, E, \bO, t) \quad  [\si{\per\centi\meter\squared\per\electronvolt\per\steradian\per\second}] \, ,
\end{align}
where $v = \lVert\mathbf{v}\rVert$ is the neutron speed.

% FIG 2. ANGULAR CURRENT diagram


\end{frame}

%%%%%%%%%%%%%%%%%%%%%%%%%%%%%%%%%%%%%%%%%%%%%%%%%%%%%%%%%%%%%%%%%%%%%%%%%%%%%%
\begin{frame}[fragile]{Scalar Quantities}


Integrating over all angles gives the \textbf{scalar density} and \textbf{scalar flux}
\begin{align}
  n(\br,E,t) &= \int_{4\pi} n(\br, E, \bO, t)\, \dO, 
  \tag{4--14}\\
  \flux(\br,E,t) &= \int_{4\pi} \angflux(\br, E, \bO, t)\, \dO.
  \tag{4--15}
\end{align}
\vfill 

The \textbf{current density} (or just \textbf{current}) is
\begin{equation}
  \tag{4--19}
  \JJ(\br,E,t) \;=\; \int_{4\pi} \jj(\br, E, \bO, t)\, \dO,
\end{equation}
so that $\JJ\!\cdot d\mathbf{A}$ is the net rate at which neutrons pass through a surface element~$d\mathbf{A}$ per unit energy at time $t$.
Although $\JJ$ and $\flux$ have the same units, $\JJ$ is a vector and represents net flow.




\end{frame}



%%%%%%%%%%%%%%%%%%%%%%%%%%%%%%%%%%%%%%%%%%%%%%%%%%%%%%%%%%%%%%%%%%%%%%%%%%%%%%
\begin{frame}[fragile]{Integration over $4\pi$}
 
Integrals of the form 
\begin{equation*}
 \int_{4\pi} \ldots \dO
\end{equation*}
are sometimes hard to imagine.

\vfill 
\pause 

An explicit example, in {\it spherical coordinates}, might be
this solid-angle integral:
\begin{equation*}
  \int_{4\pi} f(\bO)\, \dO = \int_{0}^{2\pi}\!\!\int_{0}^{\pi} f(\theta,\phi)\, \sin\theta\, d\theta\, d\phi.
\end{equation*}


\end{frame} 

%%%%%%%%%%%%%%%%%%%%%%%%%%%%%%%%%%%%%%%%%%%%%%%%%%%%%%%%%%%%%%%%%%%%%%%%%%%%%%
\begin{frame}[fragile]{The Balancing Act}

Consider some volume $V$ with boundary surface $S$ and outward normal~$d\mathbf{s}$.
The total number of neutrons in $V$ with energies in $(E,E{+}dE)$ and directions in $\dO$ about $\bO$ is
\begin{equation*}
  N(t) = \int_V n(\br, E, \bO, t)\, \dV \; \dE\, \dO.
\end{equation*}
\vfill 

If $V$ does not depend on time, then differentiating $N(t)$ with respect to time gives
\begin{equation}
  \int_V \frac{\partial n}{\partial t}(\br, E, \bO, t)\, \dV \; \dE\,\dO
  = \textcolor{wcprimary}{\text{{\bf gains}}} - \textcolor{wcalerted}{\text{{\bf losses}}} \, .
  \tag{4--24/25}
\end{equation}

\vfill 
\pause 

\begin{tabular}{ll}
 \textcolor{wcprimary}{(1) any sources of n's}  &  \\
 \textcolor{wcprimary}{(2) n's ``stream'' into $V$ through $S$} &
       \textcolor{wcalerted}{(4) n's ``stream'' out of $V$ through $S$} \\
 \textcolor{wcprimary}{(3) n's  scatter from $(E',\bO')$ to $(E,\bO)$}  & 
       \textcolor{wcalerted}{(5) n's that experience \emph{any} interaction}.
\end{tabular}


\end{frame}

%%%%%%%%%%%%%%%%%%%%%%%%%%%%%%%%%%%%%%%%%%%%%%%%%%%%%%%%%%%%%%%%%%%%%%%%%%%%%%
\begin{frame}[fragile]{Inside the Volume}
 

Let $s(\br, E, \bO, t)$ be a source density, so that $s(\cdots)\,\dV\,\dE\,\dO$ is the number of neutrons 
added to the volume per second. Then
\begin{equation*}
\tag{4--26}
 \textcolor{wcprimary}{\text{(1)}  = \left[ \int_V s(\br,E,\bO,t)\,\dV \right] \dE\,\dO} \, .
\end{equation*}

\vfill 

Given $\Sigma_t(\br,E)$,
the total interaction rate in the volume is
\begin{equation*}
\tag{4--29}
  \textcolor{wcalerted}{\text{(5)} = \left[ \int_V \Sigma_t(\br,E)\, \angflux(\br,E,\bO,t)\,\dV \right] \dE\,\dO } \, .
\end{equation*}
This assumes an {\it isotropic} medium in which $\Sigma_t$ does {\it not} depend on $\bO$.
Finally, {\it in-scattering} from $(E',\bO')$ to $(E,\bO)$ contributes
\begin{equation}
\tag{4--31}
   \textcolor{wcprimary}{\text{(3)} = \int_V \left[ \int_0^\infty \!\!\int_{4\pi} \Sigma_s(\br, E'\!\to\!E, \bO'\!\to\!\bO)\,
  \angflux(\br, E', \bO', t) \dO' \dE' \right] \dV } \, .
\end{equation}

\end{frame}

\begin{frame}[fragile]{Through the Surface}

 The rate at which neutrons are gained or lost through a surface element is
\begin{equation}
\tag{4--32}
  \jj(\br,E,\bO,t)\cdot \ds \;=\; \bO\, \angflux(\br,E,\bO,t)\cdot \ds.
\end{equation}
Integrating over the entire surface, the {\it net} number \emph{leaving} is
\begin{equation*}
   \textcolor{wcalerted}{\text{(4)}}  - \textcolor{wcprimary}{\text{(2)}} = \int_S \bO\, \angflux(\br,E,\bO,t)\cdot \ds .
  \tag{4--33}
\end{equation*}
Recall Gauss' (divergence) theorem, i.e.,
\begin{equation*}
\tag{4--34}
  \int_S \mathbf{F}(\br)\cdot \ds \;=\; \int_V \nabla\!\cdot\! \mathbf{F}(\br)\, \dV \, ,
\end{equation*}
which, because $\bO$ is independent of $\br$, lets us write
\begin{equation*}
 \textcolor{wcalerted}{\text{(4)}}  - \textcolor{wcprimary}{\text{(2)}} 
 = \int_S \bO\, \angflux\cdot \ds
  \;=\; \int_V \nabla\!\cdot\!\big(\bO\, \angflux\big)\, \dV
  \;=\; \textcolor{wcalerted}{ \int_V \bO\cdot \nabla \angflux \, \dV } \, .
\end{equation*}

\end{frame}


\begin{frame}[fragile]{Shrinking $V$ for Differential Balance}

If $V$ is arbitrary, then we collect the terms, take $V \to 0$ to eliminate the integrals,
and welcome our new friend, the {\bf neutron transport equation}:
\begin{align*}
  \frac{1}{v}\,\frac{\partial \angflux}{\partial t}
  +  \textcolor{wcalerted}{ \bO\cdot\nabla \angflux}
  &+   \textcolor{wcalerted}{ \Sigma_t(\br,E)\, \angflux(\br,E,\bO,t)}
   = \\
  & \textcolor{wcprimary}{\int_0^\infty \!\!\int_{4\pi}
  \Sigma_s(\br, E'\!\to\!E, \bO'\!\to\!\bO)\,
  \angflux(\br,E',\bO',t)\, \dO'\, \dE'} \\
  & + \textcolor{wcprimary}{ s(\br,E,\bO,t) }.
  \tag{4--40}
\end{align*}

\vfill 
{\bf Food for thought}: what term would we add to include multiplication (fission)?

\end{frame}


\begin{frame}[fragile]{Eliminating $\bO$}

While scary, the various terns comprising the NTE are familiar enough, but the dependence on $\bO$ remains a challenge for another day.  Integrating the NTE over $4 \pi$ leads to

\begin{align*}
  \frac{1}{v}\,\frac{\partial \flux}{\partial t}
  &+ \nabla\!\cdot\! \JJ(\br,E,t)
   +\Sigma_t(\br,E)\, \flux(\br,E,t)\\
  &=
  \int_0^\infty \Sigma_s(\br, E'\!\to\!E)\, \flux(\br,E',t)\, \dE'
  + S(\br,E,t) \, ,
  \tag{4--79}
\end{align*}
which is called the {\bf neutron continuity equation} (or NCE). 

\vfill 
\pause  

{\bf Ponderable}.  Have we simplified our life? \pause
Unfortunately, we went from the NTE with one unknown ($\psi$) to the new NCE 
with two unknowns ($\phi$ and $\mathbf{J}$)!


\end{frame}

\begin{frame}[fragile]{Finding Closure}

We can't really do much with one equation and the unknowns $\phi$ and $\mathbf{J}$.  
The fix is a simple one: assume that the flow points where the neutron density, or,
rather, flux falls:
\begin{equation*}
   \mathbf{J} \propto -\nabla \phi \, .
\end{equation*}
{\bf Ponderable}: How do we go from $\propto$ to $=$?
\pause 
Let $D$ be that constant of proportionality so that
\begin{equation*}
\tag{FNRP 6.11}
   \mathbf{J} = -D \nabla \phi \, ,
\end{equation*}
which is known as {\bf Fick's law}.  Substitution of Fick's law 
into the NCE yields

\begin{align*}
  \frac{1}{v}\,\frac{\partial \flux}{\partial t}
  &- \nabla\!\cdot\! D(\br,E) \nabla \phi(\br,E,t)
   +\Sigma_t(\br,E)\, \flux(\br,E,t)\\
  &=
  \int_0^\infty \Sigma_s(\br, E'\!\to\!E)\, \flux(\br,E',t)\, \dE'
  + S(\br,E,t) \, ,
\end{align*}
which is, of course, a long lost friend!  

\end{frame}


\begin{frame}[fragile]{Simplify, Simplify}

Now that we have space, it's time to say goodbye to $t$ and to 
replace $E$ by $g$:

\begin{align*}
 - \nabla\!\cdot\! D_g(\br) \nabla \phi_g(\br)
   + \Sigma_{tg}(\br)\, \flux_g(\br)
  = \sum_{g'=1}^{G} \Sigma_{sg\gets g'} \phi_{g'}(\br) +  S_g(\br) \, .
\end{align*}

{\bf Example}: For $G=2$ and no upscatter, show that
\begin{align*}
 - \nabla\!\cdot\! D_1(\br) \nabla \phi_1(\br)
   + \Sigma_{r,1}(\br)\, \flux_1(\br)
  &=  S_1(\br,t) \\ 
 - \nabla\!\cdot\! D_2(\br) \nabla \phi_2(\br)
   + \Sigma_{a,2}(\br)\, \flux_2(\br)
  &=  \Sigma_{s,2\gets 1} \phi_{1}(\br) + S_2(\br) 
\end{align*}
What if $G=1$ and we introduce multiplication?
\pause 

\begin{align*}
\tag{FNRP 6.12}
\boxed{ - \nabla\!\cdot\! D(\br) \nabla \phi(\br)
   + \Sigma_{a}(\br)\, \flux(\br) 
  =  S(\br) + \bar{\nu}\Sigma_{f}(\br)\, \flux(\br) 
}
\end{align*}

\end{frame}

\begin{frame}[fragile]{Onto Slab Geometry}

Much like integration over $4\pi$ hides some detail, so, too, does $\nabla\!\cdot\! D(\br) \nabla \phi(\br)$.  Recall that, for Cartesian (or $xyz$) coordinates, the gradient and divergence operators act on scalar and vector functions according to
\begin{align*}
\nabla \phi(\mathbf{r})
= \mathbf{i} \,\frac{\partial \phi}{\partial x}
 + \mathbf{j}  \,\frac{\partial \phi}{\partial y}
 + \mathbf{k}  \,\frac{\partial \phi}{\partial z} \quad \text{and} \quad
\nabla\!\cdot\!\mathbf{J}(\mathbf{r})
= \frac{\partial J_x}{\partial x}
 + \frac{\partial J_y}{\partial y}
 + \frac{\partial J_z}{\partial z} \, ,
\end{align*}
so that
\begin{align*}
\nabla\!\cdot\!\big(D(\mathbf{r})\,\nabla \phi(\mathbf{r})\big)
&= \frac{\partial}{\partial x}\!\left(D\,\frac{\partial \phi}{\partial x}\right)
 + \frac{\partial}{\partial y}\!\left(D\,\frac{\partial \phi}{\partial y}\right)
 + \frac{\partial}{\partial z}\!\left(D\,\frac{\partial \phi}{\partial z}\right) \\[4pt]
&= D(\mathbf{r})\,\nabla^2 \phi(\mathbf{r})
  + \big(\nabla D(\mathbf{r})\big)\!\cdot\!\big(\nabla \phi(\mathbf{r})\big).
\end{align*}

If we make the bold assumption that $\phi(\mathbf{r}) = \phi(x)$, we find 
\begin{align*}
\boxed{ - \frac{d}{dx} \left ( D(x) \frac{d\phi}{dx}   \right )
   + \Sigma_{a}(x)\, \flux(x) 
  =  \bar{\nu}\Sigma_{f}(x)\, \flux(x) +  S(x) 
} \, ,
\end{align*}
the one-speed diffusion equation in slab geometry.
\end{frame}




\end{document}

